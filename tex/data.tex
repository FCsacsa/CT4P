
\section{Initial Algebras and Inductive Datatypes}
\label{sec:initial-algs}

\begin{reading*}
  This chapter is strongly inspired by Varmo Vene's Ph.D.\ thesis \cite[Chapter 2]{vene_phd}.
\end{reading*}

In this section we introduce (initial) algebras which allows us to define inductive data types.

\begin{dfn} Let $F:\CC\to \CC$ be an endofunctor. An \textbf{$F$-algebra} consists of the following data:
\begin{enumerate}
\item An object $C\in \CC$.
\item A morphism $\phi \in \CHom{\CC}{F(C)}{C}$.
\end{enumerate}
\end{dfn}

\begin{intu} An algebra is loosely speaking a set with some \textit{algebraic} operations like multiplication. A simple example is that of a monoid. The following example shows that (the data of) monoids correspond precisely with algebras over a fixed endofunctor.
\end{intu}


\begin{exa}
 Let $\Maybe : \SET \to \SET$ be the endofunctor given on objects by $\Maybe(X) := 1 + X$. 
 A $\Maybe$-algebra is a pair $(X,\phi)$ of a set $X$ and a function $\phi : 1 + X \to X$.
 By the universal property of the coproduct, $\phi$ is given, equivalently,
 by two functions $z$ and $s$ as follows.
 \begin{align*}
    z : 1 &\to X
    \\
    s : X &\to X
 \end{align*}
Here, think of ``$z$'' standing for ``zero'', and ``$s$'' standing for ``successor''.
\end{exa}


\begin{exa}\label{example:algebra_of_monoids} Let $F$ be the endofunctor induced by:
\[
F: \SET \to \SET: X \mapsto 1 + (X\times X).
\]
An $F$-algebra consists of a set $X\in\SET$ together with a function $\phi:\CHom{\SET}{1\sqcup (X\times X)}{X}$. Since $\phi$ is function from the disjoint union, we have that $\phi$ corresponds uniquely to two functions:
\begin{align*}
e &: 1\to X
\\
m &: X\times X\to X.
\end{align*}
This is precisely the data of a monoid.

Conversely, if $(M,m,e)$ is a monoid, then this induces a function
\[
1 + (M\times M) \xrightarrow{\phi} M,
\]
which is defined by 
\begin{align*}
\phi\vert_{1} : 1 &\to M
\\
\star &\mapsto e
\\
\phi\vert_{M\times M} : M\times M &\to M 
\\
(x,y) &\mapsto m(x,y)
\end{align*}
The pair $(M, \phi)$ is an $F$-algebra.

\end{exa}

\begin{rem}
 Note that a monoid $(M,m,e)$ also satisfies some laws.
 The laws are not expressed in \cref{example:algebra_of_monoids}.
 To incorporate the laws, one studies instead algebras of a monad.
\end{rem}


\begin{dfn}\label{dfn:alg-hom}
Let $F:\CC\to \CC$ be an endofunctor and $(C,\phi)$ and $(D,\psi)$ be $F$-algebras. 
A \textbf{($F$-algebra) homomorphism} from $(C,\phi)$ to $(D,\psi)$ is a morphism $f\in \CHom{\CC}{C}{D}$ such that the following diagram commutes:
\[
\begin{tikzcd}
F(C) 
\arrow[r, "\phi"] 
\arrow[d,swap, "F(f)"]
& C 
\arrow[d, "f"] 
\\
F(D) 
\arrow[r, swap, "\psi"] 
& D
\end{tikzcd}
\]
\end{dfn}



\begin{exer} 
Let $F$ be the endofunctor defined as in \cref{example:algebra_of_monoids}, 
i.e. the endofunctor whose algebras corresponds with monoids. 
Characterize the $F$-algebra homomorphisms.
\end{exer}

\begin{dfn}\label{definition:category_of_Falgebras} Let $F:\CC\to \CC$ be an endofunctor. The \textbf{category of $F$-algebras}, denoted by $\ALG{F}$, is defined by the following data:
\begin{itemize}
\item The objects are the $F$-algebras.
\item The morphisms are the $F$-algebra homomorphisms.
\item The identity on $(C,\phi)$ is given by the identity $\Id[C]$ in $\CC$.
\item The composition is given by the composition of morphisms in $\CC$.
\end{itemize}
\end{dfn}

\begin{exer} 
Show that $\ALG{F}$ satisfies the properties of a category.
\end{exer}

We are interested in \textbf{initial objects of $\ALG{F}$}, if they exist.
We call these ``initial $F$-algebras''.
For a general endofunctor $F$, an initial $F$-algebra does not exist;
but for many interesting choices of $F$, such an initial object does exist.
Before coming to the general definition (see \cref{dfn:initial-alg}),
we consider an example.


\begin{exer} Consider the functor $\Maybe : \SET \to \SET$.
Show that the initial $\Maybe$-algebra is given by the pair $(\NN, [\Zero,\Succ])$, 
where $\NN$ is the set of natural numbers, and $\Zero : 1 \to \NN$ and $\Succ : \NN\to\NN$ 
are the function picking out zero and the successor function, respectively.

Given any other $\Maybe$-algebra $(X,[z,s])$, unpack what it means for the square of \cref{dfn:alg-hom} to commute.
Compare it to a definition of a function $f : \NN \to X$ by pattern matching.

\end{exer}



\begin{dfn}\label{dfn:initial-alg}
Let $F:\CC\to\CC$ be an endofunctor. The \textbf{initial $F$-algebra} (if it exists), is the initial object in $\ALG{F}$, i.e. it is an $F$-algebra $(\mu F, in)$ such that for any $F$-algebra $(C,\phi)$, there exists a unique morphism $\catam{\phi} \in \CHom{\CC}{\mu F}{C}$ such that the following diagram commutes:
\[
\begin{tikzcd}
{F(\mu F)} 
\arrow[r, "in"] 
\arrow[d,swap, "F(\catam{\phi})"]
& {\mu F} 
\arrow[d, "\catam{\phi}"] 
\\
F(C) 
\arrow[r, swap, "\phi"] 
& C
\end{tikzcd}
\]
A morphism of the form $\catam{\phi}$ is called a \textit{catamorphism}.
\end{dfn}


\begin{exer} Let $F:\CC\to\CC$ be an endofunctor and assume that the initial algebra $(\mu F, in)$ exists. Show that the following properties holds:
\begin{enumerate}
\item $\Id = \catam{in}$.
\item For $F$-algebras $(C,\phi)$ and $(D,\psi)$ and $f\in\CHom{\CC}{C}{D}$, we have 
\[
\phi\Comp f = F(f)\Comp \psi \implies \catam{\phi}\Comp f = \catam{\psi}.
\]
\end{enumerate} 
\end{exer}

\begin{exer} (\textbf{Lambek's theorem}) Let $F:\CC\to\CC$ be an endofunctor and assume that the initial algebra $(\mu F, in)$ exists. Then is $in$ an isomorphism whose inverse is given by $\Inv{in} = \catam{F(in)}$.
\end{exer}

\begin{exer}\label{exercise:initialalg_for_idfun_with_initialob} Let $\CC$ be a category with an initial object $\bot$. Show that $(\bot, \Id[\bot])$ is the initial algebra for the identity (endo)functor on $\CC$.
\end{exer}


\begin{exer}\label{exercise:initialalg_for_list} Let $A$ be a set and define $F$ to be the functor induced by 
\[
F_A:\SET\to\SET : X\mapsto 1 \sqcup (A\times X).
\]
Show that the initial $F_A$-algebra is given by the set $List(A)$. of $A$-valued lists.
\end{exer}

\begin{exer}\label{exercise:initialalg_for_btree} Let $A$ be a set and define $F$ to be the functor induced by 
\[
F_A:\SET\to\SET : X\mapsto A \sqcup (X\times X).
\]
Show that the initial $F_A$-algebra is given by the set $BTree(A)$ of $A$-valued binary trees.
\end{exer}

\begin{rem} Notice that in \cref{exercise:initialalg_for_list} and \cref{exercise:initialalg_for_btree}, we can consider the functor $F_A$ as a bifunctor where we vary $A$, i.e.
\[
F: \SET\to \SET\to \SET: (A,X)\mapsto F_A(X).
\]
In particular, under the assumption that for every $A\in\SET$ the initial $F_A$-algebra exists, we can wonder if the assignment 
\[
\Ob{\SET} \to \Ob{\SET} : A\mapsto \mu F_A ,
\]
can be extended to a functor. The following exercise answers this question positively for arbitrary categories.
\end{rem}

\begin{exer} Let $F:\CC\to\CC\to\CC$ be a bifunctor such that for any object $A\in\CC$, the initial algebra for the functor induced by 
\[
F_A : \CC\to\CC : X\mapsto F(A,X),
\]
exists. Show how
\[
\Ob{\CC} \to \Ob{\CC} : A\mapsto \mu F_A ,
\]
induces a functor.
\end{exer}


\section{Terminal Coalgebras and Coinductive Datatypes}
In this section we introduce (terminal) coalgebras which allows us to define coinductive data types.

\begin{dfn} Let $F:\CC\to \CC$ be an endofunctor. An \textbf{$F$-coalgebra} consists of the following data:
\begin{itemize}
\item An object $C\in \CC$.
\item A morphism $\phi \in \CHom{\CC}{C}{F(C)}$.
\end{itemize}
\end{dfn}
Notice that an $F$-algebra consists of a morphism $F(C)\to C$, while an $F$-coalgebra consists of a morphism $C\to F(C)$ in the other direction.

\begin{dfn} Let $F:\CC\to \CC$ be an endofunctor and $(C,\phi)$ and $(D,\psi)$ be $F$-coalgebras. A \textbf{($F$-algebra) homomorphism} from $(C,\phi)$ to $(D,\psi)$ is a morphism $f\in \CHom{\CC}{C}{D}$ such that the following diagram commutes:
\begin{center}
\begin{tikzcd}
C 
\arrow[r, "\phi"] 
\arrow[d,swap, "f"]
& F(C) 
\arrow[d, "F(f)"] 
\\
D
\arrow[r, swap, "\psi"] 
& F(D) 
\end{tikzcd}
\end{center}
\end{dfn}

\begin{exer} Define the category $\COALG{F}$ of $F$-coalgebras analogously to the category $\ALG{F}$ of $F$-algebras (as in \cref{definition:category_of_Falgebras}).
\end{exer}

\begin{dfn} Let $F$ be an endofunctor on $\CC$. The \textbf{terminal $F$-coalgebra} is the terminal object in $\COALG{F}$ which we denote by $(\nu F, out)$.\\
For $(C,\phi)$ an arbitrary $F$-coalgebra, we denote the unique morphism $C\to \nu F$ by $\anam{\phi}$. and we call a morphism of this form an \textit{anamorphism} (instead of a catamorphism).
\end{dfn}

\begin{exer} Spell out what it means for a coalgebra to be the terminal coalgebra.
\end{exer}

\begin{exer} Let $F:\CC\to\CC$ be an endofunctor and assume that the terminal coalgebra $(\nu F, out)$ exists. Show that the following properties holds:
\begin{enumerate}
\item $\Id = \anam{out}$.
\item For $F$-algebras $(C,\phi)$ and $(D,\psi)$ and $f\in\CHom{\CC}{C}{D}$, we have 
\[
f\Comp\psi  = \phi\Comp F(f) \implies f\Comp\anam{\psi} = anam{\phi}.
\]
\end{enumerate} 
\end{exer}

\begin{exer} (\textbf{Dual Lambek's theorem}) Let $F:\CC\to\CC$ be an endofunctor and assume that the terminal coalgebra $(\nu F, out)$ exists. Then is $out$ an isomorphism whose inverse is given by $\Inv{out} = \anam{F(out)}$.
\end{exer}

\begin{exer}\label{exercise:terminalalg_for_idfun_with_terminalob} Let $\CC$ be a category with an terminal object $\top$. Then is $(\top, \Id[\top])$ the terminal coalgebra for the identity (endo)functor on $\CC$.
\end{exer}

\begin{exer} Let $F$ be the functor induced by 
\[
F:\SET\to\SET : X\mapsto 1 \sqcup X.
\]
Show that the terminal $F$-coalgebra is given by the following data:
\begin{itemize}
\item The underlying object is given by the set $\mathbb{N} \sqcup \{\infty\}$ of natural numbers with infinity.
\item The underlying function is given by the predecessor defined as follows: 
\begin{align*}
\mathbb{N} \sqcup \{\infty\} &\to 1 \sqcup \mathbb{N} \sqcup \{\infty\},\\
0 &\mapsto \star,\\
s(n) &\mapsto n,\\
\infty &\mapsto \infty,
\end{align*}
where $\star$ is the unique element of $1$.
\end{itemize}
\end{exer}



%%% Local Variables:
%%% mode: latex
%%% TeX-master: "cats-for-programmers"
%%% End:
