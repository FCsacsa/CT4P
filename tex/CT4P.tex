\documentclass[a4paper,10pt]{scrartcl}
\usepackage[utf8]{inputenc}

\usepackage{tikz-cd}
\usepackage{xcolor}
\usepackage{amssymb}
\usepackage{amsmath}
\usepackage{thmtools}
\usepackage{amsthm}
%\usepackage{ntheorem}
\usepackage{stmaryrd} % texlive-science
\usepackage{verbatim}

%%%%%%%%%%%%%%%%%%%%%%%%%%%%%%%%%%%%%%%%%%%%%%%%%%%%%%%%%%%%%%%%%%%%%%%

\newcommand{\onlydraft}[1]{} % DON'T CHANGE THIS LINE

\renewcommand{\onlydraft}[1]{#1}  % Toggle this line to show/hide todo-notes, table of contents, etc.



\usepackage[colorinlistoftodos,prependcaption,textsize=tiny]{todonotes}%to do list and comments

\newcommand{\plan}[1]{}
\newcommand{\BA}[1]{}
\newcommand{\KW}[1]{}
\onlydraft{
   \renewcommand{\plan}[1]{{\color{blue}{#1}}\PackageWarning{TODO}{Plan: #1}}
   \renewcommand{\BA}[1]{\todo[color=orange!30]{BA: #1} \PackageWarning{TODO}{BA: #1}}
   \renewcommand{\KW}[1]{\todo[color=green!30]{KW: #1}\PackageWarning{TODO}{KW: #1}}
}

\newcommand{\issue}[1]{\href{https://github.com/benediktahrens/CT4P/issues/#1}{Issue #1}}
%%%%%%%%%%%%%%%%%%%%%%%%%%%%%%%%%%%%%%%%%%%%%%%%%%%%%%%%%%%%%%%%%%%%%%%

\usepackage[style=numeric,backend=biber]{biblatex}
\addbibresource{literature.bib}

\usepackage{listings}
\lstMakeShortInline|
\lstset{% general command to set parameter(s)
basicstyle=\small\sffamily,
% print whole listing small
keywordstyle=\color{black}\bfseries\underbar,
% underlined bold black keywords
identifierstyle=,
% nothing happens
commentstyle=\color{white}, % white comments
stringstyle=\ttfamily,
% typewriter type for strings
showstringspaces=false,
,
literate={→ }{\ARROW}1
         {∃}{\EX}1
         {∀}{\FORALL}1
         {∧}{\AND}1
         {∨}{\OR}1
         {¬}{\NOT}1
         {⊢}{\TURNSTYLE}1
         {×}{\TIMES}1
         {λ}{\LAMBDA}1
}
% no special string spaces

% \begin{comment}
% \def\lstlanguagefiles{lstlean.tex}
% \lstset{language=lean}
% \lstdefinestyle{leannocolor}{
%    basicstyle={\ttfamily},
%    identifierstyle={\ttfamily},
%    keywordstyle=[1]{\ttfamily},
%    keywordstyle=[2]{\ttfamily},
%    keywordstyle=[3]{\ttfamily},
%    stringstyle={\ttfamily},
%    commentstyle={\ttfamily}
% }
% \end{comment}


\protected\def\ARROW{\ensuremath{\to}}
\DeclareUnicodeCharacter{2192}{\ARROW}
\protected\def\EX{\ensuremath{\exists}}
\DeclareUnicodeCharacter{2203}{\EX}
\protected\def\FORALL{\ensuremath{\forall}}
\DeclareUnicodeCharacter{2200}{\FORALL}
\protected\def\AND{\ensuremath{\wedge}}
\DeclareUnicodeCharacter{2227}{\AND}
\protected\def\OR{\ensuremath{\vee}}
\DeclareUnicodeCharacter{2228}{\OR}
\protected\def\NOT{\ensuremath{\neg}}
\DeclareUnicodeCharacter{00AC}{\NOT}
\protected\def\TURNSTYLE{\ensuremath{\vdash}}
\DeclareUnicodeCharacter{22A2}{\TURNSTYLE}
\protected\def\TIMES{\ensuremath{\times}}
\DeclareUnicodeCharacter{00D7}{\TIMES}
\protected\def\LAMBDA{\ensuremath{\lambda}}
\DeclareUnicodeCharacter{03BB}{\LAMBDA}

% \usepackage{listings}
% \lstset{
%   basicstyle=\footnotesize\ttfamily,
%   breaklines=true,
%   escapechar=!,
%   literate={→ }{\ARROW}1
%            {∃}{\EX}1
%            {∀}{\FORALL}1
%            {∧}{\AND}1
%            {∨}{\OR}1
%            {¬}{\NOT}1
%            {⊢}{\TURNSTYLE}1
%            {×}{\TIMES}1
%            {λ}{\LAMBDA}1
%            ,
%   morekeywords=[1]{example, check, reduce},
%   keywordstyle=[1]{\bfseries\color{dkviolet}},
%   morekeywords=[3]{Type,Prop, variable, variables},
%   keywordstyle=[3]{\bfseries\color{dkgreen}},
%   morekeywords=[2]{assume, reduce, variable},
%   keywordstyle=[2]{\ttfamily\color{red}},
%   keywordstyle=[4]{\bfseries\color{dkblue}},
%   morekeywords=[4]{theorem, inductive,structure,def},


\usepackage{hyperref}
\usepackage[capitalize]{cleveref}



\theoremstyle{plain}
\newtheorem{thm}{Theorem}
\crefname{thm}{theorem}{theorems}
\Crefname{thm}{Theorem}{Theorems}
\newtheorem{prop}[thm]{Proposition}
\newtheorem{cor}[thm]{Corollary}
\newtheorem{lemma}[thm]{Lemma}

\newtheorem*{reading*}{Further Reading}

\theoremstyle{definition}
\newtheorem{rem}[thm]{Remark}
\newtheorem{dfn}[thm]{Definition}
\crefname{dfn}{Definition}{Definitions}
\Crefname{dfn}{Definition}{Definitions}
\newtheorem{que}[thm]{Question}
\newtheorem{exa}[thm]{Example}
\crefname{exa}{Example}{Examples}
\Crefname{exa}{Example}{Examples}
\newtheorem{exer}[thm]{Exercise}
\crefname{exer}{Exercise}{Exercises}
\Crefname{exer}{Exercises}{Exercises}
\newtheorem{app}[thm]{Application}
\newtheorem{intu}[thm]{Intuition}

\newtheorem{nota}[thm]{Notation}
\newtheorem{proposal}[thm]{Proposal}
\newtheorem{goal}[thm]{Goal}
\newtheorem{constr}[thm]{Construction}
\newtheorem{solution}[thm]{Solution}

% \declaretheorem[name=Solution,
% refname={theorem,theorems},
% Refname={Theorem,Theorems}]{solution}

%\Crefname{exer}{Exercise}{Exercises}
%\Crefname{solution}{Solution}{Solutions}

%%%%%%%%%%%%%%%%%%%%%%%%%%%%%%%%%%%%%%%%%%%%%%

\newcommand{\cfont}[1]{\ensuremath{\mathsf{#1}}}
\newcommand{\Cat}[1]{\mathcal{#1}}
\newcommand{\CC}{\Cat{C}}
\newcommand{\DD}{\Cat{D}}
\newcommand{\EE}{\Cat{E}}
\newcommand{\Catb}[1]{\mathbf{#1}}
\newcommand{\List}{\Catb{List}}
\newcommand{\BinTree}{\Catb{BinTree}}
\newcommand{\Maybe}{\Catb{Maybe}}
\newcommand{\SET}{\Catb{Set}}
\newcommand{\PTSET}{\Catb{PtSet}}
\newcommand{\FINSET}{\Catb{FinSet}}
\newcommand{\CAT}{\Catb{Cat}}
\newcommand{\POS}{\Catb{Pos}}
\newcommand{\PRE}{\Catb{Pre}}
\newcommand{\MON}{\Catb{Monoid}}
\newcommand{\HASK}{\Catb{Hask}}
\newcommand{\LEAN}{\Catb{LEAN}}
\newcommand{\ALG}[1]{\Cat{A}lg(#1)}
\newcommand{\COALG}[1]{\Cat{C}o\Cat{A}lg(#1)}
\newcommand{\Ob}[1]{{#1}_0}
\newcommand{\Hom}[3][]{\cfont{hom}_{#1}(#2,#3)}
\newcommand{\CHom}[3]{{#1}(#2,#3)}
\newcommand{\Id}[1][]{\cfont{Id}_{#1}}
\newcommand{\Comp}{\cdot}
\newcommand{\NatTrans}[3]{#1 : #2 \Rightarrow #3}
\newcommand{\op}[1]{\ensuremath{{#1}^\text{op}}}
\newcommand{\inl}{\ensuremath{\iota_l}}
\newcommand{\inr}{\ensuremath{\iota_r}}
\newcommand{\projl}{\ensuremath{\pi_l}}
\newcommand{\projr}{\ensuremath{\pi_r}}


\newcommand{\Initalg}[1]{\ensuremath{\mu^{#1}}}
\newcommand{\Inv}[1]{#1 ^{-1}}
\newcommand{\catam}[1]{\llparenthesis #1 \rrparenthesis}
\newcommand{\anam}[1]{\llbracket #1 \rrbracket} % Different notation for anamorphisms which the author of the thesis uses, but I can't find the symbol which he uses. Actually it looks like he hardcoded [( )] which doesn't look nice. But this comes close to the notation I think.
\newcommand{\In}{\ensuremath{\mathsf{in}}}

\newcommand{\nil}{\ensuremath{\mathsf{nil}}}
\newcommand{\cons}{\ensuremath{\mathsf{cons}}}


\newcommand{\co}[2]{\ensuremath{#2 \circ #1}}

\newcommand{\NN}{\ensuremath{\mathbb{N}}}
\newcommand{\Zero}{\ensuremath{\mathsf{zero}}}
\newcommand{\Succ}{\ensuremath{\mathsf{succ}}}
%%%%%%%%%%%%%%%%%%%%%%%%%%%%%%%%%%%%%%%%%%%%%%


%opening
\title{Category Theory for Programming}
\author{Benedikt Ahrens \and Kobe Wullaert}

\begin{document}

\maketitle

\vspace*{\fill}
We thank Arnoud van der Leer for their contributions to these notes.
\vspace*{\fill}


\newpage
\tableofcontents
\newpage


\section{Introduction}

Category theory is a framework which allows one to formally describe and relate mathematical structures. By a mathematical structure, we mean, informally, a collection of \textit{things} (like types, sets, etc.) and something which transforms one \textit{thing} to another \textit{thing} (like a program, function, etc.).

This framework started as a mathematical theory, but has now proven itself useful also in the world of Computer Science (and beyond). In this course we will introduce the necessary concepts from \textit{category theory} with the goal of understanding its applications in the \textit{realm} of programming.


\subsection{About These Notes}

These notes are not meant to give an exhaustive introduction to category theory.
Instead, the aim is to develop just as much category theory as is necessary to discuss some interesting applications of category theory to computing, specifically, to programming.

Throughout these notes, pointers to other sources, such as textbooks and research articles, are given;
it is highly recommended to consult these sources.

\subsection{About Category Theory}
%\label{sec:about-cats}

\emph{Category theory} is a mathematical area of endeavour and language developed to reconcile and unify mathematical phenomena from different disciplines.
It was developed from the 1940s on, in particular by Samuel Eilenberg and Saunders Mac Lane.

\emph{Computer science} is\ldots well, you know what it is.

In this course, we learn about some fundamental applications of category theory to computer science, specifically, to programming.
The power of category theory arises from \textbf{abstraction}:
by boiling down constructions to their essence, analogous situations can be formally identified using category theory.
One crucial concept provided by category theory to this end is that of \emph{universal property};
we study some universal properties in \cref{sec:universal}.
An application of universal properties to the theory of datatypes and structural recursion is studied in \cref{sec:initial-algs}.

Another categorical concept that has proved particularly useful in programming is that of a \textbf{monad}.
We study monads and their use in programming in \cref{sec:monads}.




\subsection{Learning Material on Category Theory}
\label{sec:material}

The scientific literature on category theory in computer science is vast.
%We will only point to a few books, papers, and articles, for instance to this gem~\cite{goguen_thatcher_wagner_wright_1973}.
We list some learning material on category theory.

Pierce's book \cite{pierce} (available for free) gives a brief introduction to category theory with some applications to computing.

Leinster's book \cite{leinster} (available for free online, under a free license) gives a concise introduction to category theory.
It is a good resource for the basic concepts, but does not feature many examples from computer science.

The rather substantial textbook by Barr and Wells \cite{barr-wells} (available for free online) covers a lot more than we are going to discuss in these notes.

The Catsters \cite{catsters} provide a lecture series on category theory on YouTube.



\subsection{Notations}
\label{sec:notation}

A list of notations which we use throughout these lectures notes.
\begin{itemize}
\item If $X$ is a set and $x$ is an element of $X$, we write $x \in X$.
\item If $X$ is a set and $P$ and $Q$ are properties dependent over the elements of $X$, we write $P\implies Q$ to express that if $P(x)$ holds for an element $x\in X$, then also $Q(x)$ should hold for the element $x$. Moreover, we write $P \iff Q$ if $P\implies Q$ and $Q\implies P$.
\item If $X$ is a set and $P$ is a property dependent over the elements of $X$, we write:
\begin{itemize}
\item[(*)] $\forall x\in X: P(x)$ to express that for every element in $X$, the property $P$ holds.
\item[(*)] $\exists x\in X: P(x)$ to express that there exists at least one element in $X$ for which the property holds.
\item[(*)] $\exists! x\in X: P(x)$ to express that there exists a unique element in $X$ for which the property holds.
\end{itemize}
\item Let $X$ and $Y$ be sets. We denote:
\begin{itemize}
\item[(*)] $X\times Y$ for the (cartesian) product of $X$ and $Y$.
\item[(*)] $X\sqcup Y$ for the disjoint union of $X$ and $Y$.
\end{itemize}
\end{itemize}



\section{Categories}
\label{sec:categories}

\begin{reading*}
The definition of categories is also given in \cite[\S 2.1]{barr-wells}. Plenty of examples of categories are given in \cite[\S\S 2.3--2.5]{barr-wells}.

  The definition of categories is also given in \cite[\S 1.1]{leinster}, together with some examples.
  There, also isomorphisms are discussed, which we define in \cref{sec:isos}.

  The tutorial \cite{pierce} features the definition of categories in \cite[\S 2.1]{pierce}.
  It also introduces the notion of ``diagram'', which we do not use in the present notes.
\end{reading*}

\begin{dfn}\label{dfn:category}
  A \textbf{category} $\CC$ consists of the following data:
\begin{enumerate}
\item A collection of objects, denoted by $\Ob{\CC}$.
\item For any given objects $X,Y \in \Ob{\CC}$, a collection of morphisms from $X$ to $Y$, denoted by $\Hom[\CC]{X}{Y}$ (or $\Hom{X}{Y}$ when the category $\CC$ is clear, or $\CHom \CC X Y$ or $X \to Y$) and which is called a \textit{hom-set}.
\item For each object $X\in \Ob{\CC}$, a morphism $\Id[X] \in \Hom[C]{X}{X}$, called the \textit{identity morphism} on $X$.
\item A binary operation
\[
(\co{}{})_{X,Y,Z} : \Hom{Y}{Z} \to \Hom X Y \to \Hom X Z,
\]
called the \textit{composition operator}, and written infix without the indices $X,Y,Z$ as in $\co{f}{g}$.
\end{enumerate}
Moreover, this data should satisfy the following properties:
\begin{enumerate}
\item (\textbf{Left unit law}) For any morphism $f \in \Hom X Y$, we have 
\[
 \co{\Id[X]} {f} = f.
\]
\item (\textbf{Right unit law}) For any morphism $f \in \Hom X Y$, we have 
\[
  \co f {\Id[Y]} = f.
\]
\item (\textbf{Associative law}) For any morphisms $f\in \Hom X Y$, $g\in \Hom Y Z$ and $h\in \Hom Z W$, we have
\[
     \co {(\co f g)}{h} =  \co f {(\co g  h)}.
\]
\end{enumerate}
\end{dfn}

\begin{intu} So what does a category represent? There are (at least) $3$ possible ways how one can think about this definition:
\begin{enumerate}
\item A category represents a type system in the sense that the objects are the types and each hom-set is the type\footnote{In this case, each hom-set is a type, so isn't each hom-set an object again? Categories which satisfy such a property are called \textit{cartesian closed}.} of functions. See \cref{example:hask}.
\item A category represents a \textit{bag} of instances of a particular mathematical structure (e.g. sets with a notion of addition). The objects are then instances of such a mathematical theory (e.g. $(\mathbb{N},+)$) and the morphisms are structure preserving functions (e.g. functions $f$ which satisfy $f(x+y) = f(x) + f(y)$). See \cref{example:set,example:poset,monoidcategory}.
\item A category represents a directed graph in the sense that an object is a vertex and a morphism is an edge.
\item Anything (almost at least) can be seen as a category in some exotic way. 
\end{enumerate}
\end{intu}

\begin{nota} Let $\CC$ be a category.
\begin{itemize}
\item We write $X\in\CC$ instead of $X\in \Ob{\CC}$. 
\item Let $X,Y\in \CC$ be objects. A morphism $f\in\CHom{\CC}{X}{Y}$ can be visualized as \[ X \xrightarrow{f} Y. \]
\item Let $X,Y, Z \in \Ob{\CC}$ objects in $\CC$ and consider the following morphisms:
\[
f\in\CHom{C}{X}{Y}, \quad g\in\CHom{C}{Y}{Z}, \quad h\in\CHom{C}{X}{Z}.
\]
These morphisms can be visualized as a triangle:
\[
\begin{tikzcd}
X \arrow[r, "f"] \arrow[dr,swap, "h"] & Y \arrow[d, "g"] \\
& Z
\end{tikzcd}
\]
We say that such a triangle \textbf{commutes} if $h = \co{f}{g}$.
\item Let $X,Y_1,Y_2, Z \in \Ob{\CC}$ objects in $\CC$ and consider the following morphisms:
\[
f_1\in\CHom{C}{X}{Y_1}, \quad f_2\in\CHom{C}{X}{Y_2}, \quad g_1\in\CHom{C}{Y_1}{Z}, \quad g_2\in\CHom{C}{Y_2}{Z}.
\]
These morphisms can be visualized as a square:
\[
\begin{tikzcd}
X \arrow[r, "f_2"] \arrow[d,swap, "f_1"] & Y_2 \arrow[d, "g_2"] \\
Y_1 \arrow[r, swap, "g_1"] & Z 
\end{tikzcd}
\]
We say that such a square \textbf{commutes} if $\co{f_1}{g_1} = \co{f_2}{g_2}$.
\end{itemize}
\end{nota}


\begin{exa}\label{example:set} The \textbf{Category of sets}, denoted by $\SET$, is the category specified by the following data:
\begin{itemize}
\item An object is a set.
\item If $X$ and $Y$ are sets, then is $\CHom \SET X Y$ the set of all functions from $X$ to $Y$.
\item The identity morphism $\Id[X]$ (on $X\in\Ob{\SET}$) is the identity function on $X$, i.e.
\[
\Id[X] : X\to X: x \mapsto x.
\]
\item The composition of functions is given by the usual composition of functions, i.e. for $f\in \CHom \SET X Y$ and $g\in \CHom \SET Y Z$, the composition of $f$ and $g$ is:
$$g \circ f : X\to Z: x\mapsto g(f(x)).$$
\end{itemize}
\end{exa}
\begin{lemma} The data of $\SET$ satisfies the properties of a category, hence $\SET$ is indeed a category.
\begin{proof}
We first show that the left unit law holds. Let $X,Y\in \mathbf{Set}$ be sets and $f\in \CHom \SET X Y$ a function. We have to show that $\Id[X] \Comp f = f$, hence it suffices to show that they pointwise equal which holds by the following calculation:
\[
\forall x\in X: (f\circ \Id[X])(x) = f\left(\Id[X](x)\right) = f(x),
\]
where the first (resp. second) equality holds by definition of the composition (resp. identity morphism).\\
That the right unit law holds is analogous. To show that the associator law holds, let $X,Y,Z,W\in\mathbf{Set}$ and $f\in \CHom \SET X Y, g\in \CHom \SET Y Z$ and $h\in \CHom \SET Z W$. We have to show $h\circ (g\circ f) = (h\circ g)\circ f$, hence it suffices again to show that they are pointwise equal which holds by the following calculation:
\begin{eqnarray*}
\forall x\in X: \left(h\circ (g\circ f)\right)(x) &=& h\left((g\circ f)(x)\right), \\ 
	&=& h(g(f(x))),\\ 
	&=& (h\circ g)(f(x)),\\ 
	&=& \left((h\circ g)\circ f\right)(x),
\end{eqnarray*}
where the first (resp. second, third, fourth) equality holds by definition of the composition of $h$ and $g\circ f$ (resp. composition of $g$ and $f$, composition of $h$ and $g$, composition of $h\circ g$ and $f$).
\end{proof}
\end{lemma}

We are now going to describe the category whose collection of objects is given by collection of Lean types:
\begin{exa}\label{exa:lean-cat}
  Consider the following data: 
\begin{itemize}
\item An object is a Lean type (of some fixed universe).
\item If $X$ and $Y$ are Lean types, then is $\CHom \LEAN X Y$ the function type $X\to Y$.
\item The identity morphism $\Id[X]$ (on $X\in \Ob{\LEAN}$) is the identity function on $X$, i.e.
\begin{lstlisting}
def idfun (X : Type) : X → X := λ x, x.
\end{lstlisting}
\item The composition of functions is given by the composition of functions:
\begin{lstlisting}
def compfun {X Y Z} (f : X → Y) (g : Y → Z) : X → Z
  := λ x, g (f x)
\end{lstlisting}
\end{itemize}
  Try it out, e.g., on \url{https://leanprover.github.io/live/latest/}:
\begin{lstlisting}
#eval compfun (+1) (^3) 5
\end{lstlisting}  
(You can get a pre-filled Lean input field by clicking here: \href{https://leanprover.github.io/live/latest/#code=%0Adef%20idfun%20(X%20:%20Type)%20:%20X%20%E2%86%92%20X%20:=%20%CE%BB%20x,%20x.%0Adef%20compfun%20%7BX%20Y%20Z%7D%20(f%20:%20X%20%E2%86%92%20Y)%20(g%20:%20Y%20%E2%86%92%20Z%20)%20:%20X%20%E2%86%92%20Z%0A:=%20%CE%BB%20x%20,%20g%20(%20f%20x%20)%0A%0A#eval%20compfun%20(+1)%20(%5E3)%205}{\textbf{ClickMe}}.)
\end{exa}

\begin{exer}
  Prove (on paper) that the data defined in \cref{exa:lean-cat} defines a category.
  That is, show that it satisfies the axioms of a category.
  You might need to use the \textbf{axiom of functional extensionality}:
\begin{lstlisting}
axiom funext_nondep : ∀ {A B : Type} (f g : A → B), 
  (∀ x, f x = g x) → f = g
\end{lstlisting}
\end{exer}

\begin{exa}
  We repeat the definitions of \cref{exa:lean-cat} in Haskell instead of Lean.
  Does this data satisfies the axioms of a category?

  Due to Haskell allowing for the |undefined| value in each type, the situation is slighly more complicated; consider the following two functions:
\begin{lstlisting}
undef1 :: a -> a
undef1 = undefined

undef2 :: a -> a
undef2 = \x -> undefined
\end{lstlisting}
These are not equal by definition, but we have $\Id \Comp$ |undef1| $=$ |undef2|.
So by the right unit law, we must have that |undef1| = |undef2| (as morphisms in our sought category).
\end{exa}

% \begin{exer}
%   Compose the functions |undef1| and |undef2| with some other functions of your choice, and see what happens.
% \end{exer}

\begin{exer}
  Read the Haskell wiki page on the category $\HASK$ \cite{haskell-wiki-hask}.
\end{exer}


However, when considering functions to equal when they are \textbf{pointwise} equal, we can define a category of Haskell types:
\begin{dfn}\label{example:hask} The \textbf{category of Haskell types}, denoted by $\HASK$, is the category specified by the following data:
\begin{itemize}
\item An object is a Haskell type.
\item If $X$ and $Y$ are Haskell types, then is $\CHom \HASK X Y$ the collection of functions modulo the equivalence relation $\sim$ defined by identifying pointwise equal functions:
\[
f \sim g :\iff \forall x : X, f(x) = g(x).
\]
i.e. a morphism in $\HASK$ is an equivalence class of (Haskell) functions.
\item The identity morphism $\Id[X]$ (on $X\in\HASK$) is the equivalence class of the identity function on $X$.
\item The composition of (Haskell) functions is given by the equivalence class of the composition of functions, i.e., for $f\in \CHom \HASK X Y$ and $g\in \CHom \HASK Y Z$, the composition of $f$ and $g$ is the equivalence class of:
\[g\circ f : X\to Z: \lambda x. g(f(x)).\]
\end{itemize}
\end{dfn}


\begin{exa}\label{example:posetcategories}
Recall that a \textit{preordered set} $(X,\leq)$ consists of a set $X$ together with a binary relation $(\leq)$ on $X$ which satisfies the following properties:
\begin{itemize}
\item \textbf{Reflexivity}: $\forall x\in X: x\leq x$.
\item \textbf{Transitivity}: $\forall x,y,z\in X: \left(x\leq y \wedge y\leq z\right) \implies x\leq z$.
\end{itemize}

  Let $(X,\leq)$ be a preordered set. The following data induces a category $\PRE(X,\leq)$:
\begin{itemize}
\item The objects are the elements of $X$.
\item Let $x,y \in X$ be elements. The hom-set $\Hom x y$ consists of a unique element if $x\leq y$ and is empty otherwise.
\item  We need an identity morphism for each $x\in X$.
  By reflexitivity (i.e., $x\leq x$), we have that $\Hom x x$ consists of a unique element, which we take to be the identity.
\item We need to define for each $x,y,z\in X$, a composition operator:
\[
\Hom y z \to \Hom x y \to \Hom x z.
\]
By definition of the hom-sets, we only have to define it in case $x\leq y$ and $y\leq z$.
But then, by transitivity (i.e. if $x\leq y$ and $y\leq z$, then $x\leq z$), we have that $\Hom x z$ consists of a unique element which we take to be the composition.
\end{itemize}
%
We are now going to show that the axioms of a category holds.
To show the right unit law, we have to show that for each $x,y\in X$ and $f\in \Hom x y$, we have $\co{\Id[x]}{f} = f$.
This indeed holds since every hom-set has a unique element, but both $\co{\Id[x]}{f}$ and $f$ live in the same hom-set, hence they must be equal.
The proof that left unit law and associator law hold are analogous.
\end{exa}

\begin{exer}[\cref{sol:post_antisymmetry}]\label{exer:post_antisymmetry}
  A \textbf{partially ordered set} (poset) is a preordered set $(X,\leq)$ satisfying the following additional axiom:
  \begin{itemize}
  \item \textbf{Antisymmetry}: $\forall x,y\in X: (x\leq y \wedge y\leq x) \implies x=y$.
  \end{itemize}
  What does this axiom say about $\PRE(X,\leq)$?
\end{exer}

\begin{rem}
  To understand a definition in category theory, it is very helpful to think about what the definition means in a preordered set, viewed as a category. 
\end{rem}

\begin{exa}\label{example:poset} The category of posets, denoted by $\POS$, is the category specified by the following data:
\begin{itemize}
\item An object is a poset $(X,\leq)$.
\item A morphism from a poset $(X,\leq_X)$ to $(Y,\leq_Y)$ consists of a function $f:X\to Y$ such that the following property holds:
\[
\forall x_1, x_2 \in X: x_1\leq_X x_2 \implies f(x_1)\leq_Y f(x_2).
\]
\item The identity morphism on $(X,\leq_X)$ is the identity function on $X$.
\item The composition given by the composition of functions.
\end{itemize}

Before we can show that this data satisfies the axioms of a category, notice that the identity function is a morphism of posets and that the composition of poset morphisms is again a poset morphism, indeed: If $x_1\leq_X x_2$, then we also have $\Id[X](x_1) \leq_X \Id[X](x_2)$ because $\Id[X](x) = x$. If $f\in\CHom{\POS}{(X,\leq_X)}{(Y,\leq_Y)}$ and $g\in\CHom{\POS}{(Y,\leq_Y)} {(Z,\leq_Z)}$ are morphisms of posets, then we have 
\[
\forall x_1,x_2\in X: x_1\leq_X x_2 \implies f(x_1)\leq_Y f(x_2) \implies g(f(x_1))\leq_Z g(f(x_2)),
\]
where the first (resp. second) inequality holds by $f$ (resp. $g$) being a morphism of posets. So our data is indeed well-defined.\\
That the axioms of a category are satisfied by this data, is exactly the same proof as showing that $\SET$ is a category because the identity and composition are defined in the same way.
\end{exa}

\begin{exer}[\cref{sol:POS_isnt_a_posetcat}]\label{exer:POS_isnt_a_posetcat}
  Is $\POS$ a preorder-category itself? That is, is there at most one morphism between any two objects?
\end{exer}

\begin{que}\label{que:posetcatstoallcats} In \cref{example:poset}, we have shown that $\POS$ is a category. However, by \cref{example:posetcategories}, we know that any poset also is a category. So we have that $\POS$ is a category whose objects are certain categories. Can we also have some category whose collection of objects is the collection of all categories, and if so, what are the morphisms of categories? 
\begin{proof}[Solution]
\cref{sec:functors} is devoted completely to this answer.
\end{proof}
\end{que}

\begin{lemma}\label{lemma:uniqueid} Let $\CC$ be a category. For any object $X\in\CC$, $\Id[X]$ is the unique morphism which satisfies the following property: For any $Y\in\CC$ and $f\in\CHom \CC X Y$, we have 
\[
\co{\Id[X]} f = f.
\]
\begin{proof}
Assume $\tilde{\Id[X]}$ also satisfies this property, in particular we have $\co {\tilde{\Id[X]}} {\Id[X]} = \Id[X]$. However, by the right unit law, we also must have $\co{\tilde{\Id[X]}}{\Id[X]} = \tilde{\Id[X]}$. Hence, $\Id[X] = \tilde{\Id[X]}$.
\end{proof}
\end{lemma}

\begin{exa}\label{exa:monoidofrationalnumbers} In this example we are going to associate a category which captures the multiplication of the rational numbers. Let $\CC$ be the category defined by the following data:
\begin{itemize}
\item There is a unique object $\star$.
\item The (only) hom-set is given by
\[
\Hom{\star}{\star} = \mathbb{Q},
\]
i.e. each morphism corresponds with a rational number.
\item The composition is defined by the multiplication of rational numbers:
\[
\mathbb{Q} \to\mathbb{Q}\to\mathbb{Q} : (p,q)\mapsto p\cdot q.
\]
\item The identity morphism (of $\star$) is given by $1$.
\end{itemize}
That $\CC$ is indeed a category follows because for each $p\in\mathbb{Q}$, we have $p\cdot 1 = p = 1\cdot p$ (which shows the unit laws) and by associativity of multiplication, i.e. $(p\cdot q)\cdot h = p\cdot (h \cdot q)$ (which shows the associativity of the composition).
\end{exa}
The construction in \cref{exa:monoidofrationalnumbers} uses no specific properties of the rational numbers, only that it has a multiplication which is associative and such that there is a special element which does not change an element when it is multiplied with this special element. Hence, \cref{exa:monoidofrationalnumbers} can be generalized as follows:
\begin{dfn}\label{monoidcategory}
Recall that a monoid is a set $M$ equipped with binary operation $m : M \to M \to M$ which is associative, i.e. 
$$\forall x,y,z\in M: m(x,m(y,z)) = m(m(x,y),z),$$
and such that there is an identity element, i.e. 
\[
\exists e\in M: \forall x\in M: m(e,x)=x=m(x,e).
\]
Let $(M,m,e)$ be a monoid. The category $\MON(M,m,e)$ is defined by the following data:
\begin{itemize}
\item There is a unique object $\star$.
\item The (only) hom-set is given by 
\[
\Hom{\star}{\star} = M.
\]
\item The identity morphism on $\star$ is the identity element $e$.
\item The composition of morphisms $x$ and $y$ is given by $\co{x}{y} := m(x,y)$.
\end{itemize}
\end{dfn}

That for each monoid $(M,m,e)$, $\MON(M,m,e)$ is indeed a category, follows directly by the properties of being a monoid. Indeed, the axioms of a category become precisely:
\begin{enumerate}
\item $\forall x\in M: m(x,e)=x$,
\item $\forall x\in M: m(e,x)=x$,
\item $\forall x,y,z\in M: m(m(x,y),z) = m(x,m(y,z))$.
\end{enumerate}

\begin{rem} Notice that this category illustrates that there is no relation between the collection of objects and the hom-sets since there is now only one object and the collection of the hom-set can be as small or as large as possible.\\
In fact, we can associate a different number of categories to a single monoid. We can for example consider an arbitrary set of objects $I$ and the defining the hom-sets as follows:
\[
\Hom{i}{j} := 
\begin{cases}
M ,\quad \text{ if } i=j,\\
\emptyset, \quad \text{ if } i\not=j.
\end{cases}
\]
\end{rem}

\begin{exer}[\cref{sol:categories_coming_from_monoids}]\label{exer:categories_coming_from_monoids}
  Let $\CC$ be a category. When is $\CC$ of the form $\MON(M,m,e)$, i.e. does there exists a monoid $(M,m,e)$ such that $\CC =  \MON(M,m,e)$?
\end{exer}

\begin{exer}[\cref{sol:category_of_monoids}]\label{exer:category_of_monoids}
  Define a category $\MON$ whose objects are monoids, i.e. define a suitable notion of morphism between monoids and moreover show that this indeed defines a category.
\end{exer}

\begin{exer}[\cref{sol:category_with_naturalnumbers_and_matrices}]\label{exer:category_with_naturalnumbers_and_matrices}
  Define a category $\CC$ whose objects are the natural numbers (i.e. $\Ob{\CC} = \mathbb{N}$) and whose hom-sets $\CHom{\CC}{n}{m}$ are given by the $(n\times m)$-matrices.
\end{exer}

\begin{exer}[\cref{sol:opposite}]\label{exer:opposite}
  Let $\CC$ be a category. Define a category $\op\CC$ such that
  \begin{itemize}
  \item the objects of $\op\CC$ are the same as those of $\CC$; and
  \item the morphisms $\CHom {\op\CC} X Y$ are morphisms $\CHom \CC Y X$.
  \end{itemize}
  The category $\op\CC$ is called the \textbf{opposite (category)} of $\CC$.
\end{exer}

\begin{exer} Let $G$ be a directed graph. Then $G$ induces a category $\mathbf{Graph}(G)$ as follows:
\begin{itemize}
\item The collection of objects $\Ob{\mathbf{Graph}(G)}$ is the set of vertices of $G$. 
\item The morphisms between object are the (directed) paths, that is, finite sequences of composible edges, between them.
\item For each object $x$ (i.e. vertex), the identity morphism on $x$ is the \textit{identity path}.
\item The composition of morphisms is the composition of paths.
\end{itemize}
We call $\mathbf{Graph}(G)$ the \textbf{category generated by $G$}. Show that $\mathbf{Graph}(G)$ is indeed a category.
\end{exer}

\begin{exer} Argue why the morphisms are chosen to be paths and it is not sufficient to just take the vertices. 
\end{exer}

\begin{exa}\label{exa:graph_terminalcat} Consider the following graph $G$:
\[
\begin{tikzcd}
x
\end{tikzcd}
\]
i.e. the graph with only object vertex and no edges. The category generated by $G$ is the category generated is the so-called \textit{terminal category}, that is, the category with a single object and a single morphism (the identity morphism of the unique object). 
\end{exa}

\begin{exa}\label{exa:graph_intervalcat} Consider the following graph $G$:
\[
\begin{tikzcd}
x \arrow[r] & y
\end{tikzcd}
\]
The category generated by $G$ is the category generated is the so-called \textit{interval category}, that is, the category with two objects and, besides the identity morphisms, a unique morphism (living in $\Hom{x}{y}$).
\end{exa}

In the following example we use the following notation: 
\begin{itemize}
\item If $f$ is a morphism in a category, we denote $f^{2} := \co{f}{f}, f^{3} := \co{f}{f^2}$, etc.
\item We also label the edges in order to refer to them.
\end{itemize}
\begin{exa}\label{exa:graph_xy_yx} Consider the following graph $G$:
\[
\begin{tikzcd}
x \arrow[r, "f", bend left] & y \arrow[l, "g", bend left]
\end{tikzcd}
\]
The category generated by $G$ consists of the following data:
\begin{itemize}
\item The collection of objects is $\{x,y\}$.
\item The hom-sets are given as follows:
\begin{itemize}
\item $\Hom{x}{x}$ contains
\[
\Id[x], \co{f}{g}, (\co{f}{g})^2, (\co{f}{g})^3, \cdots,
\]
But these are not the only ones, we also have that each of these can be precomposed or postcomposed with $\Id[x]$, however, by the unit laws, we know that these don't give us any \textit{new} morphisms. The same remark holds for the associativity law. This comment also holds for the upcoming hom-sets.
\item $\Hom{y}{y}$ contains
\[
\Id[y], \co{g}{f}, (\co{g}{f})^2, (\co{g}{f})^3, \cdots,
\]
\item $\Hom{x}{y}$ contains  
\[
f, \co{f}{(\co{g}{f})}, \co{f}{(\co{g}{f})^2}, \co{f}{(\co{g}{f})^3}, \cdots
\]
\item $\Hom{y}{x}$ contains
\[
g, \co{g}{(\co{f}{g})}, \co{g}{(\co{f}{g})^2}, \co{g}{(\co{f}{g})^3}, \cdots
\] 
\end{itemize}
\end{itemize}
\end{exa}

\begin{exa}\label{exa:graph_yx_yz_zw} Consider the following graph $G$:
\[
\begin{tikzcd}
& w & \\
x & & z \arrow[lu] \\
& y \arrow[lu] \arrow[ru] &
\end{tikzcd}
\]
The category generated by $G$ has four objects (namely $x,y,z,w$) and the hom-sets are: 
\begin{itemize}
\item $\Hom{y}{x}, \Hom{y}{z}$ and $\Hom{z}{w}$ are singleton sets, 
\item $\Hom{x}{y}, \Hom{z}{y}, \Hom{x}{w}, \Hom{w}{x}$ and $\Hom{w}{z}$ are all empty. 
\item $\Hom{y}{w}$ consists of the path $y\to z\to w$.
\item For each vertex $v$, we have that $\Hom{v}{v}$ consists only of the identity path on $v$.
\end{itemize}
\end{exa}

\begin{exer}[\cref{sol:connection_graphs_preordersets}] \label{exer:connection_graphs_preordersets}
Describe the connection between the categories generated by graphs and the categories associated to preordered sets. What is the property of anti-symmetry translated under this connection with graphs?
\end{exer}

\begin{exer} Define a category $\Catb{Aut}$ whose objects are (deterministic finite) automata. 
\end{exer}

\subsection{Isomorphisms}
\label{sec:isos}

\begin{reading*}
  In this section, we study properties of arrows in a category.
  More information on this topic is given in \cite[\S 2.7]{barr-wells}.

  Also, \cite[\S 2.2]{pierce} briefly discusses isomorphisms.
\end{reading*}


\begin{dfn}[Isomorphism]
  Given a category $\CC$, objects $a,b \in \Ob{\CC}$ and a morphism $f : a \to b$ in $\CC$, we say that $f$ is an \textbf{isomorphism} when there is a morphism $g : b \to a$ (in the other direction!) such that $f \Comp g = \Id$ and $g \Comp f = \Id$.
  We write $f : a \cong b$ for a morphism $f$ that is an isomorphism.

  In this case, we call $g$ the \textbf{inverse} of $f$ and $f$ the inverse of $g$. (The latter is justified by \cref{exer:inverse-iso}.)
\end{dfn}

\begin{exer}[\cref{sol:inverse-iso}]\label{exer:inverse-iso}
  Show that if $f : a \to b$ is an isomorphism with inverse $g : b \to a$, then $g$ is an isomorphism with inverse $f$.
\end{exer}

\begin{exer}[\cref{sol:inverse_uniqueness}]\label{exer:inverse_uniqueness}
  Show that a morphism $f : a \to b$ in $\CC$ is an isomorphism \textbf{in at most one way}, that is, show that its inverse is unique if it exists.
\end{exer}

\begin{exer}[\cref{sol:compofiso}]\label{exer:compofiso}
  Show that the composition of two isomorphisms is an isomorphism.
\end{exer}
\begin{rem} 
Since any identity morphism is an isomorphism (check this!), we conclude by \cref{exer:compofiso} that given any category $\CC$, we always get a new category $isos(\CC)$ by restricting the morphisms to be isomorphisms, i.e. \[
\Ob{isos(\CC)} = \Ob{\CC}, \quad \CHom{isos(\CC)}{X}{Y} = \left\{f \in \CHom{\CC}{X}{Y} \mid f \text{ is an isomorphism}\right\}
\] 
and where the identity and composition is the same as in $\CC$.
\end{rem}


\begin{exer}[\cref{sol:iso-bool}]\label{exer:iso-bool}
  Consider the Haskell datatype
\begin{lstlisting}
data BW = Black | White
\end{lstlisting}
Construct two (different!) isomorphisms between |BW| and the type |Bool| of booleans.
\end{exer}

\begin{exer}[\cref{sol:iso_in_sets}]\label{exer:iso_in_sets}
  Characterize the isomorphisms in $\SET$.
\end{exer}

\begin{exer}[\cref{sol:iso_in_pos}]\label{exer:iso_in_pos}
  Describe the isomorphisms in $\POS$.
\end{exer}

\begin{exer}[\cref{sol:iso_in_posetcategory}]\label{exer:iso_in_posetcategory} Let $(X,\leq)$ be a poset. Can you characterize the isomorphisms in $\POS(X,\leq)$?
\end{exer}

\begin{exer} Can you characterize the isomorphisms in $\MON$?
\end{exer}

\begin{exer} Let $\mathcal{G}$ be the category generated by the following graph:
\[
\begin{tikzcd}
& w & \\
x & & z \arrow[lu, bend left] \arrow[lu,bend right] \\
& y \arrow[lu] \arrow[ru] &
\end{tikzcd}
\]
Show that the only isomorphisms in $\mathcal{G}$ are the identity morphisms (i.e. the identity paths).
\end{exer}

\subsection{Sections and Retractions}
\label{sec:sections}


\begin{dfn}[Section, Retraction]
  A pair $(s,r)$ of morphisms $s : a \to b$ and $r : b \to a$ in $\CC$ is called a \textbf{section-retraction pair} if $\co{s}{r} = \Id[b]$.

  In such a case, we call $s$ a section and $r$ a retraction.
\end{dfn}

\begin{rem}
  Note that a morphism can be a retraction in more than one way, that is, there can be more than one section $s$ such that $\co{s}{r} = \Id$.
\end{rem}

Intuitively, a section-retraction pair $(s,r)$ of morphisms $s : a \to b$ and $r : b \to a$ in a category $\CC$ provides a way for $a$ to ``live inside'' $b$.
Note that for a given $a$ and $b$ there can be many ways for $a$ to live inside $b$.

\begin{exer}[\cref{sol:section-retraction-bool-int}]\label{exer:section-retraction-bool-int}
  Construct two different section-retraction pairs between the type |Bool| of booleans and the type |Int| of integers (e.g., in Haskell).
\end{exer}



\begin{exer}
 Show that the type |Maybe a| is a retract of the type |[a]|. 
 
 Hint: The idea is that |Nothing| corresponds to the empty list |[]| and that |Just x| corresponds to the one-element list |[x]|. Make this idea precise by writing back and forth functions between these types so that they exhibit |Maybe a| as a retract of |[a]|. 
\end{exer}


\subsection{Monomorphisms and Epimorphisms}
\label{sec:mono-epi}

\begin{reading*}
See also \cite[p. 134]{leinster} and \cite[\S\S 2.8--2.9]{barr-wells}.
Also, \cite[\S 2.2]{pierce} briefly discusses monomorphisms and epimorphisms.
\end{reading*}

From undergraduate mathematics courses you know what injective and surjective functions between sets are.
The definitions of ``injective'' and ``surjective'' do not carry over to any category (though they do for categories that are, in some sense, ``similar'' to the category of sets).
In this section, we study two properties of morphisms in a category that, in the category of sets, are equivalent to ``injective'' and ``surjective'', respectively.



\begin{dfn}[Monomorphism]
  Let $f : a \to b$ be a morphism in $\CC$. We say that $f$ is a \textbf{monomorphism} if, for any two morphisms $g_1, g_2 : z \to a$, like in the following diagram,
  \begin{center}
    \begin{tikzcd}
    z \arrow[r, "g_2"', shift right] \arrow[r, "g_1", shift left] & a \arrow[r, "f"] & b
    \end{tikzcd}
  \end{center}
  we have
  \[ \co{g_1}{f} = \co{g_2}{f} \text{ implies } g_1 = g_2 .\]
\end{dfn}

\begin{exer}[\cref{sol:mono-inj}]\label{ex:mono-inj}
  In the category of sets, show that a morphism $f : X \to Y$ is a monomorphism if and only if it is injective.
\end{exer}

\begin{dfn}[Epi]
  Let $f : a \to b$ be a morphism in $\CC$. We say that $f$ is an \textbf{epimorphism} if, for any two morphisms $g_1, g_2 : b \to z$, like in the following diagram,
  \begin{center}
    \begin{tikzcd}
    a \arrow[r, "f"] & b \arrow[r, "g_2"', shift right] \arrow[r, "g_1", shift left] & z
    \end{tikzcd}
  \end{center}
  we have
  \[ \co{f}{g_1} = \co{f}{g_2} \text{ implies } g_1 = g_2 .\]
\end{dfn}

\begin{exer}\label{ex:epi-surj}
  In the category of sets, show that a morphism $f : X \to Y$ is an epimorphism if and only if it is surjective.
\end{exer}


\begin{exer}[\cref{sol:sections_in_set_injective}]\label{exer:sections_in_set_injective}
  In the category of sets, show that if $(s,r)$ is a section-retraction pair, then is the section $s$ injective.
  Hint: you can use \cref{ex:mono-inj}.
\end{exer}
\begin{exer}
  In the category of sets, show that if $(s,r)$ is a section-retraction pair, then the retraction $r$ is surjective.
    Hint: you can use \cref{ex:epi-surj}.
\end{exer}

\begin{exer}[\cref{sol:iso_to_monoepi}]\label{exer:iso_to_monoepi}
  Show that any isomorphism $f : a\cong b$ (in some arbitrary category $\CC$) is both a monomorphism and an epimorphism.
\end{exer}

\begin{exer}[\cref{sol:counterexample_monoepi_not_iso}]\label{exer:counterexample_monoepi_not_iso}
  Show that the converse of \cref{exer:iso_to_monoepi} does not hold in general, i.e. give an example of a category where there exists a morphism which is both an epi- and a monomorphism, but which is not an isomorphism.

Hint: Consider a preordered set.
\end{exer}

\begin{exer} Let $\mathcal{G}_1$ (resp. $\mathcal{G}_2$ and $\mathcal{G}_3$) be the category generated by the following graph:
\[
\begin{tikzcd}
& w & \\
x & & z \arrow[lu] \\
& y \arrow[lu] \arrow[ru] &
\end{tikzcd}
\]
resp.
\[
\begin{tikzcd}
& w & \\
x & & z \arrow[lu, bend left] \arrow[lu,bend right] \\
& y \arrow[lu] \arrow[ru] &
\end{tikzcd}
\]
resp.
\[
\begin{tikzcd}
& w \arrow[rd,bend left] & \\
x & & z \arrow[lu, bend left] \\
& y \arrow[lu] \arrow[ru] &
\end{tikzcd}
\]

Can you characterize the mono- and epimorphisms in these categories?
\end{exer}

\begin{exer} Can you characterize the monomorphisms, epimorphisms and isomorphisms in the category generated by the following graph:
\[
\begin{tikzcd}
x \arrow[r] & y
\end{tikzcd}
\]
\end{exer}

\section{Universal Properties}\label{sec:universal}

In this section, we discuss special objects in a category. 

\begin{reading*}
  On initial and terminal objects, see also \cite[\S 2.7.16]{barr-wells} and \cite[p. 48ff]{leinster}.

  Products and coproducts, other special limits and colimits, and the general definition of limits and colimits, are discussed in \cite[\S\S 5.1, 5.2]{leinster}.

  Pierce's tutorial discusses the (co)limits defined here in \cite[\S\S 2.3--2.4]{pierce}, and further (co)limits in  \cite[\S\S 2.5--2.7]{pierce}.

  
\end{reading*}

\subsection{Initial Objects}
\label{sec:initial-objects}



\begin{dfn}
  Let $\CC$ be a category. An object $A \in \Ob{\CC}$ is \textbf{initial} if there is exactly one morphism from $A$ to any object $B \in \Ob{\CC}$.
\end{dfn}

\begin{exer}
  Does the category $\bullet$ have an initial object?
\end{exer}

\begin{exer}
  Does the category 
  \[ 
  \begin{tikzcd}
  	A & B
  \end{tikzcd}  
   \] 
  have an initial object?
\end{exer}


\begin{exer}
  Does the category $A \to B$ have an initial object?
\end{exer}

\begin{exer}
  Does the category $A \rightleftarrows B$ have an initial object? (Here, the morphism $A \to B$ is inverse to the morphism $B \to A$.)
\end{exer}

\begin{exer}
  Does the category $A \rightrightarrows B$ have an initial object?
\end{exer}

\begin{exer}
  Does the category 
  \[
  \begin{tikzcd}
  A \arrow[r] \arrow[loop, swap, looseness=4, "f"] & B
  \end{tikzcd}
  \]
  have an initial object? (Here, the morphism $f$ is different from $\Id[A]$).
\end{exer}

\begin{exer}[\cref{sol:initial_set}]\label{exer:initial_set}
  Identify an initial object in the category $\SET$ of sets.
  Prove that it is indeed initial.
\end{exer}

\begin{exer}
  Identify an initial object in the category $\LEAN$ of Lean types.
  Prove that it is indeed initial.
\end{exer}

\begin{exer}[\cref{sol:initial_posetcat}]\label{exer:initial_posetcat}
  Let $(X,\leq)$ be a poset. Describe what an initial object looks like in  $\POS(X,\leq)$.
\end{exer}

\begin{exer}[\cref{sol:initial-unique}]\label{exer:initial-unique}
  Let $A$ and $A'$ be initial objects in $\CC$. Construct an isomorphism $i : A \cong A'$.
\end{exer}

\begin{exer}[\cref{sol:initiality_preserved_by_iso}]\label{exer:initiality_preserved_by_iso}
  Let $A$ be an initial object in $\CC$, and let $A'$ be isomorphic to $A$ (via an isomorphism $i : A \cong A'$).
  Show that $A'$ is an initial object of $\CC$.
\end{exer}

\begin{rem}
  \Cref{exer:initial-unique} shows that initial objects in a category $\CC$ are \textbf{essentially unique}, that is, they are \textbf{unique up to (unique) isomorphism}.

  
  This justifies using the determinate article: we will say that $A$ is \textbf{the} initial object of $\CC$.

  
  This is more generally the case for any object with a universal property, see, e.g., \cref{exer:terminal-unique,exer:product-unique}.
\end{rem}

\begin{exer}[\cref{sol:cat-without-initial}]\label{exer:cat-without-initial}
  Construct a category that does not have an initial object.
\end{exer}

\begin{exer}[\cref{sol:initial_pointset}]\label{exer:initial_pointset} Let $\PTSET$ be the category of pointed sets, that is the category whose objects are pairs $(X,x)$ with $X$ a set and $x\in X$ and a morphism from $(X,x)$ to $(Y,y)$ is defined as a function $f:X\to Y$ such that $f(x)=y$. Identify an initial object in $\PTSET$.
\end{exer}

\begin{rem}
  The concept of initial object seems trivial and boring in the categories considered above.
  However, in complicated categories, initial objects can be complicated and exciting;
  we will see this in \cref{sec:initial-algs}.
\end{rem}

\subsection{Terminal Objects}
\label{sec:terminal-objects}



\begin{dfn}
  Let $\CC$ be a category. An object $B \in \Ob{\CC}$ is \textbf{terminal} (or \textbf{final}) if there is exactly one morphism to $B$ from any object $A \in \Ob{\CC}$.
\end{dfn}

\begin{exer}
  Does the category $\bullet$ have a terminal object?
\end{exer}

\begin{exer}
  Does the category $A \to B$ have a terminal object?
\end{exer}

\begin{exer}
  Does the category $A \rightleftarrows B$ have a terminal object?
\end{exer}

\begin{exer}
  Does the category $A \rightrightarrows B$ have a terminal object?
\end{exer}



\begin{exer}[\cref{sol:terminal_set}]\label{exer:terminal_set}
  Identify a terminal object in the category $\SET$ of sets.
  Prove that it is indeed terminal.
\end{exer}

\begin{exer}
  Identify a terminal object in the category $\LEAN$ of Lean types.
  Prove that it is indeed terminal.
\end{exer}

\begin{exer}[\cref{sol:terminal_posetcat}]\label{exer:terminal_posetcat}
  Let $(X,\leq)$ be a poset. Describe what a terminal object looks like in  $\POS(X,\leq)$.
\end{exer}

\begin{exer}[\cref{sol:terminal-unique}]\label{exer:terminal-unique}
  Let $B$ and $B'$ be terminal objects in $\CC$. Construct an isomorphism $i : B \cong B'$.
\end{exer}

\begin{exer}[\cref{sol:terminality_preserved_by_iso}]\label{exer:terminality_preserved_by_iso}
  Let $B$ be a terminal object in $\CC$, and let $B'$ be isomorphic to $B$ (via an isomorphism $i : B \cong B'$).
  Show that $B'$ is a terminal object of $\CC$.
\end{exer}

\begin{exer}[\cref{sol:terminal_iff_initial_op}]\label{exer:terminal_iff_initial_op}
  Show that $\CC$ has a terminal object if and only if $\op\CC$ has an initial object.
\end{exer}

\begin{exer}[\cref{sol:cat-without-terminal}]\label{exer:cat-without-terminal}
  Construct a category that does not have an terminal object.
\end{exer}

\subsection{(Binary) Products}
\label{sec:products}



\begin{dfn}
  Let $\CC$ be a category and let $A,B \in \Ob\CC$ be objects of $\CC$.

  A triple $(P,\projl : P \to A ,\projr : P \to B)$ is called a \textbf{product of $A$ and $B$} if for any triple $(Q,q_1 : Q \to A, q_2 : Q \to B)$ there is exactly one morphism $f : Q \to P$ such that the following diagram commutes:
  \[
    \begin{tikzcd}
      &
      Q \ar[ld, "q_1"'] \ar[rd, "q_2"] \ar[d, dashed, "f"]
      &
      \\
      A
      &
      P \ar[l, "\projl"] \ar[r, "\projr"']
      &
      B
    \end{tikzcd}
  \]
  If $A$ and $B$ have a specified product $(P,\projl : P \to A ,\projr : P \to B)$, then the object $P$ is often called $A \times B$.
  The morphism $f : Q \to A \times B$ determined by $(Q, q_1, q_2)$ is denoted by $\langle q_1, q_2 \rangle$.
\end{dfn}



\begin{exer}
  Does the category $\bullet$ have products?
\end{exer}

\begin{exer}
  Does the category $A \to B$ have products?
\end{exer}

\begin{exer}
  Does the category $A \rightleftarrows B$ have products?
\end{exer}

\begin{exer}
  Does the category $A \rightrightarrows B$ have products?
\end{exer}


\begin{exer}
  Identify a product of sets $X$ and $Y$ in the category $\SET$ of sets.
  Prove that it is indeed a product.
\end{exer}

\begin{exer}
  Identify a product of types $A$ and $B$ in the category $\LEAN$ of Lean types.
  Prove that it is indeed a product.
\end{exer}

\begin{exer}
  Let $(X,\leq)$ be a poset. Describe what a product looks like in  $\POS(X,\leq)$.
\end{exer}

\begin{exer}\label{exer:product-unique}
  Given two products of $A$ and $B$ in a category $\CC$, construct an isomorphism between them, that is, between their underlying objects.
\end{exer}

\begin{exer}
  Given a product $(P,\projl : P \to A ,\projr : P \to B)$ of $A$ and $B$ in $\CC$, and an object $P'$ that is isomorphic to $P$ via an isomorphism $i : P \cong P'$, construct a product with object $P'$ of $A$ and $B$.
\end{exer}

\begin{exer} Let $\CC$ be a category and $T\in\Ob{\CC}$ a terminal object.
  For any object $A\in \Ob{\CC}$, construct a product of $A$ and $T$.

  Hint: to form an idea what the object $A \times T$ should be, solve the exercise first in a specific category, e.g., in the category of sets or in a category coming from a preordered set.
\end{exer}




\begin{exer} Let $\CC$ be a category and $A,B\in\Ob{\CC}$ be objects. Show that the product of $A$ and $B$ exists if and only if the following category has a terminal object:
\begin{itemize}
\item The objects are triples $(P,p_l: P\to A, p_r:P\to B)$.
\item A morphism from $(P,p_l,p_r)$ to $(Q,q_l,q_r)$ is a morphism $f\in\Ob{\CC}$ such that the following diagram commutes:
\[
\begin{tikzcd}
& P \arrow[ld,swap, "p_l"] \arrow[rd, "p_r"] \arrow[d, "f"] & \\
A & Q \arrow[l, "q_l"]  \arrow[r,swap, "q_r"] & B
\end{tikzcd}
\]
\item The composition and identity is inherented from the structure of $\CC$.
\end{itemize}
\end{exer}

\begin{exer}
  Let $\CC$ be a category with a choice of product $(A\times B, \projl, \projr)$ for any two objects $A,B\in \Ob{\CC}$.
  Given morphisms $f : A \to C$ and $g : B \to D$ in $\CC$, construct a morphism
  \[ f \times g : A \times B \to C \times D.\]
\end{exer}


\begin{exer}
  Let $\CC$ be a category with a choice of product $(A\times B, \projl, \projr)$ for any two objects $A,B\in \Ob{\CC}$.
  For any $A, B \in \Ob\CC$, construct an isomorphism
  \[ A \times B \cong B \times A. \]
\end{exer}


\subsection{(Binary) Coproducts}
\label{sec:coproducts}



\begin{dfn}
   Let $\CC$ be a category and let $A,B \in \Ob\CC$ be objects of $\CC$.

  A triple $(C,\inl : A \to C,\inr : B \to C)$ is called a \textbf{coproduct of $A$ and $B$} if for any triple $(D,i_l : A \to D, i_r : B \to D)$ there is exactly one morphism $f : C \to D$ such that the following diagram commutes:
  \[
    \begin{tikzcd}
      A \ar[r, "\inl"] \ar[rd, "i_l"']
      &
      C  \ar[d, dashed, "f"]
      &
      B \ar[l, "\inr"'] \ar[ld, "i_r"]
      \\
      &
      D %\ar[ld, "q_1"'] \ar[rd, "q_2"] \ar[d, "f"]
      &
      \\
    \end{tikzcd}
  \]
  If $A$ and $B$ have a specified coproduct $(C,\inl : A \to C,\inr : B \to C)$, then the object $C$ is often called $A + B$.
  The morphism $f : A + B \to D$ determined by $(D, i_l, i_r)$ is denoted by $[ i_l, i_r]$.
  
\end{dfn}

\begin{exer}
  Write a sequence of suitable exercises about coproducts.
\end{exer}

\begin{exer}
  Consider the category with one object and the rational numbers $\mathbb{Q}$ as morphisms.
  Can you construct an initial object in this category? A terminal object? Products? Coproducts?
\end{exer}

\begin{rem}
  Initial and terminal objects and products and coproducts are special cases of \textbf{limits and colimits}.
  We are not studying, in these notes, the general notion of (co)limit.
  However, the examples above should suffice for you to understand, in your own time, other (co)limits, such as
  \begin{itemize}
  \item pullbacks and pushouts;
  \item products and coproducts of families of objects (not just of pairs of objects); and
  \item equalizers and coequalizers.
  \end{itemize}
\end{rem}

\begin{reading*}
  We do not discuss here whether/when/how (co)limits can be transported along functors.
  You can find some information on this in \cite[\S 5.3]{leinster}.
\end{reading*}


\section{Functors}\label{sec:functors}
An important aspect in computer programming is the transformation of data. For example, if you have a data type $X$, then one can consider also the data type $\List(X)$ of lists with values in $X$. If one thinks of the objects in a category to be data types, then we can ask even more. If $f:X\to Y$ is a function (between the data types), then this also induces  a function from the $X$-valued lists to the $Y$-valued lists as follows:
\begin{align}\label{eqn:function_on_list}
  \List(f) : \List(X)&\to \List(Y)
  \\
  [x_1,\ldots x_n] &\mapsto [f(x_1),\ldots,f(x_n)]. \notag
\end{align}

\begin{rem}
  The ``\ldots'' above are informal --- a formal definition would define $\List(f)$ by structural recursion on lists, of course.
\end{rem}

A \textit{functor} formalizes this phenomenon:
\begin{dfn} Let $\CC$ and $\DD$ be categories. A \textbf{functor} $F$ from $\CC$ to $\DD$ consists of the following data:
\begin{itemize}
\item A function 
\[
\Ob{\CC} \to \Ob{\DD},
\]
written as $X\mapsto F(X)$.
\item For each $X,Y\in \Ob{\CC}$, a function
\[
\CHom{\CC}{X}{Y} \to \CHom{\DD}{F(X)}{F(Y)},
\]
written as $f\mapsto F(f)$.
\end{itemize}
Moreover, this data should satisfy the following properties:
\begin{itemize}
\item (\textbf{Preserves composition}) For $f\in \Hom[\CC]{X}{Y}$ and $g\in \Hom[\CC]{Y}{Z}$, we have $F(\co f g) = \co {Ff}{Fg}$.
\item (\textbf{Preserves identity}) For $X\in\CC$, we have $F(\Id[X]) = \Id[F(X)]$.
\end{itemize}
\end{dfn}

\begin{exa} \label{example:functor_list} The \textbf{list-functor} (on sets), denoted by $\List$, is the functor from $\SET$ to $\SET$ defined by the following data:
\begin{itemize}
\item The function on objects is given by:
\[
\Ob{\SET}\to \Ob{\SET}: X\mapsto \List(X).
\]
\item For each $X,Y\in\SET$, the function on morphisms is given by
\[
\CHom{\SET}{X}{Y} \to \CHom{\SET}{\List(X)}{\List(Y)}: f\mapsto \List(f),
\]
where $\mathbf{List}(f)$ is given in \cref{eqn:function_on_list}.
\end{itemize}


\begin{comment}
\begin{proof}
The data is clearly well-defined since we work with \textit{mere} sets, i.e. no extra structure. That $\List$ would preserve the identity means that $\List(\Id[X]) = \Id[\List(X)]$, i.e. we have to show that for each $X$-valued list $\ell$, we have:
\[
\List(\Id[X])(\ell)) = \Id[\List(X)](A)(\ell).
\] 
The lefthand-side of the equation is given by:
$$\mathbf{List}(\Id[X])(\{x_i\}_i) = \{\Id[X] (x_i)\}_i = \{x_i\}_i,$$
where the first (resp. second) equality holds by definition of $\mathbf{List}$ on morphisms (resp. by definition of $\Id[X]$).\\
The righthand-side of the equation is given by:
$$\Id[\mathbf{List}(X)](A)(\{x_i\}_i) = \{x_i\}_i,$$
by definition of the identity morphism in $\SET$. Hence, the left and right hand side are equal which shows that $\mathbf{List}$ indeed preserves the identity.\\
We now show that $\mathbf{List}$ preserves composition. Let $f\in \CHom{\SET}{X}{Y}$ and $g\in \CHom{\SET}{Y}{Z}$ be functions. By definition of the composition in $\SET$ and by definition of $\mathbf{List}$ on morphisms, we have for each $X$-valued list $\{x_i\}_i$:
\begin{eqnarray}\label{eqn:functor_list_comp1}
\mathbf{List}(f\Comp g)(\{x_i\}_i) = \{(f\Comp g)(x_i)\}_i = \{g(f(x_i)\}_i.
\end{eqnarray}
Again by definition of the composition in $\SET$ and by definition of $\mathbf{List}$ on morphisms, we have for each $X$-valued list $\{x_i\}_i$:
\begin{eqnarray}\label{eqn:functor_list_comp2}
\left(\mathbf{List}(f)\Comp \mathbf{List}(g)\right)(\{x_i\}_i) = \mathbf{List}(g)\left(\{f(x_i)\}_i\right) = \{g(f(x_i)\}_i.
\end{eqnarray}
Hence, by combining \cref{eqn:functor_list_comp1, eqn:functor_list_comp2}, we conclude that for each $X$-valued list $\{x_i\}_i$ we have
\[
\mathbf{List}(f\Comp g)(\{x_i\}_i) = \left(\mathbf{List}(f)\Comp \mathbf{List}(g)\right)(\{x_i\}_i).
\]
Since this holds for every such list, we indeed have that the composition is preserved.
\end{proof}
\end{comment}
\end{exa}

\begin{exer}
  Show that $\List$ is a  functor, that is, show that it preserves identity and composition of functions.
  Hint: use structural induction on lists.
\end{exer}

\begin{exer}
  Consider the function $\Ob{\Maybe} : \Ob\SET \to \Ob\SET$ sending a set $X$ to $X + \{*\}$.
  For any two sets $X$ and $Y$ and $f : X \to Y$, define a function
  \[ \Maybe(f) : \Ob\Maybe X \to \Ob\Maybe Y\]
  and show that this assignment satisfies the functor laws.
\end{exer}

\begin{exer}
  Let $A \in \Ob\SET$.
  Construct a functor $(\times A) : \SET \to \SET$ that, on objects, is given by
  \[ (\times A) X := X \times A. \]
\end{exer}


\begin{exer}
  Let $\CC$ be a category with chosen products, and let $A \in \Ob\CC$.
  Construct a functor $(\times A) : \CC \to \CC$ that, on objects, is given by
  \[ (\times A) X := X \times A. \]
\end{exer}

\begin{exer}
  Let $A \in \Ob\SET$.
  Construct a functor $(+ A) : \SET \to \SET$ that, on objects, is given by
  \[ (+ A) X := X + A. \]
\end{exer}

\begin{exer}
  Let $\CC$ be a category with chosen coproducts, and let $A \in \Ob\CC$.
  Construct a functor $(+ A) : \CC \to \CC$ that, on objects, is given by
  \[ (+ A) X := X + A. \]
\end{exer}

\begin{exer}
  Let $R \in \Ob\SET$ be a set.
  Construct a functor $(R \to) : \SET \to \SET$ that, on objects, is given by
  \[ (R \to) X := R \to X. \]
\end{exer}

\begin{exer}
  Let $\CC$ be a category and let $R \in \Ob\CC$.
  Construct a functor $\CHom \CC R - : \CC \to \SET$ that, on objects, is given by
  \[ (\CHom \CC R -) X := \CHom \CC R X. \]
\end{exer}


\begin{exer}\label{ex:poset_functors} Let $(X,\leq_X)$ and $(Y,\leq_Y)$ be posets. Can you characterize/describe the functors from $\POS(X,\leq_X)$ to $\POS(Y,\leq_Y)$  ? Before writing out the definitions, what would you expect the answer to be?
\end{exer}

\begin{exer}
  Let $\CC$ be a category with chosen products $(A\times B, \pi_A, \pi_B)$ for any two objects $A$ and $B$.
  Construct a functor
  \[ (\times) : \CC\times \CC \to \CC\]
  from the product category $\CC\times \CC$ to $\CC$.
  The objects of $\CC\times \CC$ are pairs of objects in $\CC$, and morphisms $\CHom{(\CC\times\CC)}{(X,X')}{(Y,Y')}$ are pairs $(f : X \to Y, f' : X' \to Y')$ of morphisms in $\CC$.
\end{exer}

\begin{exer}
  Let $\CC$ be a category with chosen coproducts $(A + B, \iota_A, \iota_B)$ for any two objects $A$ and $B$.
  Construct a functor
  \[ (+) : \CC\times \CC \to \CC\]
  from the product category $\CC\times \CC$ to $\CC$.
\end{exer}


\begin{exer}\label{ex:monoid_functors} Let $(M,m,e)$ be a monoid and let $\MON(M,m,e)$ be its corresponding category as defined in \cref{monoidcategory}. Can you characterize/describe the functors from $\MON(M,m,e)$ to $\SET$?
\end{exer}

\subsection{Categories as objects of a category?}
Notice that a functor is a function between categories which preserves the structure of a category. So by the \textit{philosophy} of category theory, this would define a category whose objects are categories and whose morphisms are functors. In order to make this precise, we would also need a \textit{identity functor} and we should have a \textit{composition of functors}.

\begin{exa}\label{example:functor_id} Let $\CC$ be a category. The \textbf{identity functor on $\CC$}, denoted by $\Id[\CC]$, is the functor specified by the following data:
\begin{itemize}
\item The function on objects is given by
\[
\Ob{\CC}\to \Ob{\CC}: X\mapsto X.
\]
\item For each $X,Y\in\CC$, the function on morphisms is given by
\[
\CHom \CC X Y\to \CHom \CC X Y: f\mapsto f.
\]
\end{itemize}
\end{exa}

\begin{exer} Show that $\Id[\CC]$ (defined in \cref{example:functor_id}) satisfies the properties of a functor, i.e. $\Id[\CC]$ is indeed a functor.
\end{exer}

\begin{exa}\label{example:functor_comp} Let $\CC,\DD$ and $\EE$ be  categories and $F:\CC\to\DD$ and $G:\DD\to\EE$ functors. The \textbf{composition functor of $F$ with $G$}, denoted by $F\Comp G$, is the functor specified by the following data:
\begin{itemize}
\item The function on objects is given by
\[
\Ob{\CC}\to \Ob{\EE}: X\mapsto G(F(X)).
\]
\item For each $X,Y\in\CC$, the function on morphisms is given by
\[
\CHom \CC X Y\to \CHom{\EE}{G(F(X))}{G(F(Y))}: f\mapsto G(F(f)).
\]
\end{itemize}
\end{exa}

\begin{exer} Show that $F\Comp G$ (defined in \cref{example:functor_comp}) satisfies the properties of a functor, i.e. $F\Comp G$ is indeed a functor.
\end{exer}

\begin{dfn} The \textbf{Category of categories}, denoted by $\CAT$, is the category specified by the following data:
\begin{itemize}
\item An object is a category.
\item If $\CC, \DD\in\CAT$ are categories, then is $\CHom \CAT \CC \DD$ the collection of all functors from $\CC$ to $\DD$.
\item The identity morphism on a category $\CC$ is the identity functor on $\CC$ defined in \cref{example:functor_id}.
\item The composition of morphisms, i.e. functors, is the composition of functors defined in \cref{example:functor_comp}.
\end{itemize} 
\end{dfn}

\begin{exer} Show that $\CAT$ satisfies the property of a category, i.e. $\CAT$ is indeed a category.
\end{exer}

\begin{rem}
  When showing that $\CAT$ is a category, one is forced to consider \emph{equality of objects} when showing that two functors are equal.
  This goes against the spirit of category theory, where we only ever consider \emph{equality of (parallel) morphisms}.
  We want to consider two objects ``the same'' when they are \emph{isomorphic}, not when they are \emph{equal}. Of course, any two equal objects are isomorphic to each other, but not the other way round; for instance, in the category of sets, the cartesian product $A \times B$ is isomorphic to $B \times A$, but they are not equal.

  To stay within the spirit of category theory, one can instead consider $\CAT$ as a \textbf{bicategory}.%
  \footnote{See, e.g.,   \url{https://ncatlab.org/nlab/show/bicategory\#detailedDefn} for a definition of bicategories.}
  In a bicategory, one has one more layer of things: objects, morphisms, and 2-cells between parallel morphisms. One also calls objects ``0-cells'' and morphisms ``1-cells'', for consistency.
  Importantly, in a bicategory, the laws concerning 1-cells (as stated in \cref{dfn:category}) do not hold up to equality, but only up to isomorphism of 2-cells.

  An important example is the bicategory given by the following data, which we only list partially:
  \begin{enumerate}
  \item 0-cells are categories;
  \item 1-cells are functors;
  \item 2-cells are natural transformations (see \cref{sec:nat-trans});
  \item composition and identity of 1-cells is composition and identity of functors.
  \end{enumerate}

  We do not delve into bicategories in these notes; an introductory text is, for instance, Leinster's \cite{leinster:basic-bicats}.
\end{rem}




\section{Natural transformations}
\label{sec:nat-trans}

\begin{dfn} Let $F,G: \CC\to\DD$ be functors. A \textbf{natural transformation} $\alpha$ from $F$ to $G$ consists of the following data:
\begin{itemize}
\item For each $X\in \Ob{\CC}$, a morphism $\alpha_X \in \CHom{\DD}{F(X)}{G(X)}$.
\end{itemize}
Moreover, this data should satisfy the following \textit{naturality condition}:\\
For each $f\in \CHom{\CC}{X}{Y}$, the following diagram should commute:
\begin{center}
\begin{tikzcd}
F(X) \arrow[r, "\alpha_X"] \arrow[d,swap, "F(f)"] & G(X) \arrow[d, "G(f)"]\\
F(Y) \arrow[r,swap, "\alpha_Y"] & G(Y)
\end{tikzcd}
\end{center}
Moreover, we call $\alpha$ a \textbf{natural isomorphism} if for each $X\in\Ob{\CC}$, we have that $\alpha_X$ is an isomorphism in $\DD$.
\end{dfn}

\begin{nota} If $F,G:\CC\to\DD$ are functors. A natural transformation $\alpha$ from $F$ to $G$, is denoted as $\NatTrans{\alpha}{F}{G}$ or 
\begin{center}
\begin{tikzcd}[column sep=huge]
\CC
  \arrow[bend left=50]{r}[name=U,label=above:$\scriptstyle F$]{}
  \arrow[bend right=50]{r}[name=D,label=below:$\scriptstyle U$]{} &
\DD
  \arrow[shorten <=5pt, Rightarrow,to path={(U) -- node[label=right:$\alpha$] {} (D)}]{}
\end{tikzcd}
\end{center}
\end{nota}

\begin{exa} (\textbf{Currying}) Let $X$ be a set. Let $F := \SET(X, -)\times X : \SET\to\SET$ be the functor induced by the following data (on objects):
\[
Y\mapsto \SET(X,Y)\times X.
\]
The evaluation defines a natural transformation $\NatTrans{ev}{F}{\Id[\SET]}$ as follows:
\[
ev_Y : \SET(X,Y) \times X \to Y : (f,x) \mapsto f(x).
\]
Show that this indeed satisfies the naturality condition.
\end{exa}

\subsection{Functor categories}
\begin{dfn}\label{dfn:nattrans_id} Let $F:\CC\to\DD$ be a functor. The \textbf{identity natural transformation} $\Id[F]$ on $F$ is given by the following data:
\[
\forall X\in\Ob{\CC}: (\Id[F])_{X} := \Id[F(x)].
\]
\end{dfn}

\begin{exer} Show that for any functor $F:\CC\to\DD$, the identity natural transformation $\Id[F]$ satisfies the properties of a natural transformation.
\end{exer}

\begin{dfn}\label{dfn:nattrans_comp} Let $F,G,H: \CC\to\DD$ be functors and $\NatTrans{\alpha}{F}{G}$, $\NatTrans{\beta}{G}{H}$ be natural transformations. The \textbf{(vertical) composition} of $\alpha$ and $\beta$ is the natural transformation $\Comp{\alpha}{\beta}$ is given by the following data:
\[
\forall X\in\Ob{\CC}: (\co{\alpha}{\beta})_{X} := \co{\alpha_X}{\beta_X}.
\]
\end{dfn}

\begin{exer} Show that for any functors $F,G,H: \CC\to\DD$ and $\NatTrans{\alpha}{F}{G}$, $\NatTrans{\beta}{G}{H}$ natural transformations, the (vertical) composition of $\alpha$ and $\beta$ satisfies the properties of a natural transformation.
\end{exer}

\begin{dfn} Let $\CC,\DD$ be categories. The \textbf{category of functors} or the \textbf{functor category} from $\CC\to\DD$, denoted by $Fun(\CC,\DD)$ or $[\CC,\DD]$, is given by the following data:
\begin{itemize}
\item An object is a functor $F:\CC\to\DD$.
\item A morphism from $F$ to $G$ is a natural transformation $\NatTrans{\alpha}{F}{G}$.
\item The identity morphism on $F$ is given by the identity natural transformation $\Id[F]$ defined in \cref{dfn:nattrans_id}.
\item The composition of $F$ and $G$ is given by the composition $\co{\alpha}{\beta}$ defined in \cref{dfn:nattrans_comp}.
\end{itemize}
\end{dfn}

\begin{exer} Show that for any two categories $\CC$ and $\DD$, the functor category from $\CC$ to $\DD$ satisfies the properties of a category.
\end{exer}

\begin{dfn}\label{dfn:nattrans_horcomp} Let $F,G : \CC\to\DD$ and $\tilde{F},\tilde{G}:\DD\to\EE$ be functors and $\NatTrans{\alpha}{F}{G}, \NatTrans{\beta}{\tilde{F}}{\tilde{G}}$ be natural transformations. The \textbf{horizontal composition} (also called the \textbf{Godement product}) of $\alpha$ and $\beta$, denoted by $\beta \bullet \alpha$, is defined as:
\begin{equation}\label{eqn:nattrans_horcomp}
\forall X\in \Ob{\CC}: (\beta\bullet\alpha)_X := \co{\tilde{F}(\alpha_X)}{\beta_{G(X)}}.
\end{equation}
\end{dfn}

\begin{exer} Show that $\alpha\bullet\beta$ (defined as in \cref{dfn:nattrans_horcomp}), is indeed a natural transformation.
\end{exer}

\begin{exer} Show the following property: 
\[
\forall X\in \Ob{\CC}: (\beta\bullet\alpha)_X = \co{\beta_{F(X)}}{\tilde{G}(\alpha_X)}.
\]
Hint: Write the equality as a (not-known commutative) square.
\end{exer}

\subsection{Exercises}
\begin{exer} Let $(M,m,e)$ be a monoid and let $\MON(M,m,e)$ be its corresponding category. Recall from \cref{ex:monoid_functors} that a functor from $\mathcal{M}$ to $\SET$ is a set $X$ together with an action of $M$ on $X$, i.e. a function $\mu: M\times X\to X$ such that 
\[
\forall x\in X: \mu(e,x) = x, \quad \forall n_1,n_2\in M, x\in X: \mu(n_1, \mu(n_2,x)) = \mu(m(n_1,n_2), x).
\]
We will call a set $ X $ with an action of $ M $ on $ X $ an $ M $-set.
Characterize the natural transformations between $M$-sets.
\end{exer}

\begin{exer} Let $(X,\leq_X)$ and $(Y,\leq_Y)$ be posets. Recall from \cref{ex:poset_functors} that a functor between posets corresponds with an order-preserving function, i.e. $x_1 \leq_X x_2 \implies f(x_1) \leq_Y f(x_2)$. Characterize the natural transformations between order-preserving functions.
\end{exer}

\subsection{Equivalence of categories}
Recall that objects $X,Y\in\Ob{\CC}$ are isomorphic if there exist morphisms $f\in\CHom{C}{X}{Y}$ and $g\in\CHom{C}{Y}{X}$ such that $\co{f}{g} = \Id[X]$ and $\co{g}{f} = \Id[Y]$. So in particular we have the notion of an isomorphism in the category $\CAT$ of categories. Spelled out, this means categories $\CC$ and $\DD$ are isomorphic if there exist functors $F:\CC\to\DD$ and $G:\CC\to\DD$ such that $\co{F}{G}= \Id[\CC]$ and $\co{G}{F} = \Id[\DD]$.\\
However, the following exercise shows that isomorphism of categories is not the correct notion of \textit{equivalence/sameness} between categories:\\
Let $\FINSET$ be the category whose objects are given by finite sets and whose morphisms are given by functions\footnote{Notice that the objects of $\FINSET$ form a subset of the objects of $\SET$, but given any two finite sets $X,Y in \Ob{\FINSET}$, we have $\CHom{FINSET}{X}{Y}$ = $\CHom{SET}{X}{Y}$. We say in this case that $\FINSET$ is a (full) subcategory of $\SET$.}. That this is a category follows since $\SET$ is a category.\\
Let $\Catb{FinOrd}$ be the category whose objects are given by sets of the form 
\[
[n] := \left\{0,1,\cdots,n-1\right\},
\]
and whose morphisms are given by functions between these sets.\\
Every finite set $X$ is always in bijection with a set of the form $[n]$ (where $n$ is the size $\vert X\vert$ of $X$). For each set $X$, we fix a bijection $\phi^X: X\to [\vert X\vert]$. Consequently, we have a functor:
\begin{dfn} Let $U: \FINSET\to \Catb{FinOrd}$ be the functor specified by the following data:
\begin{itemize}
\item For $X\in \Ob{\FINSET}$, we define $U(X) := [\vert X\vert]$.
\item For $f\in \CHom{\FINSET}{X}{Y}$, we define $U(f): [\vert X\vert]\to [\vert Y\vert]$ as the unique function such that the following diagram commutes:
\begin{center}
\begin{tikzcd}
X \arrow[r, "\phi^X"] \arrow[d,swap, "f"] & {[\vert X\vert]} \arrow[d, "U(f)"] \\
Y \arrow[r,swap, "\phi^Y"] & {[\vert Y\vert]}
\end{tikzcd}
\end{center}
\end{itemize}
\end{dfn}

\begin{exer} Show that $U: \FINSET\to \Catb{FinOrd}$ is indeed a functor. In particular, you have to show that $U$ is well-defined on the morphisms.
\end{exer}

In order to show that $U$ is not an isomorphism, one can use the following lemma/exercise:
\begin{exer} Show that a functor $F:\CC\to\DD$ is an isomorphism if and only if $F$ satisfies the following properties:
\begin{itemize}
\item $F$ is injective on objects, i.e. 
\[
\forall X,Y\in\Ob{\CC}: F(X) = F(Y) \implies X=Y.
\]
\item $F$ is surjective on objects, i.e. 
\[
\forall Y\in\Ob{\DD}: \exists X\in\Ob{\CC} : F(X) = Y.
\]
\item $F$ is faitful, i.e. the following functions are injective
\[
\forall X,Y\in\Ob{\CC}: \CHom{\CC}{X}{Y} \xrightarrow{F_{X,Y}} \CHom{\DD}{F(X)}{F(Y)} : f\mapsto F(f)
\]
\item $F$ is full, i.e. for all $X,Y\in \Ob{\CC}$, $F_{X,Y}$ is surjective.
\end{itemize}
\end{exer}

\begin{exer} Show that $U$ is not an isomorphism, i.e. state which part of an isomorphism fails and give a concrete example that it fails.
\end{exer}

\begin{rem} So the problem with $U$ (in the sense that it is not an isomorphism) is that multiple (finite) sets are mapped to the same set. For this reason, a good notion of equivalence between categories should not be injective on objects. Also, which is not clear from this example, we should also weaken the condition of $F$ being surjective on objects. Instead, we need that $F$ is \textbf{essentially surjective on objects}:
\[
\forall Y\in\Ob{\DD}: \exists X\in\Ob{\CC} : F(X) \cong Y.
\]
\end{rem}
So motivated by the remark, we define:
\begin{dfn} Categories $\CC$ and $\DD$ are \textbf{equivalent} if there exists a pair of functors $(F:\CC\to\DD, G:\DD\to\CC)$ such that there exists natural isomorphisms 
\[
\co{F}{G} \to \Id[\CC], \quad \co{G}{F} \to \Id[\DD]
\]
\end{dfn}

So although $U$ is not an isomorphism, it does induce an equivalence of categories:
\begin{exer} Show that $U$ induces an equivalence of categories.
\end{exer}

\begin{exer} Let $\CC$ be the category whose objects are categories with a unique object and whose morphisms are functors between these one-object categories, i.e. $\CC$ is the (full) subcategory of $\CAT$ generated by the categories with a unique object. Show that $\CC$ is equivalent to the category $\MON$ of monoids. \\
What happens if we do not consider $\CC$ to consist of those categories with a unique object, but with a unique object up to isomorphism, i.e. $\CC$ is the category whose objects are categories $\DD$ which satisfy the following property: 
\[
\forall X,Y \in \Ob{\DD}: X\cong Y.
\]
\end{exer}

The following exercise gives a characterization of a functor being an equivalence. However, in order to show this, one has to use the axiom of choice which means (informally) that if the following property holds:
\[ 
\exists x: P(x),
\]
then we can fix some $x$ such that $P(x)$ holds.
\begin{exer} Show that a functor $F:\CC\to\DD$ induces an equivalences of categories if and only if it is essentially surjective on objects and fully faithful.
\end{exer}


\onlydraft{\section{Adjunctions}
\label{sec:adjunctions}

\begin{dfn} A pair $(F,G)$ of functors $F : \CC\to\DD, G:\DD\to\CC$ is called an \textbf{adjoint pair} if for every objects $X\in\Ob{\CC}$ and $Y\in\Ob{\DD}$, there exists a bijection 
\[
\alpha_{X,Y} : \CHom{\DD}{F(X)}{Y} \to \CHom{\CC}{X}{G(Y)},
\]
which are moreover natural in both $X$ and $Y$, i.e. for each $f\in\CHom{\CC}{X_1}{X_2}$ and $g\in\CHom{\DD}{Y_1}{Y_2}$, the following diagrams commute:
\begin{eqnarray}
\begin{tikzcd}
\CHom{\DD}{F(X_2)}{Y} \arrow[r, "\alpha_{X_2,Y}"] \arrow[d,swap, "\co{F(f)}{-}"] & \CHom{\CC}{X_2}{G(Y)} \arrow[d, "\co{f}{-}"] \\
\CHom{\DD}{F(X_1)}{Y} \arrow[r,swap, "\alpha_{X_1,Y}"] & \CHom{\CC}{X_1}{G(Y)}
\end{tikzcd}\\
\begin{tikzcd}
\CHom{\DD}{F(X)}{Y_1} \arrow[r, "\alpha_{X,Y_1}"] \arrow[d,swap, "\co{-}{G(g)}"] & \CHom{\CC}{X}{G(Y_1)} \arrow[d, "\co{-}{g}"] \\
\CHom{\DD}{F(X)}{Y_2} \arrow[r,swap, "\alpha_{X,Y_2}"] & \CHom{\CC}{X}{G(Y_2)}
\end{tikzcd}
\end{eqnarray}
If $(F,G)$ is an adjoint pair, we call $F$ the \textbf{left adjoint} of $G$ and we call $G$ the \textbf{right adjoint} of $F$ and we denote $F \dashv G$.
\end{dfn}

\begin{exer} Let $F:\CC\to\DD$ be a functor. Show that if $F$ has a left (resp. right) adjoint $G$, then $G$ must be unique up to isomorphism\footnote{Isomorphism w.r.t the functor category.}.
\end{exer}

\begin{exer} Let $U:\MON\to \SET$ be the forgetful functor (defined in \cref{example:forgetful_montoset}) which maps a monoid to its underlying set and let $F : \SET\to\MON$ be the functor which maps a set to the free monoid of this set (defined in \cref{exa:freemonoids}). Then is $(F,U)$ an adjoint pair. (Hint: Use \cref{prop:UVP_forget_montoset}). 
\end{exer}

\begin{exer}\label{exer:adjunction_homtensor_currying} Let $Y$ be a set. Show that the functor (induced by the following mapping on objects)
\[
- \times Y : \SET\to \SET: X\mapsto X\times Y,
\]
has a right adjoint, which is given by the functor (induced by the following mapping on objects)
\[
\CHom{\SET}{Y}{-} : \SET\to\SET : X\mapsto \CHom{\SET}{Y}{X}.
\]
Is there an analogous statement in the category $\LEAN$ instead of $\SET$?
\end{exer}

\begin{thm} Let $(F,G)$ be a pair of functors $F : \CC\to\DD, G: \DD\to\CC$. The following statements are equivalent:
\begin{enumerate}
\item $(F,G)$ is an adjoint pair.
\item There exists natural transformations 
\[
\NatTrans{\eta}{\Id[\CC]}{\co{F}{G}}, \quad \NatTrans{\epsilon}{\co{G}{F}}{\Id[\DD]},
\]
such that for all $X\in\CC$ and $Y\in\DD$ the following diagrams commute:
\[
\begin{tikzcd}
F(X) \arrow[r,"{F(\eta_X)}"] \arrow[rd,swap, "{\Id[F(X)]}"] & F(G(F(X))) \arrow[d, "{\epsilon_{F(X)}}"] \\
& F(X)
\end{tikzcd} \quad 
\begin{tikzcd}
G(Y) \arrow[r,"{\eta_{G(Y)}}"] \arrow[rd,swap, "{\Id[G(Y)]}"] & G(F(G(Y))) \arrow[d, "{G(\epsilon_{Y})}"] \\
& G(Y)
\end{tikzcd}
\]
\end{enumerate}
In case $(F,G)$ satisfies these (equivalent) conditions, we call $\eta$ the \textbf{unit of the adjunction} and $\epsilon$ the \textbf{counit of the adjunction}. The equalities in condition $2$ are called the \textbf{triangle identities}.
\begin{proof}
First, assume that $(F,G)$ is an adjoint pair. We have to define the unit and counit and show that the triangle identities hold:
\begin{itemize}
\item \textbf{Unit}: For each $X\in\Ob{\CC}$, we should first define $\eta_X \in \CHom{\CC}{X}{G(F(X))}$. Since $F \dashv G$, we have a bijection
\[
\alpha_{X,FX} : \CHom{\DD}{FX}{FX} \to \CHom{\CC}{X}{G(F(X))},
\]
hence, we define $\eta_X := \alpha_{X,FX}(\Id[FX])$. We now show that $(\eta_X)_{X\in\Ob{\CC}}$ forms a natural transformation: Assume $f\in\CHom{\CC}{X}{Y}$. We have to show that the following diagram commutes:
\[
\begin{tikzcd}
X \arrow[rr, "{\alpha_{X,FX}(\Id[FX])}"] \arrow[d,swap, "f"] && G(F(X)) \arrow[d, "G(F(f))"] \\
Y \arrow[rr, swap, "{\alpha_{Y,FY}(\Id[FY])}"] && G(F(Y))
\end{tikzcd}
\]
That this square is indeed commutative follows from the following computation:
\begin{equation}\label{eqn:unitnaturality_fromhomsetadj}
\co{f}{\alpha(\Id[FY])} = \alpha(\co{F(f)}{\Id[FY]}) = \alpha(F(f)) =  \alpha(\co{\Id[FX]}{F(f)}) = \co{\alpha(\Id[Fx])}{G(F(f))},
\end{equation}
where the first and last equality hold by naturality of $\alpha$.
\item \textbf{Counit}: For each $Y\in\Ob{\DD}$, we the counit $\epsilon_Y \in \CHom{\DD}{F(G(Y))}{Y}$ is defined as the image of $\Id[G(Y)]$ of the bijection
\[
\alpha^{-1}_{GY,Y} : \CHom{\CC}{GY}{GY}\to \CHom{\DD}{F(G(Y))}{Y}.
\]
That $\epsilon$ indeed forms a natural transformation is analogous to the computation in \cref{eqn:unitnaturality_fromhomsetadj}.
\item \textbf{Triangle identities}: Both the triangle identities are proved analogously, hence we will only show the first triangle identity, i.e. $\Id[FX] = \co{F(\eta_X)}{\epsilon_{FX}}$. Unfolding the definition of $\epsilon$, this is equivalent to showing:
\[
\Id[FX] = \co{F(\eta_X)}{\alpha^{-1}_{GFX,FX}(\Id[GF(X)])}.
\]
Since the components of $\alpha$ are bijections, this is equivalent to showing 
\[
\alpha(\Id[FX]) = \alpha(\co{F(\eta_X)}{\alpha^{-1}_{GFX,FX}(\Id[GF(X)])})
\]
This indeed holds by the following computation:
\[
\alpha(\co{F(\eta_X)}{\alpha^{-1}_{GFX,FX}(\Id[GF(X)])}) = \co{\eta_X}{\alpha(\alpha^{-1}(\Id[GFX]))} = \co{\eta_X}{\Id[GFX]} = \eta_X,
\]
where the first (resp. second) equality holds by naturality (resp. bijectiveness) of $\alpha$.
\end{itemize}
This concludes the proof of $(1)\implies (2)$. Now assume that $(2)$ holds. We have to construct bijections
\[
\alpha_{X,Y} : \CHom{\DD}{F(X)}{Y} \to \CHom{\CC}{X}{G(Y)},
\]
which are natural in $X$ and $Y$. Let $g\in \CHom{\DD}{F(X)}{Y}$. Define $\alpha_{X,Y}(g)$ as the composite:
\[
X \xrightarrow{\eta_X} G(F(X)) \xrightarrow{G(g)} G(Y).
\]
For the other direction, let $f\in \CHom{\CC}{X}{G(Y)}$. Define $\alpha^{-1}_{X,Y}(f)$ as the composite:
\[
FX \xrightarrow{F(f)} F(G(Y)) \xrightarrow{\epsilon_Y} Y.
\]
That $\alpha$ and $\alpha^{-1}$ are inverses of each other follows from the following computation:
\begin{eqnarray*}
\alpha(\alpha^{-1}(f)) = \alpha(\co{F(f)}{\epsilon_Y}) =& \co{\eta_X}{G(\co{F(f)}{\epsilon_Y})} &\text{ by definition, }\\
	=& \co{\eta_X}{(\co{GF(f)}{G\epsilon_Y})} &\text{ by functoriality of $G$},\\
	=& \co{(\co{\eta_X}{GF(f)})}{G\epsilon_Y} &\text{ by associativity},\\
	=& \co{(\co{f}{\eta_{GY})}}{G\epsilon_Y} &\text{ by naturality of $\eta$}\\
	=& \co{f}{(\co{\eta_{GY}}{G\epsilon_Y})} &\text{ by associativity}\\
	=& f &\text{by triangle identity}.
\end{eqnarray*}
The other equality is shown analogous by using functoriality of $F$, naturality of $\epsilon$ and the other triangle identity.

It remains to show the naturality of $\alpha$ in both $x$ and $y$ which is left to the reader as a good exercise on diagram chasing. %\KW{TODO: Give the proof of the naturality}
\end{proof}
\end{thm}
%%% Local Variables:
%%% mode: latex
%%% TeX-master: "CT4P"
%%% End:
}

\onlydraft{
\section{Monads and Effects}
\label{sec:monads}
Recall that a monad (as defined, e.g., in Haskell) is a function |m :: * -> *| together with the additional data of a function |pure :: a -> m a| (for each type |a|) and a function |(>>=) :: m a -> (a -> m b) -> m b| (for types |a| and |b|).

The operations |pure| and |(>>=)| are expected to satisfy the following laws:
such that they satisfy the following properties:
\begin{enumerate}
\item |t >>= pure == t| %$\bind(X, pure) = X$.
\item |pure(x) >>= f == f x| %$bind(pure(X), f) = f(X)$.
\item |(t >>= f) >>= g == t >>= (\x -> f x >>= g)|% $bind(bind(X,f), g) = bind(X, \lambda x. bind(f(x), g)$.
\end{enumerate}
However, these laws cannot be enforced in Haskell, since Haskell does not have any infrastructure for logic.

In a category, however, we can define monads including the monad laws.
We will actually give two different definitions of monad;
one called ``Kleisli triple'' (\cref{def:kleisli-triple}), which corresponds to what is called ``monad'' in Haskell,
and one called ``monad'' (\cref{def:monad}). 
The formulation of monads uses that the arguments to |(>>=)| can be reordered.

\begin{dfn}\label{def:kleisli-triple}
  A \textbf{Kleisli triple} over a category $\CC$ is  consisting of the following data:
\begin{itemize}
\item A function $T: \Ob{\CC}\to \Ob{\CC}$.
\item For each $X\in\Ob{\CC}$, a morphism $\eta_X \in \CHom{\CC}{X}{T(X)}$.
\item For each $f\in\CHom{\CC}{X}{T(Y)}$, a morphism $f^{*} \in \CHom{\CC}{T(X)}{T(Y)}$.
\end{itemize}
such that the following properties holds:
\begin{enumerate}
\item For each $X \in \Ob\CC$, we have $\eta_X^{*} = \Id[T(X)]$.
\item For each $f\in\CHom{\CC}{X}{T(Y)}$, the following diagram commutes:
\begin{center}
\begin{tikzcd}
X \arrow[r, "{\eta_X}"] \arrow[rd,swap,"f"] & T(X) \arrow[d, "f^{*}"] \\
& T(Y)
\end{tikzcd}
\end{center}
\item For each $f\in\CHom{\CC}{X}{T(Y)}$ and $g\in\CHom{\CC}{Y}{T(Z)}$, the following diagram commutes:
\begin{center}
\begin{tikzcd}
T(X) \arrow[r, "f^{*}"] \arrow[rd,swap,"{(\co{f}{g^{*}})^{*}}"] & T(Y) \arrow[d, "g^{*}"] \\
& T(Z)
\end{tikzcd}
\end{center}
\end{enumerate}
We denote a Kleisli triple as $(T,\eta, (-)^{*})$.
\end{dfn}

\begin{exer}
  Convince yourself that the operations and laws of a Kleisli triple correspond, in the category $\HASK$, to the operations and properties of a monad in Haskell.
\end{exer}

\begin{exer}[\cref{sol:kleisli_triple_list}]\label{exer:kleisli_triple_list} Show how the following assignment induces a Kleisli triple over the category $\SET$:
\[
X\mapsto \List(X).
\]
The resulting monad is called the \textbf{List monad}.
\end{exer}

\begin{exer}[\cref{sol:kleisli_triple_bintree}]\label{exer:kleisli_triple_bintree} Show how the following assignment induces a Kleisli triple over the category $\SET$:
\[
X\mapsto \BinTree(X),
\]
where $\BinTree(X)$ is the set of binary trees labelled with elements from $X$ at the leaves, that is the set inductively generated by the constructors $leaf: X\to \BinTree(X)$ and $branch : \BinTree(X)\to \BinTree(X)\to \BinTree(X)$.
The resulting monad is called the \textbf{Tree monad}.
\end{exer}



\begin{exer}[\cref{sol:kleisli_triple_maybe}]\label{exer:kleisli_triple_maybe} Let $E$ be a set (considered as a set of \textit{exceptions}). Show how the following assignment induces a Kleisli triple over the category $\SET$:
\[
X\mapsto (X + E),
\]
The resulting monad is called the \textbf{Exception monad}.
\end{exer}



\begin{exer}[\cref{sol:kleisli_triple_nondeterminism}]\label{exer:kleisli_triple_nondeterminism} Show how the following assignment induces a Kleisli triple over the category $\SET$:
\[
X\mapsto \mathbb{P}_{fin}(X) := \left\{A\subseteq X \mid  A \text{ is finite}\right\}.
\]
The resulting monad is called the \textbf{Monad of nondeterminism}.
\end{exer}


\begin{exer}[\cref{sol:kleisli_triple_continuation}]\label{exer:kleisli_triple_continuation} Let $R$ be a set (considered as a set of \textit{results}). Show how the following assignment induces a Kleisli triple over the category $\SET$:
\[
X\mapsto Cont^R(X) := (X \to R) \to R.
\]
The resulting monad is called the \textbf{Continuation monad}.
\end{exer}


\begin{exer}\label{exer:kleisli_triple_familiesofelements} Let $R$ be a set. Show how the following assignment induces a Kleisli triple over the category $\SET$: 
\[
X \mapsto R \to X
\]
The resulting monad is called the \textbf{Monad of families of elements}.

\end{exer}



The notion of a Kleisli triple can equivalently be described  as follows:
\begin{dfn}\label{def:monad}
A \textbf{monad} over a category $\CC$ consists of the following data:
\begin{itemize}
\item A (endo)functor $T:\CC\to\CC$.
\item A natural transformation $\NatTrans{\eta}{\Id[\CC]}{T}$.
\item A natural transformation (``multiplication'') $\NatTrans{\mu}{\co{T}{T}}{T}$.
\end{itemize}
such that for each $X\in\Ob{\CC}$ the following diagrams commute:
\begin{center}
\begin{tikzcd}
T^3(X) \arrow[r, "\mu_{T(X)}"] \arrow[d,swap, "T(\mu_X)"] & T^2(X) \arrow[d, "\mu_X"] \\
T^2(X) \arrow[r,swap, "\mu_X"] & T(X)
\end{tikzcd}
\quad
\begin{tikzcd} 
T(X) \arrow[r, "{\eta_{T(X)}}"] \arrow[rd,swap, "{\Id[T(X)]}"] 
& T^2(X) \arrow[d,"{\mu_X}"] & T(X) \arrow[l,swap,"{T(\mu_X)}"] \arrow[ld, "{\Id[T(X)]}"] \\
& T(X) &
\end{tikzcd}

\end{center}
where we denote $T^2 := \co{T}{T}$ and $T^3 := \co{T}{\co{T}{T}}$.
\end{dfn}

\begin{exer} Given a monad, construct a Kleisli triple from it.
Conversely, given a Kleisli triple, construct a monad from it.
\end{exer}

\begin{exer}
  For each of the Kleisli triples above, describe the monad multiplication $\mu$ obtained from it.
\end{exer}

Every Kleisli triple induces a category:
\begin{dfn} Let $(T,\eta, (-)^{*})$ be a Kleisli triple over $\CC$. The \textbf{Kleisli category}, denoted by $\CC_{T}$, is the category defined by the following data:
\begin{itemize}
\item $\Ob{(\CC_T)} := \Ob{\CC}$.
\item For each $X,Y\in\Ob{(\CC_T)}$, $\CHom{\CC_T}{X}{Y} := \CHom{\CC}{X}{TY}$.
\item The identity on $X\in\Ob{(\CC_T)}$ is $\eta_X$.
\item The composition of $f\in \CHom{\CC_T}{X}{Y}$ and $f\in \CHom{\CC_T}{Y}{Z}$ is $\co{f}{g^{*}}$.
\end{itemize}
\end{dfn}

\begin{exer} Show that for every Kleisli triple, its Kleisli category satisfies the properties of a category.
\end{exer}


%%% Local Variables:
%%% mode: latex
%%% TeX-master: "CT4P"
%%% End:
}

\onlydraft{
\section{Inductive Datatypes and Initial Algebras}
\label{sec:initial-algs}

\subsection{Examples}
\label{sec:examples}



\begin{exer}
  Consider the datatype
\begin{lstlisting}[mathescape=true]
data $\NN$ ::=
| $\Zero$ : $\NN$
| $\Succ$ : $\NN \to \NN$
\end{lstlisting}
and the following  category:
  \begin{itemize}
  \item Objects are triples $(X, z \in X, s : X \to X)$ with $X$ a set/type;
  \item Morphisms from $(X, z \in X, s : X \to X)$ to $(X', z' \in X', s' : X' \to X')$ are functions
    $f : X \to X'$ such that the following diagrams commute:
    \[
      \begin{tikzcd}
        1 \ar[r, "z"] \ar[rd, "z'"']
        &
        X \ar[d, "f"]
        \\
        &
        X'
      \end{tikzcd}
      \quad
      \begin{tikzcd}
        X \ar[r, "s"] \ar[d, "f"']
        &
        X \ar[d, "f"]
        \\
        X' \ar[r, "s'"]
        &
        X'
      \end{tikzcd}
    \]
  \item Composition and identity are given by composition of functions in $\SET$.
    (Check that this is well-defined, that is, that the composition of two functions making the above diagrams commute makes the right diagrams commute again.)
  \end{itemize}

  Show that the triple $(\NN, \Zero, \Succ)$ is an initial object in this category.
\end{exer}

\begin{exer}
  Consider the datatype
\begin{lstlisting}[mathescape=true]
data Exp ::=
| Int     : $\mathbb{Z}$ $\to$ Exp
| Plus    : Exp $\times$ Exp $\to$ Exp
| Squared : Exp $\to$ Exp
\end{lstlisting}
  and consider the following category:
  \begin{itemize}
  \item Objects are quadruples $(X, I : \mathbb{Z} \to X, P : X \to X \to X, S : X \to X)$ with $X$ a set/type;
  \item Morphisms from $(X, I, P, S)$ to $(X', I', P', S')$ are functions
    $f : X \to X'$ such that the following diagrams commute:
    \[
      \begin{tikzcd}
        \mathbb{Z} \ar[r, "I"] \ar[rd, "I'"']
        &
        X \ar[d, "f"]
        \\
        &
        X'
      \end{tikzcd}
      \quad
      \begin{tikzcd}
        X\times X \ar[r, "P"] \ar[d, "f \times f"']
        &
        X \ar[d, "f"]
        \\
        X' \times X' \ar[r, "P'"]
        &
        X'
      \end{tikzcd}
      \quad
      \begin{tikzcd}
        X \ar[r, "S"] \ar[d, "f"']
        &
        X \ar[d, "f"]
        \\
        X' \ar[r, "S'"]
        &
        X'
      \end{tikzcd}
    \]
  \item Composition and identity are given by composition of functions in $\SET$. (Check that this is well-defined.)
  \end{itemize}
  
    Show that the quadruple consisting of the type |Exp| together with the functions |Int|, |Plus|, and |Squared|, is an initial object in this category.
\end{exer}

\subsection{Datatypes as Initial Algebras}
\label{sec:datatypes-as-initial}



\begin{reading*}
  This chapter is strongly inspired by Varmo Vene's Ph.D.\ thesis \cite[Chapter 2]{vene_phd}.

  A good explanation of recursion on lists is given in Graham Hutton's tutorial paper \cite{DBLP:journals/jfp/Hutton99}.
\end{reading*}

In this section we introduce (initial) algebras which allows us to define inductive data types.

\begin{dfn} Let $F:\CC\to \CC$ be an endofunctor. An \textbf{$F$-algebra} consists of the following data:
\begin{enumerate}
\item An object $X\in \Ob \CC$.
\item A morphism $\phi \in \CHom{\CC}{F(X)}{X}$.
\end{enumerate}
\end{dfn}

\begin{intu}
An algebra is roughly a set equipped with some operations, such as multiplication.
The \emph{arities}, that is, the inputs, of the operations are determined by the functor $F$.
An important class of functors are \emph{polynomial functors} built using the coproduct $(+)$ and the product $(\times)$.
Intuitively, the different summands of a polynomial functor each correspond to one datatype constructor, whereas the use of the product indicates that a constructor takes several inputs.
\end{intu}

\begin{exa}\label{exa:nno_initial_alg_maybe}
 Let $\Maybe : \SET \to \SET$ be the endofunctor given on objects by $\Maybe(X) := 1 + X$. 
 A $\Maybe$-algebra is a pair $(X,\phi)$ of a set $X$ and a function $\phi : 1 + X \to X$.
 By the universal property of the coproduct, $\phi$ is given, equivalently,
 by two functions $z$ and $s$ as follows.
 \begin{align*}
    z : 1 &\to X
    \\
    s : X &\to X
 \end{align*}
Here, think of ``$z$'' standing for ``zero'', and ``$s$'' standing for ``successor''.
\end{exa}


\begin{exa}\label{example:algebra_of_monoids} Let $F$ be the endofunctor induced by:
\begin{align*}
  F: \SET &\to \SET
  \\
  X &\mapsto 1 + (X\times X).
\end{align*}
An $F$-algebra consists of a set $X\in\SET$ together with a function $\phi:\CHom{\SET}{1 + (X\times X)}{X}$. Since $\phi$ is function from the disjoint union, we have that $\phi$ corresponds uniquely to two functions:
\begin{align*}
e &: 1\to X
\\
m &: X\times X\to X.
\end{align*}
This is precisely the data of a monoid.

Conversely, if $(M,m,e)$ is a monoid, then this induces a function
\[
1 + (M\times M) \xrightarrow{\phi = [\phi_e,\phi_m]} M,
\]
defined by pattern matching as follows:
\begin{align*}
\phi_e : 1 &\to M
\\
\star &\mapsto e
\\
\phi_m : M\times M &\to M 
\\
(x,y) &\mapsto m(x,y)
\end{align*}

The pair $(M, \phi)$ is an $F$-algebra.

\end{exa}

\begin{rem}
 Note that a monoid $(M,m,e)$ also satisfies some laws.
 The laws are not expressed in \cref{example:algebra_of_monoids}.
 To incorporate the laws, one studies instead algebras of a monad.
\end{rem}


\begin{dfn}\label{dfn:alg-hom}
Let $F:\CC\to \CC$ be an endofunctor and $(X,\phi)$ and $(Y,\psi)$ be $F$-algebras. 
A \textbf{($F$-algebra) homomorphism} from $(X,\phi)$ to $(Y,\psi)$ is a morphism $f\in \CHom{\CC}{X}{Y}$ such that the following diagram commutes:
\[
\begin{tikzcd}
FX
\arrow[r, "\phi"] 
\arrow[d,swap, "Ff"]
& X
\arrow[d, "f"] 
\\
FY
\arrow[r, swap, "\psi"] 
& Y
\end{tikzcd}
\]
\end{dfn}



\begin{exer} 
Let $F$ be the endofunctor defined as in \cref{example:algebra_of_monoids}, 
i.e. the endofunctor whose algebras correspond with monoids. 
Characterize/describe the $F$-algebra homomorphisms.
\end{exer}

\begin{dfn}\label{definition:category_of_Falgebras} Let $F:\CC\to \CC$ be an endofunctor. The \textbf{category of $F$-algebras}, denoted by $\ALG{F}$, is defined by the following data:
\begin{itemize}
\item The objects are the $F$-algebras.
\item The morphisms are the $F$-algebra homomorphisms.
\item The identity on $(X,\phi)$ is given by the identity $\Id[X]$ in $\CC$.
\item The composition is given by the composition of morphisms in $\CC$.
\end{itemize}
\end{dfn}

\begin{exer}
  \begin{enumerate}
  \item[]
  \item Draw two diagrams to illustrate \cref{definition:category_of_Falgebras}.
  \item Show that $\ALG{F}$ satisfies the properties of a category.
  \end{enumerate}
\end{exer}

We are interested in \textbf{initial objects of $\ALG{F}$}, if they exist.
We call these ``initial $F$-algebras''.
For a general endofunctor $F$, an initial $F$-algebra does not exist;
but for many interesting choices of $F$, such an initial object does exist.
Before coming to the general definition (see \cref{dfn:initial-alg}),
we consider an example.


\begin{exer}
  Consider the functor $\Maybe : \SET \to \SET$.
  \begin{enumerate}
  \item Show that the initial $\Maybe$-algebra is given by the pair $(\NN, [\Zero,\Succ])$, 
    where $\NN$ is the set of natural numbers, and $\Zero : 1 \to \NN$ and $\Succ : \NN\to\NN$ 
    are the function picking out zero and the successor function, respectively.
  \item Given any other $\Maybe$-algebra $(X,[z,s])$, unpack what it means for the square of \cref{dfn:alg-hom} to commute.
  \item Compare the data from the previous exercise to a definition of a function $f : \NN \to X$ by pattern matching (e.g., in Haskell).
  \end{enumerate}
\end{exer}

\begin{dfn}\label{dfn:initial-alg}
  Let $F:\CC\to\CC$ be an endofunctor. An \textbf{initial $F$-algebra} (if it exists) is an initial object in $\ALG{F}$.
  Unfolding the definition, this means that it is an $F$-algebra $(\Initalg{F}, \In)$ such that for any $F$-algebra $(X,\phi)$, there exists a unique morphism $\catam{\phi} \in \CHom{\CC}{\Initalg{F}}{X}$ such that the following diagram commutes:
\begin{equation}\label{eq:initial-f-alg}
\begin{tikzcd}
{F\Initalg{F}} 
\arrow[r, "\In"] 
\arrow[d,swap, "F\catam{\phi}"]
& {\Initalg{F}} 
\arrow[d, "\catam{\phi}"] 
\\
FX
\arrow[r, swap, "\phi"] 
& X
\end{tikzcd}
\end{equation}
A morphism of the form $\catam{\phi}$ is called a \textit{catamorphism}.
\end{dfn}

\begin{exer}[\cref{sol:in_catamorphism_id}]\label{exer:in_catamorphism_id}
  Let $F:\CC\to\CC$ be an endofunctor and let $(\Initalg F, \In)$ be an initial algebra. Show that
  \[\catam{\In} = \Id[\Initalg{F}].\]
\end{exer}

\begin{exer}\label{exer:bool_as_initial_algebra}
 Let $\mathbf{Bool}$ be the inductive data type generated by the following two constructors:
 \begin{lstlisting}
    True  : Bool
    False : Bool
  \end{lstlisting}
 Define an endofunctor $F:\SET\to \SET$ such that the $F$-algebras can be characterized as triples $(X, b_1, b_2)$ with $X$ a set and $b_1,b_2\in X$.
 
Moreover, show that $(\mathsf{Bool}, \mathsf{True}, \mathsf{False})$ is an initial object in $\ALG{F}$.
\end{exer}

\begin{exer}\label{exer:coproduct_as_initial_algebra}
The disjoint union (i.e., the coproduct) of two sets $X$ and $Y$ can also be characterized as an inductive data type; indeed, it is generated by the following two constructors:
 \begin{lstlisting}
    f : X -> X+Y
    g : Y -> X+Y
  \end{lstlisting}
Define an endofunctor $F:\SET\to \SET$ such that the $F$-algebras can be characterized as triples $(C, i_l, i_r)$ with $C$ a set and $i_l : X\to C, i_r : Y\to C$ be functions.
 
Moreover, show that $(X + Y, \inl, \inr)$ is the initial object in $\ALG{F}$, where $X+Y$ is the disjoint union of $X$ and $Y$ (i.e. the coproduct in $\SET$) and $\inl:X\to X+Y, \inr:Y\to X+Y$ the canonical inclusions.
\end{exer}

\begin{exer}\label{exer:conatural_numbers_are_not_initial}
Let $\mathbb{N}^{c}$ be the conatural numbers, i.e. $\mathbb{N} + \{\infty\}$. Consider the endofunctor $\Maybe : \SET \to \SET$ (defined in \cref{exa:nno_initial_alg_maybe}), i.e. the functor given on objects by
\begin{align*}
  \Ob{\Maybe} : \Ob{\SET} &\to \Ob{\SET}
  \\
  X &\mapsto 1 + X.
\end{align*}
The functions
\begin{align*}
zero^{c} : \mathbf{1}&\to \mathbb{N}^{c} \\ 
           \star&\mapsto 0,\\
succ^{c} : \mathbb{N}^{c}&\to \mathbb{N}^{c} \\
          x & \mapsto 
\begin{cases}
n+1,\quad  \text{ if } x := n\in\mathbb{N},\\
\infty,\quad  \text{ if } x := \infty.\\
\end{cases}
\end{align*}
form a $\Maybe$-algebra. However, show that $(\mathbb{N}^{c},zero^{c},succ^{c})$ is not an initial $\Maybe$-algebra.
\end{exer}

\begin{exer}[\textbf{Fusion property}, \cref{sol:fusion-property}]\label{exer:fusion-property}
  Let $F:\CC\to\CC$ be an endofunctor and let $(\Initalg F, \In)$ be an initial algebra. Show that
  for $F$-algebras $(X,\phi)$ and $(Y,\psi)$ and $f\in\CHom{\CC}{X}{Y}$, we have 
\[
\co{\phi}{f} = \co{F(f)}{\psi} \implies \co{\catam{\phi}}{f} = \catam{\psi}.
\]
This is summarized in the following diagram:
\begin{equation}\label{eq:initial-f-alg-composition}
\begin{tikzcd}
F\Initalg{F} 
\arrow[r, "\In"] 
\arrow[d,swap, "F\catam{\phi}"]
\ar[dd, bend right=60, "F\catam{\psi}"']
&
\Initalg{F}
\arrow[d, "\catam{\phi}"]
\ar[dd, bend left=60, "\catam{\psi}"]
\\
FX
\arrow[r, swap, "\phi"]
\ar[d, "Ff"'] 
&
X \ar[d, "f"]
\\
FY \ar[r, "\psi"']
&
Y
\end{tikzcd}
\end{equation}
\end{exer}

\begin{thm}[\textbf{Lambek's theorem}]
  Let $F:\CC\to\CC$ be an endofunctor and let $(\mu^F, \In)$ be an initial algebra. Then, $\In$ is an isomorphism whose inverse is given by $\Inv{\In} = \catam{F(\In)}$.
\end{thm}

\begin{exer} 
  Prove Lambek's theorem. 
\end{exer}

\begin{exer}[\cref{sol:initialalg_for_idfun_with_initialob}]\label{exer:initialalg_for_idfun_with_initialob} Let $\CC$ be a category with an initial object $\bot$. Show that $(\bot, \Id[\bot])$ is the initial algebra for the identity (endo)functor on $\CC$.
\end{exer}

\begin{exer}\label{exer:initialalg_for_list}
  Let $A$ be a set and define $F$ to be the functor induced by 
\begin{align*}
  F_A:\SET &\to\SET
\\
  X &\mapsto 1 + (A\times X).
\end{align*}

\begin{enumerate}
\item \label{enum:list-alg} Show that an $F_A$-algebra consists of a triple $(X,n,c)$, where $X$ is a set, $n\in X$ is an element of $X$, and $c : A \times X \to X$ is a function.
\item Show that the initial $F_A$-algebra is given by the set $\List(A)$ of $A$-valued lists, together with constants $\nil \in \List(A)$ and $\cons : A \times \List(A) \to \List(A)$.
\item Given any other $F_A$-algebra $(X,n,c)$, unpack what it means for the square of \cref{dfn:alg-hom} to commute.
Compare it to a definition of a function $f : \List(A) \to X$ by pattern matching.
\end{enumerate}

\end{exer}

\begin{rem}
  In the case of lists, the operator $\catam{\_}$ is also known as |fold|, which in Haskell is defined as follows:
  \begin{lstlisting}
    fold            :: (a → b → b) → b → ([a] → b)
    fold f v   []   =  v
    fold f v (x:xs) =  f x (fold f v xs)
  \end{lstlisting}
  Compare the input of |fold| to the data of an $F_A$-algebra given in \cref{enum:list-alg} of \cref{exer:initialalg_for_list}.
\end{rem}


\begin{exer}\label{exer:list-functions-as-fold}
  Define the following functions as a catamorphism, that is, using |fold|.
  In each case, draw the diagram corresponding to Diagram~\ref{eq:initial-f-alg} of \cref{dfn:initial-alg}.
  \begin{enumerate}
  \item |sum :: [Int] →  Int|
  \item |product :: [Int] →  Int|
  \item |and :: [Bool ] →  Bool|
  \item |or :: [Bool ] →  Bool|
  \item |(++) :: [a] →  [a] →  [a]|
  \item |length :: [a] →  Int|
  \item |reverse :: [a] →  [a]|
  \item |map :: (a →  b) →  ([a] →  [b])|
  \item |filter :: (a →  Bool ) →  ([a] →  [a])|
  \end{enumerate}
  Solutions are given in \cite[\S2]{DBLP:journals/jfp/Hutton99}.

  Hint: a systematic approach to reformulating functions on lists defined by explicit recursion in terms of |fold| is described in \cite[\S3.3]{DBLP:journals/jfp/Hutton99}.
\end{exer}

\begin{exer}[\cref{sol:list-concat-nil}, see also \cite{DBLP:journals/scp/Malcolm90}]
  \label{exer:list-concat-nil}
  Consider the function |(++) :: [a] →  [a] →  [a]| defined in \cref{exer:list-functions-as-fold},
  and |l :: [a]|.
  \begin{enumerate}
  \item Show that |nil ++ l = l|.
    
  \item Show that |l ++ nil = l|.
  \end{enumerate}
\end{exer}

\begin{exer}[{\cite[\S3.1]{DBLP:journals/jfp/Hutton99}}]
  Show that |(+1) . sum = fold (+) 1|, by showing that |(+1) . sum| makes Diagram~\ref{eq:initial-f-alg} of \cref{dfn:initial-alg} commute.
\end{exer}

% \begin{exer}
%   An exercise about (the limits of) representing functions as catamorphisms (or rewriting functions using `fold`):
%   \begin{enumerate}
%     \item We can represent numbers as big-endian binary numbers, in the set $ \List(\{0, 1\}) $: the lists with elements in the set $ \{0, 1 \} $. For example, $ 13 $ becomes $ [1, 1, 0, 1] $, which we represent as |(cons 1 (cons 1 (cons 0 (cons 1 nil))))|. We define the function $ \texttt{bin2int}: \List(\{0, 1\}) \to \NN $, that converts big-endian binary representations to positive integers. For example, |(bin2int (cons 1 (cons 0 nil))) = 2| and |(bin2int (cons 1 (cons 1 (cons 0 (cons 1 nil))))) = 13|.
%    
%       Show that we can give |bin2int| as a catamorphism. i.e. with $ F $ the functor for which $ \List(\{0, 1\}) $ is an initial algebra, show that there exists an $ F $-algebra $ (\NN, f) $ such that $ \catam f = \texttt{bin2int} $.
%
%     \item We can also represent a number as a little-endian binary number. Then 13 becomes $ [1, 0, 1, 1] $, which we represent as |(cons 1 (cons 0 (cons 1 (cons 1 nil))))|. We define the function $ \texttt{bin2int2}: \List(\{0, 1\}) \to \NN $, that converts little-endian binary representations back to positive integers.
%    
%       Why can't we give |bin2int2| as a catamorphism?
%
%       \textit{(Hint: how would such a catamorphism calculate the values for $ [1] $, $ [0, 1] $ and $ [0, 0, 1] $?)}
%
%     \item Show that there exists an $ F $-algebra $ (\NN \times \NN, f) $ such that $ \projl \circ \catam{f} = \texttt{bin2int2} $, with $ \projl $ the projection on the first coordinate.
%   \end{enumerate}
% \end{exer}

\begin{exer}\label{exer:initialalg_for_btree} Let $A$ be a set and define $F$ to be the functor induced by 
\[
F_A:\SET\to\SET : X\mapsto A + (X\times X).
\]
Show that the initial $F_A$-algebra is given by the set $\BinTree(A)$ of $A$-valued binary trees.
\end{exer}

\begin{rem} Notice that in \cref{exer:initialalg_for_list} and \cref{exer:initialalg_for_btree}, we can consider the functor $F_A$ as a bifunctor where we vary $A$, i.e.
\[
F: \SET\to \SET\to \SET: (A,X)\mapsto F_A(X).
\]
In particular, under the assumption that for every $A\in\SET$ the initial $F_A$-algebra exists, we can wonder if the assignment 
\[
\Ob{\SET} \to \Ob{\SET} : A\mapsto \mu F_A ,
\]
can be extended to a functor. The following exercise answers this question positively for arbitrary categories.
\end{rem}

\begin{exer}[\cref{sol:initialalg_for_bifunctor_functor}]\label{exer:initialalg_for_bifunctor_functor} Let $F:\CC\to\CC\to\CC$ be a bifunctor such that for any object $A\in\CC$, the initial algebra for the functor induced by 
\[
F_A : \CC\to\CC : X\mapsto F(A,X),
\]
exists. Show how
\[
\Ob{\CC} \to \Ob{\CC} : A\mapsto \mu F_A ,
\]
induces a functor.
\end{exer}



\section{Fusion Property}\label{sec:fusion}

The fusion property of \cref{exer:fusion-property} can be used to ``fuse'' a composition of functions into one function, possibly leading to more efficient code.

We are going to exemplify this using the datatypes of lists and of natural numbers.
Recall that $(\NN, [\Zero,\Succ])$ is the initial $\Maybe$-algebra.
We also write $(+1)$ for $\Succ$.

\begin{reading*}
  The content of this section is very much inspired by \cite[\S3.2]{DBLP:journals/jfp/Hutton99}.
  You are strongly encouraged to read that section before reading the present section.

  The fusion property is called ``promotion theorem'' in Malcolm's work~\cite{DBLP:journals/scp/Malcolm90}.
\end{reading*}

\begin{exer}[See also \cite{DBLP:journals/scp/Malcolm90}]
  Consider the function |(++) :: [a] →  [a] →  [a]| defined in \cref{exer:list-functions-as-fold}.
  Show that |(l ++ m) ++ n = l ++ (m ++ n)| for any |l, m, n :: [a]|.
\end{exer}


\begin{exer}[{\cite[\S3.2]{DBLP:journals/jfp/Hutton99}}]
  Show that |(+1) . sum = fold (+) 1|, by using the fusion property of \cref{exer:fusion-property}.
\end{exer}

\begin{solution}
  We also write $\sum$ for |sum|. Note that |sum| is defined as the catamorphism $\catam{[\Zero, (+)]}$.
  The situation is summarized in the following diagram
  
  \[
    \begin{tikzcd}[row sep = huge, column sep = huge]
      \{*\} + \NN \times \List(\NN)
      \ar[d, "\Id + \Id \times \sum" description]
      \ar[r, "{[\nil, \cons]}"]
      &
      \List(\NN)
      \ar[d, "\sum" description]
      \ar[dd, bend left, "\catam{[1, (+)]}"]
      \\
      \{*\} + \NN \times \NN
      \ar[r, "{[\Zero, (+)]}"]
      \ar[d, "{\Id + \Id \times (+1)}" description]
      &
      \NN
      \ar[d, "{(+1)}" description]
      \\
      \{*\} + \NN \times \NN
      \ar[r, "{[1, (+)]}"]
      &
      \NN 
    \end{tikzcd}
  \]

  In order to show that |(+1) . sum = fold (+) 1|, in the diagram expressed as $\co{\sum}{(+1)} = \catam{[1, (+)]}$, we need to show that the lower rectangle commutes---this is what  \cref{exer:fusion-property} says.

  There are two cases to consider: the case of $*$, and the case of a pair $(m,n) \in \NN\times\NN$.
  In the case of $*$, we obtain
  \[
    (+1)~0 = 1
  \]
  and in the case of $(m,n)$, we obtain
  \[
    (+1)~ ((+)~ m~ n) = (+)~ m~ ((+1)~n)
  \]
  both of which hold.
  
\end{solution}

\begin{exer}
  Prove that |map g . map f = map (g . f)| using the fusion property.
\end{exer}


\begin{reading*}
  In \cref{sec:initial-algs,sec:fusion}, we have only looked at functions defined via \emph{iteration}, that is, functions defined as \emph{catamorphisms} of some $F$-algebra.
  While \emph{primitive recursive} functions can always be expressed as a catamorphism via tupling (see, e.g., \cite[\S4]{DBLP:journals/jfp/Hutton99} and \cite[\S3.1]{vene_phd}), it is more natural specify them via a slightly more sophisticated universal property, explained in detail in Vene's dissertation~\cite[Chapter~3]{vene_phd}.

  The interested reader might also study the tutorial by Fokkinga~\cite{Fokkinga_homo-cata} or the paper by Meijer et al.~\cite{DBLP:conf/fpca/MeijerFP91}.
  The paper \cite{DBLP:journals/scp/Malcolm90} by Malcolm contains more examples.
  
  A guide to further literature on recursion operators is given in \cite[\S6]{DBLP:journals/jfp/Hutton99}.
\end{reading*}

\section{Terminal Coalgebras and Coinductive Datatypes}


\begin{reading*}
  We only give a brief introduction to (terminal) coalgebras in this section.
  A more systematic exploration of the topic is given in~\cite{DBLP:conf/fpca/MeijerFP91}.

  The paper \cite{DBLP:journals/scp/Malcolm90} by Malcolm contains further examples.
\end{reading*}


In this section we introduce (terminal) coalgebras which allows us to define coinductive data types.

\begin{dfn} Let $F:\CC\to \CC$ be an endofunctor. An \textbf{$F$-coalgebra} consists of the following data:
\begin{itemize}
\item An object $X\in \CC$.
\item A morphism $\phi \in \CHom{\CC}{X}{F(X)}$.
\end{itemize}
\end{dfn}
Notice that an $F$-algebra consists of a morphism $F(X)\to X$, while an $F$-coalgebra consists of a morphism $X\to F(X)$ in the other direction.

\begin{dfn} Let $F:\CC\to \CC$ be an endofunctor and $(X,\phi)$ and $(Y,\psi)$ be $F$-coalgebras. A \textbf{($F$-coalgebra) homomorphism} from $(X,\phi)$ to $(Y,\psi)$ is a morphism $f\in \CHom{\CC}{X}{Y}$ such that the following diagram commutes:
\[
\begin{tikzcd}
X
\arrow[r, "\phi"] 
\arrow[d,swap, "f"]
& FX
\arrow[d, "F(f)"] 
\\
Y
\arrow[r, swap, "\psi"] 
& FY
\end{tikzcd}
\]
\end{dfn}

\begin{exer} Define the category $\COALG{F}$ of $F$-coalgebras analogously to the category $\ALG{F}$ of $F$-algebras (as in \cref{definition:category_of_Falgebras}).
\end{exer}

\begin{dfn} Let $F$ be an endofunctor on $\CC$. The \textbf{terminal $F$-coalgebra} is the terminal object in $\COALG{F}$ which we denote by $(\Terminalcoalg{F}, \Out)$.
  
For $(X,\phi)$ an arbitrary $F$-coalgebra, we denote the unique morphism $(X,\phi) \to (\Terminalcoalg{F}, \Out)$ by $\anam{\phi}$, and we call a morphism of this form an \textit{anamorphism} (instead of a catamorphism as in \cref{dfn:initial-alg}).
\end{dfn}

\begin{exer} Spell out what it means for a coalgebra to be the terminal coalgebra.
\end{exer}

\begin{exer} Let $F:\CC\to\CC$ be an endofunctor and assume that the terminal coalgebra $(\Terminalcoalg{F}, \Out)$ exists. Show that the following properties holds:
\begin{enumerate}
\item $\Id = \anam{\Out}$.
\item For $F$-algebras $(X,\phi)$ and $(Y,\psi)$ and $f\in\CHom{\CC}{X}{Y}$, we have 
\[
  \co f \psi   = \co \phi {F(f)} \implies \co{f}{\anam{\psi}} = \anam{\phi}.
\]
This is summarized in the following diagram:
\[
\begin{tikzcd}
X
\arrow[r, "\phi"] 
\arrow[d,swap, "f"]
\ar[dd, "\anam{\phi}"', bend right = 60]
&
FX
\arrow[d, "Ff"]
\ar[dd, "F\anam{\phi}", bend left = 60]
\\
Y
\arrow[r, swap, "\psi"]
\ar[d, "\anam{\psi}"']
&
FY
\ar[d, "F\anam{\psi}"]
\\
\Terminalcoalg{F}
\ar[r, "\Out"]
&
F\Terminalcoalg{F}
\end{tikzcd}
\]
\end{enumerate} 
\end{exer}

\begin{thm}[\textbf{Dual of Lambek's theorem}] Let $F:\CC\to\CC$ be an endofunctor and let $(\Terminalcoalg{F}, \Out)$ be a terminal coalgebra. Then is $\Out$ an isomorphism whose inverse is given by $\Inv{\Out} = \anam{F(\Out)}$.
\end{thm}

\begin{exer}
  Prove the dual of Lambek's theorem. 
\end{exer}

\begin{exer}\label{exer:terminalalg_for_idfun_with_terminalob} Let $\CC$ be a category with a terminal object $\top$. Show that $(\top, \Id[\top])$ is a terminal coalgebra for the identity (endo)functor on $\CC$.
\end{exer}

\begin{exer}[\cref{sol:conatural_numbers_terminal_coalgebra}] \label{exer:conatural_numbers_terminal_coalgebra} Let $F$ be the functor induced by 
\[
F:\SET\to\SET : X\mapsto 1 + X.
\]
Show that the terminal $F$-coalgebra is given by the following data:
\begin{itemize}
\item The underlying object is given by the set $\mathbb{N} + \{\infty\}$ of natural numbers with infinity.
\item The underlying function is given by the predecessor defined as follows: 
\begin{align*}
  \mathbb{N} + \{\infty\} &\to 1 + \mathbb{N} + \{\infty\}
  \\
  0 &\mapsto \star
  \\
  s(n) &\mapsto n
  \\
  \infty &\mapsto \infty
\end{align*}
where $\star$ is the unique element of $1$.
\end{itemize}
\end{exer}


\begin{exa}[Streams]
  The codata type of streams over a given set $A$ is given by the terminal coalgebra $(\Terminalcoalg{F_A}, \Out)$ of the functor $F_A (X) := A \times X$.
  We write $\Stream(A)$ for $\Terminalcoalg{F_A}$.
  The functions $\head : \Stream(A) \to A$ and $\tail : \Stream(A) \to \Stream(A)$
  are given by
  \begin{align*}
    \head &= \co{\Out}{\projl} : \Stream(A) \to A
    \\
    \tail &= \co{\Out}{\projr} : \Stream(A) \to \Stream(A)
  \end{align*}
  respectively. Put differently (recall the definition of the product in \cref{def:binproduct}), we have
  \[
    \Out = \langle \head, \tail \rangle : \Stream(A) \to A \times \Stream(A)
  \] 
  Given any two functions
  \begin{align*}
    h : C \to  A
    \\
    t : C \to C
  \end{align*}
  the anamorphism $\anam{\langle h, t \rangle }$ is the unique solution
  $f : C \to  \Stream(A)$
  of the equation system
  \begin{align*}
    \co{f} \head &= h
    \\
    \co f \tail &=  \co t f
  \end{align*}
  that is, the unique function $f : C \to \Stream(A)$ making the following square commute:
  \[
    \begin{tikzcd}[column sep = large]
      C
      \ar[r, "{\langle h, t \rangle}"]
      \ar[d, "\anam{\langle h, t \rangle}"']
      &
      A \times C
      \ar[d, "\Id \times \anam{\langle h, t \rangle}"]
      \\
      \Stream(A)
      \ar[r, "{\langle \head, \tail \rangle}"]
      &
      A \times \Stream(A).
    \end{tikzcd}
  \]
  
\end{exa}

\begin{exer}[\cref{sol:stream-of-nats}]\label{exer:stream-of-nats}
  Define, as an anamorphism, the function $\nats : \NN \to \Stream(\NN)$ which returns the stream of all natural
  numbers starting with the natural number given as the argument
\end{exer}


\begin{exer}[\cref{sol:zip}]\label{exer:zip}
  Define, as an anamorphism, the function $\zip : \Stream(A) \times \Stream(B) \to \Stream(A \times B)$ that zips the argument streams together.
\end{exer}


\begin{exer}
  For a fixed set $A$, consider the functor given on objects by $F(X) := 1 + A \times X$.
  Its terminal $F$-coalgebra is the datatype $\Colist(A)$ of potentially infinite lists over elements in $A$, with suitable destructors $\head$ and $\tail$ and an ``exception'' in case the colist is empty.

  Define a function $\Colist(A) \to  \mathbb{N} + \{\infty\}$ counting the elements in a colist.
  
\end{exer}


%%% Local Variables:
%%% mode: latex
%%% TeX-master: "CT4P"
%%% End:
}

\section{Solutions}
\label{sec:solutions}

\begin{solution}[\cref{exer:post_antisymmetry}]\label{sol:post_antisymmetry}
The inequality $x\leq y$ (resp. $y\leq x$) means that we have a (unique) morphism from $f\in \Hom{x}{y}$ (resp. $g\in \Hom{y}{x}$). Consequently, we get a \textit{loop} $\co{f}{g} \in \Hom{x}{x}$. Since the hom-sets are either empty or a singleton, we have $\co{f}{g} = \Id[x]$. Hence, antisymmetry means that if $\Id[X] = \co{f}{g}$ for some $f\in \Hom{x}{y}$ and $g\in \Hom{y}{x}$, we must have 
\[x=y,\quad f = \Id[x] = g.\]
Rephrased a little bit different, we get: The category $\PRE(X,\leq)$ has no non-trivial loops if $(X,\leq)$ is antisymmetric.
\end{solution}

\begin{solution}[\cref{exer:POS_isnt_a_posetcat}]\label{sol:POS_isnt_a_posetcat}
That $\POS$ would be a preorder-category means that each of the hom-sets is either empty or a singleton. Hence, it is not a preorder-category if there exists $(X,\leq_X)$ and $(Y,\leq_Y)$ such that $\CHom{\POS}{(X,\leq_X)}{(Y,\leq_Y)}$ has more then one element.\\
We choose $(X,\leq) := (\mathbb{N},\leq) =: (Y,\leq)$. A morphism $f\in \CHom{\POS}{(\mathbb{N},\leq)}{(\mathbb{N},\leq)}$ consists of a function $f:\mathbb{N}\to\mathbb{N}$ such that the following property holds:
\[
\forall n,m\in\mathbb{N}: n\leq m \implies f(n)\leq f(m).
\]
But there are a lot of functions from $\mathbb{N}$ to $\mathbb{N}$ which satisfies this property, indeed: For any $k\in\mathbb{N}$, we have that
\[
f_k : \mathbb{N}\to\mathbb{N}: n\mapsto n+k,
\]
is a morphism in $\POS$. Hence $\CHom{\POS}{(\mathbb{N},\leq)}{(\mathbb{N},\leq)}$ consists of an infinite amount of distinct morphisms.
\end{solution} 

\begin{solution}[\cref{exer:categories_coming_from_monoids}]\label{sol:categories_coming_from_monoids}
A category $\CC$ is of the form $(M,m,e)$ if and only if $\CC$ has a unique object. Indeed, if $\CC$ has a unique object, lets denote this by $X$, then we can define a monoid $(M,m,e)$ as follows:
\begin{itemize}
\item The underlying set of the monoid is $M := \CHom{\CC}{X}{X}$.
\item The multiplication $m$ is given by $m(f,g) := \co{f}{g}$.
\item The identity element $e$ is given by $e := \Id[X]$.
\end{itemize}
That $(M,m,e)$ is indeed a monoid, i.e. satisfies the monoid laws, is just a translation of the axioms of $\CC$ being a category.
\end{solution}

\begin{solution}[\cref{exer:category_of_monoids}]\label{sol:category_of_monoids}
Since a monoid consists of a set $M$ together with a binary operation $m:M\to M\to M$ (called the \textit{multiplication}) and a \textit{identity} element $e\in M$, a suitable \textit{morphism of monoids}, from $(M_1,m_1,e_1)$ to $(M_2,m_2,e_2)$ should consists of a function $f:M\to N$ which preserves the structure (i.e. the multiplication and the identity element). More precisely:
\begin{itemize}
\item Preservation of the multiplication:
\[
\forall x,y\in M_1: f(m_1(x,y)) = m_2(f(x), f(y)).
\]
\item Preservation of the identity:
\[
f(e_1) = e_2.
\]
\end{itemize}
We will now prove that the monoids (as objects) and morphisms of monoids (as the morphisms) carry the data of a category, i.e. we have to define identity morphisms and the composition of morphisms:
\begin{itemize}
\item Let $(M,m,e)$ be a monoid. The identity morphism is given by the identity function $\Id[M]$ on the underlying set $M$. 
\item Let $(M_i,m_i,e_i)$ be a monoid for $i=1,2,3$ and let 
\[ 
f:(M_1,m_1,e_1)\to (M_2,m_2,e_2),\quad g:(M_2,m_2,e_2)\to (M_3,m_3,e_3)
\] 
be morphisms of monoids. The composition $\co{f}{g}$ of $f$ and $g$ is defined as the composition of the underlying functions.
\end{itemize}
Before we show that this data satisfies the properties of a category, we first have to show that everything is well-defined, i.e. that the identity is a morphism of monoids and that the composition of morphisms of monoids is again a morphism of monoids:
\begin{itemize}
\item That the identity is a morphism of monoids follows by the following calculations:
\begin{eqnarray*}
\forall x,y\in M &:& \Id[M](m(x,y)) = m(x,y) = m(\Id[M](x),\Id[M](y)),\\
&& \Id[M](e) = e.
\end{eqnarray*}
The equalities holds because $\Id[M]$ is the identity function on $M$.
\item That the composition of morphism of monoids is again a morphism of monoids follows by the following calculations:
\begin{itemize}
\item Composition preserves multiplication: for any $x,y \in M_1$, we have
\begin{eqnarray*}
 \co{f}{g}(m_1(x,y)) &=& g\left(f(m_1(x,y))\right)\\ 
	&=& g\left(m_2(f(x),f(y)))\right) \quad \text{ ($f$ preserves mult.)} \\
	&=& m_3(g(f(x)),g(f(y))) \quad \text{ ($g$ preserves mult.)}\\
	&=& m_3(\co{f}{g}(x),\co{f}{g}(y))
\end{eqnarray*}
Composition preserves identity element by:
\[
\co{f}{g}(e_1) = g(f(e_1)) = g(e_2) = e_3,
\]
where the second (resp. third) equality holds since $f$ (resp. $g$) preserves the identity element.
\end{itemize}
\end{itemize}
So everything is indeed well-defined. So we are now ready to show that composition of some morphism of monoids $f$ with the identity morphism is again $f$ (both on the left and right) and that the composition of morphisms of monoids is associative. This follows immediate since everything is defined using functions and we know that functions satisfy these properties.
\end{solution}

\begin{solution}[\cref{exer:category_with_naturalnumbers_and_matrices}]
  \label{sol:category_with_naturalnumbers_and_matrices}
  In order to make $\mathcal{C}$ into a category, we have to define the identity morphisms and the composition of morphisms.
  
  Let $n\in \mathbb{N} =: \Ob{\CC}$ be a natural number. The identity on $n$ is an element in $\CHom{\CC}{n}{n}$, which is the set of $n\times n$-matrices. Hence, for the identity on $n$, we can take $\Id[n]$ to be the identity $n\times n$-matrix, i.e. all coefficients are zero except on the diagonal where everything is $1$.
  
Let $l,m,n \in\mathbb{N}$ be natural numbers and $M$ (resp. $N$) be a $l\times m$-matrix (resp. $m\times n$-matrix). The composition $\co{M}{N}$ should be an $l\times n$-matrix, hence we define $\co{M}{N}$ as the matrix multiplication of $M$ and $N$, i.e. $\co{M}{N} = M\cdot N$.\\
That this data satisfies the properties of being a category follows immediate because matrix multiplication is associative and that the multiplication of any matrix $M$ with the identity matrix (of the right size and both on the left or on the right) is again $M$.
\end{solution}

\begin{solution}[\cref{exer:opposite}]\label{sol:opposite}
  In order to avoid confusion, we use the following notation: For any $f\in\CHom{\CC}{X}{Y}$ morphism, we denote by $\op{f}$ the corresponding morphism in $\CHom{\op\CC}{Y}{X}$.
  
  Let $X\in \Ob{\op\CC} = \Ob{\CC}$. The identity morphism is defined as the morphism corresponding to the identity, i.e. it is $\op{(\Id[X])}$.
  
  Let $\op{g} \in \CHom{\op\CC}{Z}{Y}, \op{f}\in \CHom{\op\CC}{Y}{X}$. The composition is defined as: $\co{\op g}{\op f} := \op{(\co{f}{g})}$.
  
That this data satisfies the properties of a category, follows because $\CC$ is a category, indeed:
\begin{itemize}
\item That the left unit law holds follows by the right unit law of $\CC$ as follows:
\[
\co{\op{\Id}}{\op{f}} = \op{(\co{f}{\Id})} = \op{f},
\]
where the first equality holds by definition of the \textit{opposite composition} and the second holds by the right unit law of $\CC$.\\
The right unit law holds analoguously by the left unit law of $\CC$.
\item That the associativity holds follows by the associativity of $\CC$ as follows:
\begin{eqnarray*}
\co{\op{h}}{(\co{\op g}{\op f})} &=& \co{\op{h}}{\op{(\co{f}{g})}}\\
 	&=& \op{(\co{(\co{f}{g})}{h})}\\ 
 	&=& \op{(\co{f}{(\co{g}{h})})} \text{ by associativity of $\CC$},\\
 	&=& \co{\op{(\co{g}{h})}}{\op f}\\
 	&=& \co{(\co{\op h}{\op g})}{\op f}
\end{eqnarray*}
\end{itemize}
\end{solution}

\begin{solution}[\cref{exer:connection_graphs_preordersets}]\label{sol:connection_graphs_preordersets}
Any preordered set $(X,\leq)$ can be described by a graph where the vertices are given by the elements of $X$ and there exists an edge from $x$ to $y$ if and only if $x \leq y$. Hence, if we denote by $G$ the corresponding graph of $(X,\leq)$, we have that $\POS(X,\leq) = \mathbf{Graph}(G)$.\\
In particular we have that the number of edges is either $0$ or $1$. Hence, a category generated by a graph comes from a preordered set if and only if the number of morphisms in any fixed hom-set is either $0$ or $1$.

If $(X,\leq)$ is a poset, i.e. we have antisymmetry, then we have that the corresponding graph (and consequently the corresponding category) have no (non-trivial) loops.
\end{solution}


%% Section on isomorphisms
\begin{solution}[\cref{exer:inverse-iso}]\label{sol:inverse-iso}
That $f:a\to b$ is an isomorphism with inverse $g$ means precisely that $\co{f}{g} = \Id[a]$ and $\co{g}{f} = \Id[b]$. But stating that $g$ is an isomorphism with inverse $f$ means precisely those conditions. Hence, this hold by definition.
\end{solution}

\begin{solution}[\cref{exer:inverse_uniqueness}]\label{sol:inverse_uniqueness}
Let $f:a\to b$ be an isomorphism. That $f$ has a unique inverse means that if $g,h : b\to a$ are morphisms such that 
\[
\co{f}{g} = \Id[a], \co{g}{f} = \Id[b], \co{f}{h} = \Id[a], \co{h}{f} = \Id[b]
\]
then we must have $g = h$.\\
So assume $g$ and $h$ satisfy the condition of being an inverse of $f$. Then we have:
\begin{eqnarray*}
g =& \co{g}{\Id[b]} &,\text{ by left unit law},\\
	=& \co{g}{(\co{f}{h})} &, \text{ since $h$ is inverse of $f$},\\
	=& \co{(\co{g}{f})}{h} &, \text{ by associativity},\\
	=& \co{\Id[a]}{h} &, \text{ since $g$ is inverse of $f$}\\
	=& h &, \text{ by right unit law}
\end{eqnarray*}
\end{solution}

\begin{solution}[\cref{exer:compofiso}]\label{sol:compofiso}
Let $f: a\to b$ and $g:b\to c$ be isomorphisms. Denote their (unique) inverses by $f^{-1}$ and $g^{-1}$. We have to show that there exists a morphism $h : c\to a$ such that 
\[
\co{(\co{f}{g})}{h} = \Id[a], \quad \co{h}{(\co{f}{g})} = \Id[c].
\]
We define $h := \co{g^{-1}}{f^{-1}}$. The left equality then holds by the following computation:
\begin{eqnarray*}
\co{(\co{f}{g})}{h} =& \co{(\co{f}{g})}{(\co{g^{-1}}{f^{-1}})} &\\
	=& \co{f^{-1}}{\co{(\co{g}{g^{-1}})}{f}} &\text{ by associativity,}\\
	=& \co{f^{-1}}{\co{\Id[b]}{f}} &\text{ since $g^{-1}$ inverse of $g$,}\\
	=& \co{f^{-1}}{f} &\text{ by unit law,}\\
	=& \Id[a] &\text{ since $f^{-1}$ inverse of $f$.}
\end{eqnarray*}
The right equality holds analoguously.
\end{solution}


\begin{solution}[\cref{exer:iso-bool}]\label{sol:iso-bool}
The Haskell datatype |Bool| is given by:
\begin{lstlisting}
data Bool = True | False
\end{lstlisting}
In order to construct a (Haskell) function $f$ from |BW| to |Bool|, it suffices to define $f(Black)$ and $f(White)$.\\
The first isomorphism, denoted by $f_1$, is given by $f_1(Black)=True$ and $f_1(White) = False$. Its inverse (denoted by $g_1$) is given by $g_1(True) = Black$ and $g_1(False) = White$. To show that these are inverse, we have to show 
\[
g_1 (f_1 (White)) = White, \quad g_1 (f_1 (Black)) = Black.
\]
These equalities holds by definition of $f_1$ and $g_1$.\\
The second isomorphism, denoted by $f_2$, is given by $f_2(Black)=False$ and $f_2(White) = True$. Its inverse (denoted by $g_2$) is given by $g_2(False) = Black$ and $g_1(True) = White$. That $g_2$ is the inverse of $f_2$ is also immediate.
\end{solution}

\begin{solution}[\cref{exer:iso_in_sets}]\label{sol:iso_in_sets}
The isomorphisms in $\SET$ are precisely the bijective functions, indeed:
\begin{itemize}
\item Assume $f: X\to Y$ is a bijection, i.e.
\[
\forall y\in Y: \exists! x_{y}\in X: f(x)=y 
\]
We show that the inverse of $f$ is given by:
\[
g : Y\to X: y\mapsto x_y.
\]
So we have to show $\co{f}{g} = \Id[X]$ and $\co{g}{f} = \Id[Y]$. Let $x\in X$, since that $g(f(x))$ is the unique element $z\in X$ such that $f(z) = f(x)$ (and since $x$ satisfies this condition), we have $g(f(x)) = x$. Since this holds for all $x\in X$, we have $\co{f}{g} = \Id[X]$.\\
Let $y\in Y$ and let $x := x_y$  be the unique element in $X$ such that $f(x)=y$. So by definition of $g$, we have $g(y) = x$, hence 
$f(g(y)) = f(x) = y$.
\item Assume $f:X\to Y$ is an isomorphism with inverse $f^{-1}$. Let $y\in Y$, we have to show that there exists a unique $x\in X$ such that $f(x)=y$. Define $x := g(y)$. Since $\co{g}{f} = \Id[Y]$, we have $y = f(g(y)) = f(x)$, hence, this $x$ indeed satisfies the condition. To show that $x$ is unique, let $z\in X$ satisfy $f(z)=y$. That $z=x$ now follows from $\co{f}{g} = \Id[X]$, indeed: 
$z = g(f(z)) = g(y) = x$.
\end{itemize}
\end{solution}

\begin{solution}[\cref{exer:iso_in_pos}]\label{sol:iso_in_pos}
The isomorphisms in $\POS$ are precisely the bijections $f:(X,\leq_X) \to (Y,\leq_Y)$ such that 
\begin{equation}\label{eqn:order_iso}
x_1 \leq_X x_2 \iff f(x_1) \leq_Y f(x_2),
\end{equation}
Indeed:
\begin{itemize}
\item Assume $f$ is a bijection which satisfies \cref{eqn:order_iso}. Since it is a bijection, we know (by the solution to \cref{exer:iso_in_sets}), that there exists a function $g:(Y,\leq_Y)\to (X,\leq_X)$ such that $\co{f}{g}=\Id[(X,\leq_X)]$ and $\co{g}{f}=\Id[(Y,\leq_Y)]$. However, this does not conclude the proof of the first implication, because we do not know appriori, that $g$ is a morphism of posets. So we have to show
\[
\forall y_1,y_2\in Y: y_1\leq_Y y_2 \implies g(y_1)\leq_X g(y_2).
\]
Let $y_1,y_2\in Y$. Since $f$ is bijective, there exists $x_1,x_2 \in X$ such that $f(x_1)=y_1$ and $f(x_2)=y_2$. If $f(x_1) = y_1\leq_Y y_2 = f(x_2)$, then by \cref{eqn_order_iso}, we also have that $x_1 \leq x_2$. But by definition of $g$, we have $g(y_1)=x_1$ and $g(y_2)=x_2$, hence $g(y_1)\leq_X g(y_2)$ which shows that $g$ is an order-preserving morphism, i.e. $g \in \CHom{\POS}{(Y,\leq_Y)}{(X,\leq_X)}$.
\item Assume $f$ is an isomorphism in $\POS$ with inverse $g$. Since $f$ is a function which satisfies $\co{f}{g}=\Id[(X,\leq_X)]$ and $\co{g}{f}=\Id[(Y,\leq_Y)]$, we have (by the same argument as in the solution to \cref{exer:iso_in_sets}), that $f$ is a bijection. Hence, it remains to show that \cref{eqn:order_iso} holds. Let $x_1,x_2\in X$.
  
  If $x_1\leq_X x_2$, then we have $f(x_1)\leq_Y f(x_2)$ since $f \in\CHom{\POS}{(X,\leq_X)}{(Y,\leq_Y)}$.
  
Assume $f(x_1)\leq_Y f(x_2)$. Since $g \in\CHom{\POS}{(Y,\leq_Y)}{(X,\leq_X)}$, we have $g(f(x_1)) \leq_X g(f(x_2))$. But $\co{f}{g} = \Id[X]$, hence $x_1 \leq_X x_2$.
\end{itemize}
\end{solution}

\begin{solution}[\cref{exer:iso_in_posetcategory}]\label{sol:iso_in_posetcategory}
First, let $(X,\leq_Y)$ be a preorder. A morphism $f:x\to y$ is an isomorphism if and only if there exists a morphism $g: y\to x$ such that $\co{f}{g}=\Id[x]$ and $\co{g}{f}=\Id[y]$. But, in a preorder category, each hom-set has a unique element if it is non-empty. So, for any $g\in \Hom{y}{x}$ and $f\in \Hom{x}{y}$, we always have $\co{f}{g}=\Id[x]$ and $\co{g}{f}=\Id[y]$. Hence a morphism $f:x\to y$ in a preorder-category is an isomorphism if and only if there exists a morphism $g:y\to x$. The existence of a morphism $f:x\to y$ means precisely that $x\leq y$. Hence, isomorphisms in a preorder-category corresponds with a pair of elements $(x,y)$ in $X$ such that $x\leq y$ and $y\leq x$.\\
If $(X,\leq_X)$ is a poset, i.e. satisfies antisymmetry, then if $x\leq y$ and $y\leq x$, we must have $x=y$. Consequently, in a poset-category, the only isomorphisms are the identity morphisms (i.e. corresponding with $x\leq x$).
\end{solution}


\begin{solution}[\cref{exer:section-retraction-bool-int}]\label{sol:section-retraction-bool-int}
Consider
\begin{lstlisting}
bool2Int :: Bool -> Int
bool2Int False = 0
bool2Int True  = 1
\end{lstlisting}    

We can go back, so that we get |False| and |True| from |0| and |1|:
\begin{lstlisting}
int2Bool :: Int -> Bool
int2Bool n | n == 0    = False
           | otherwise = True
\end{lstlisting}
However, notice that not only |1| is converted back to |True|, but also everything other than |0| is converted to |True|.

We have
\begin{lstlisting}
   Int2Bool (bool2Int y) = y
\end{lstlisting}
for every |y :: Bool|, but we don't have |bool2Int (int2Bool x) = x| for all |x :: Int|.

We can say that there is enough room in the type integers for it to host a copy of the type of booleans, but there isn't enough room in the type of booleans for it to host a copy of the type of integers.

But notice that there are other ways in which the type |Bool| lives inside the type |Int| as a retract: for example, we can send |False| to |13| and |True| to |17|, and then send back everything bigger than |15| to |True| and everything else to |False|.
\end{solution}




\begin{solution}[\cref{ex:mono-inj}]\label{sol:mono-inj}
The monomorphisms in $\SET$ correspond precisely with the injective functions, i.e. the functions $f:X\to Y$ which satisfy
\[
\forall x_1,x_2\in X: f(x_1)=f(x_2) \implies x_1=x_2,
\]
Indeed:
\begin{itemize}
\item Assume $f$ is injective. Let $g,h: Z\to X$ be functions such that $\co{g}{f}=\co{h}{f}$. We have to show $g=h$, i.e. 
\[
\forall z\in Z: g(z)=h(z).
\]
Since $f$ is injective, it suffices to show 
\[
\forall z\in Z: f(g(z))=f(h(z)).
\]
But this holds by the condition of $g$ and $h$. Hence, $f$ is indeed a monomorphism.
\item Assume $f$ is a monomorphism. We have to show that for each $x_1,x_2\in X$, we have $f(x_1) = f(x_2)$. Let $\mathbf{1} = \{\star\}$ be a singleton set and define
\[
g_1 : \mathbf{1}\to X: \star\mapsto x_1,\quad  g_2 : \mathbf{1}\to X: \star\mapsto x_2, 
\]
Since $f(x_1)=f(x_2)$, we have $\co{g_1}{f} = \co{g_2}{f}$. But $f$ is a monomorphism, hence $g_1 = g_2$ which means $x_1 = g_1(\star) = g_2(\star) = x_2$. Thus, $f$ is indeed injective.
\end{itemize}
\end{solution}

\begin{solution}[\cref{exer:sections_in_set_injective}]\label{sol:sections_in_set_injective}
Let $f:X\to Y$ be a section with a retraction $h:Y\to X$, i.e. $\co{f}{h} = \Id[Y]$. By \cref{ex:mono-inj}, it suffices to show that $f$ is a monomorphism. Let  $g_1, g_2 : Z \to X$ be morphisms in $\SET$ such that $\co{g_1}{f} = \co{g_2}{f}$. We have to show $g_1=g_2,$, this follows from the following computation:
\begin{eqnarray*}
g_1 =& \co{g_1}{\Id[Y]} & \text{ by unit law},\\ 
	=& \co{g_1}{\co{f}{h}} & \text{ since $f$ section},\\ 
	=& \co{\co{g_1}{f}}{h} & \text{ by associativity},\\
	=&  \co{\co{g_2}{f}}{h} & \text{ by assumption},\\
	=& \co{g_2}{\co{f}{h}} & \text{ by associativity},\\ 
	=& \co{g_2}{\Id[Y]} & \text{ since $f$ section},\\
	=& g_2 & \text{ by unit law}.
\end{eqnarray*}
Notice that this proof shows that in an arbitrary category, a section is always a monomorphism.
\end{solution}

\begin{solution}[\cref{exer:iso_to_monoepi}]\label{sol:iso_to_monoepi}
Let $f : a\cong b$ be an isomorphism with inverse $f^{-1}$. There are multiple proofs which one can give, an abstract one (which is \textit{indirect} in the sense that we use another exercise/lemma) and a more \textit{direct} one.
\begin{itemize}
\item \textbf{Indirect proof:} By the solution of \ref{exer:sections_in_set_injective}, we know that any section (from a section-retraction pair) is a monomorphism. An analoguous argument shows that any retraction (section-retraction pair) is an epimorphism. Hence it suffices to show that $f$ is both a section and a retraction, but this is immediate because $\co{f}{f^{-1}} = \Id[a]$ and $\co{f^{-1}}{f} = \Id[b]$.
\item \textbf{Direct proof:} We first show that $f$ is a monomorphism. Assume $g_1,g_2 : c\to a$ are morphisms such that $\co{g_1}{f} = \co{g_2}{f}$. We then have that $g_1 = g_2$ because 
\begin{eqnarray*}
g_1 = \co{g_1}{\Id[a]} =& \co{g_1}{(\co{f}{f^{-1}})} & \text{ since $f : a\cong b$},\\ 
	=& \co{(\co{g_1}{f})}{f^{-1}} & \text{ by associativity}, \\ 
	=& \co{(\co{g_2}{f})}{f^{-1}} & \text{ by assumption}, \\
	=& \co{g_2}{(\co{f}{f^{-1}})} & \text{ by associativity} \\
	=& \co{g_2}{\Id[a]} & \text{ since $f : a\cong b$} \\
	=& g_2.
\end{eqnarray*}
That $f$ is also an epimorphism is analoguous, indeed: Assume $g_1,g_2 : b\to c$ are morphisms such that $\co{f}{g_1} = \co{f}{g_2}$. We then have that $g_1 = g_2$ because 
\begin{eqnarray*}
g_1 = \co{\Id[b]}{g_1} =& \co{(\co{f^{-1}}{f})}{g_1} & \text{ since $f : a\cong b$},\\ 
	=& \co{f^{-1}}{(\co{f}{g_1})} & \text{ by associativity}, \\ 
	=& \co{f^{-1}}{(\co{f}{g_2})} & \text{ by assumption}, \\
	=& \co{(\co{f^{-1}}{f})}{g_2} & \text{ by associativity} \\
	=& \co{\Id[b]}{g_2} & \text{ since $f : a\cong b$} \\
	=& g_2.
\end{eqnarray*}
\end{itemize}
\end{solution}

\begin{solution}[\cref{exer:counterexample_monoepi_not_iso}]\label{sol:counterexample_monoepi_not_iso}
Let $(X,\leq_X)$ be a preordered set. Any morphism $f \in \POS(X,\leq_X)(x,y)$ is always both a monomorphism and an epimorphism because hom-sets have at most on element. But, by \cref{exer:iso_in_posetcategory}, we know that in a poset (not a preordered set!), the only isomorphisms are the identity morphisms. Hence, if $x\leq y$ but $x\not=y$ (living in a poset), then the corresponding morphism in $\Hom{x}{y}$ is both an epimorphism and monomorphisms but not an isomorphism.\\
A concrete example is given by e.g. the poset of truth values $\{0, 1\}$. We have $0\leq 1$ and those are not equal.
\end{solution}

%% Universal properties
\begin{solution}[\cref{exer:initial_set}] \label{sol:initial_set}
An initial object (and the only one), is the emptyset $\emptyset$, indeed: Let $X$ be a set. Then there is clearly a unique function $\emptyset\to X$.
\end{solution}

\begin{solution}[\cref{exer:initial_posetcat}]\label{sol:initial_posetcat}
A initial object in $\POS(X,\leq)$ is the minimal object, that is an element $\bot\in X$ such that
\begin{equation}
\forall y\in X: \bot \leq y.
\end{equation}
Indeed: Assume $x$ is an initial object in $(X,\leq)$, i.e. for any other element $y\in X$, there exists a (unique) morphism $x\to y$, i.e. hence, by definition of the hom-sets, we have $x\leq y$. So $x$ is indeed the minimal object.

Conversely, assume $\bot$ is a minimal element, hence, for each $y\in X$, we have $\bot\leq y$. Hence $\Hom{\bot}{y}$ is non-empty. So it must contain exactly one element. This means precisely that it is initial.


A somewhat \textit{more compact} solution is as follows: By definition of $\POS(X,\leq)$, for each $x\in X$, we have:
\[
\forall y\in X: \left(x\leq y \iff \exists! f\in \Hom{x}{y}\right).
\]
Hence an object $x$ is initial if and only if, $x\leq y$ for all $y\in X$, if and only if it is minimal.
\end{solution}

\begin{solution}[\cref{exer:initial-unique}]\label{sol:initial-unique}
Let $A$ and $B$ be initial objects in $\CC$. By initiality of $A$ (resp. $B$), there exists a unique morphism $f \in \CHom{\CC}{A}{B}$ (resp. $g \in \CHom{\CC}{B}{A}$). That $f$ and $g$ are inverses follows because $\co{f}{g} \in \CHom{\CC}{A}{A}$ (resp. $\co{g}{f}\in \CHom{\CC}{B}{B}$). But by initiality of $A$ (resp. $B$), $\CHom{\CC}{A}{A}$ (resp. $\CHom{\CC}{B}{B}$) has a unique element, but both $\co{f}{g}, \Id[A] \in \CHom{\CC}{A}{A}$ (resp. $\co{g}{f}, \Id[B] \in \CHom{\CC}{B}{B}$), hence they must be equal.
\end{solution}

\begin{solution}[\cref{exer:initiality_preserved_by_iso}]\label{sol:initiality_preserved_by_iso}
Assume $A\in \Ob{\CC}$ is initial, $B\in\Ob{\CC}$ an arbitrary object and $i:A\cong B$ an isomorphism. We have to show that $B$ is initial, i.e. for each $X\in\Ob{\CC}$, there should exists a unique morphism $B\to X$.

Fix such an $X$. By initiality of $A$, there exists a (unique) morphism $f\in \CHom{\CC}{A}{X}$. If we denote the inverse of $i$ by $j$, we have $\co{j}{f} \in\CHom{\CC}{B}{X}$. To show that $\co{j}{f}$ is the unique morphism in this hom-set, let $g\in \CHom{\CC}{B}{X}$. So we have $\co{i}{g} \in \CHom{\CC}{A}{X}$. By initiality of $A$, we have $\co{i}{g} = f$. The claim now follows by the following computation:
\[
\co{j}{f} = \co{j}{(\co{i}{g})} = \co{(\co{j}{i})}{g} = \co{\Id[B]}{g} = g.
\]
\end{solution}

\begin{solution}[\cref{exer:cat-without-initial}]\label{sol:cat-without-initial}
We give two solutions to this exercise.
\begin{itemize}
\item Consider the category generated by the graph: 
\[
\begin{tikzcd}
x & y
\end{tikzcd}
\]
This category can not have an initial object since there is no morphism from $x$ to $y$ or vice versa.
\item Consider the category generated by the graph: 
\[
\begin{tikzcd}
x \arrow[r, bend left, "f"] \arrow[r, bend right, "g"] & y
\end{tikzcd}
\]
This category also can not have an initial object, indeed: There is no morphism from $y$ to $x$, hence $y$ can not be initial. But also $x$ can not be initial since $f$ and $g$ are different morphisms.
\end{itemize}
\end{solution}



%% Solutions to monads

\begin{solution}[ to \cref{exer:kleisli_triple_list}]
\label{sol:kleisli_triple_list}
For each set $X\in\Ob \SET$, we define:
\[
\eta_X : X \to \List(X) : x\mapsto [x].
\]
For each function $f\in\CHom{\Ob \SET}{X}{\List(Y)}$, we define:
\begin{align*}
f^{*} : \List(X) \to \List(Y) : [x_1,\cdots,x_n] \mapsto 
\begin{cases}
[\:] &\quad \text{ if } n=0,\\
append(f(x_1),\cdots, f(x_n)) &\quad \text{ if } n>0.
\end{cases}
\end{align*}
i.e. $f^{*}([\:]) = [\:]$ and if $l := append(x,s)$ with $x\in X$ and $s\in \List(X)$, then $f^{*}(append(x,s)) = append(f(x), f^{*}(s))$.

We now show that the properties of a Kleisli triple hold:
\begin{enumerate}
\item For each set $X$, we have to show $\eta_X^{*} = \Id[T(X)]$:

we give two (equivalent) ways to writing it. Option $1$ is:
\begin{eqnarray*}
\eta_X^{*}\left([x_1,\cdots,x_n]\right) =& append\left(\eta_X(x_1),\cdots,\eta_X(x_n)\right) & \text{ by definition of $(-)^{\star}$,}\\ 
	=& append([x_1],\cdots,[x_n]) &\text{ by definition of $\eta_X$,}\\ 
	=& [x_1,\cdots,x_n] &\text{ by definition of append,}\\
	=& \Id[{\List([x_1,\cdots,x_n])}].
\end{eqnarray*}
Options $2$ is as follows:
\begin{itemize}
\item By definition of $(-)^{*}$ on the empty list, we have :
\[\eta_{X}^{*}([\:]) = [\:]. \]
\item Let $l := append(x,s)$, then 
\begin{eqnarray*}
\eta_X^{*}(append(x,s)) =& append (\eta_X(x), \eta_X^{*}(s)) &\text{ by definition of } (-)^{*},\\
	=& append(\eta_X(x), \Id[\List{X}](s)) &\text{ by the induction hypothesis,}\\
	=& append(x, \Id[\List{X}](s)) &\text{ by definition of } \eta_X,\\
	=& append(x,s) = t.
\end{eqnarray*}
\end{itemize}

\item For each function $f:X\to Y$, we have to show $f^{*}(\eta_X(x)) = f(x)$, this indeed holds by the following computation:
\[
f^{*}(\eta_X(x)) = f^{*}([x]) = f(x),
\]
where the first equality holds by definition of $\eta_X$ and the second equality holds by definition of $f^{*}$ (and using that append on a single list equals the single list).
\item Let $f:X\to \List Y$ and $g:Y\to \List Z$ be functions, we have to show 
\[
g^{*}(f^{*}([x_1,\cdots,x_n])) = (\co{f}{g^{*}})^{*}([x_1,\cdots,x_n]).
\] 
That this equality holds follows by the following argument:
\begin{itemize}
\item By the definition of $(-)^{*}$ on the empty list, this equality trivially holds for $n=0$.
\item Let $l := append(x,s)$. By definition of $(f)^{*}$ we have:
\[
g^{*}(f^{*}(append(x,s))) = g^{*}(append(f(x),f^{*}(s)).
\]
We also have:
\[
(\co{f}{g^{*}})^{*}(append(x,s)) = append((\co{f}{g^{*}})(x), (\co{f}{g^{*}})^{*}(s)) = append((\co{f}{g^{*}})(x), g^{*}(f^{*}((s))),
\]
where the first equality holds by definition of $(\co{f}{g^{*}})^{*}$ and the second holds by the induction hypothesis. Hence it remains to show the following equality:
\[
g^{*}(append(f(x),f^{*}(s)) = append((\co{f}{g^{*}})(x), g^{*}(f^{*}((s))).
\]

We do a pattern matching on $f(x)$, i.e. we either have $f(x) = [\:]$ or we have $f(x) = append(y,u)$ with $y\in Y$ and $u\in \List Y$:
\begin{itemize}
\item If $f(x) = [\:]$, then 
\begin{eqnarray*}
g^{*}(append(f(x),f^{*}(s)) =& g^{*}(append([\:],f^{*}(s)))\\ 
	=& g^{*}(f^{*}(s))\\ 
	=& append([\:], g^{*}(f^{*}(s)))\\ 
	=& append(g^{*}([\:]), g^{*}(f^{*}(s)))\\ 
	=& append(g^{*}(f(x)), g^{*}(f^{*}(s)))\\ 
	=& append((\co{f}{g^{*}})(x), g^{*}(f^{*}((s))).
\end{eqnarray*}
\item If $f(x) = append(y,u)$, then 
\begin{eqnarray*}
g^{*}(append(f(x),f^{*}(s)) =& g^{*}(append(y,u),f^{*}(s)))\\ 
	=& g^{*}(append(y, append(u,f^{*}(s))))\\
	=& append(g(y), g^{*}(append(u,f^{*}(s))))\\ 
	=& \\ %\KW{This I have to redo}\\
	=& append((append(g(y), g^{*}(u)), g^{*}(f^{*}(s)))\\
	=& append(g^{*}(append(y,u)), g^{*}(f^{*}(s)))\\ 
	=& append(g^{*}(f(x)), g^{*}(f^{*}(s)))\\ 
	=& append((\co{f}{g^{*}})(x), g^{*}(f^{*}((s))).
\end{eqnarray*}
\end{itemize}
\end{itemize}
\end{enumerate}
\end{solution}

\begin{solution}[ to \cref{exer:kleisli_triple_bintree}]
\label{sol:kleisli_triple_bintree}
For each set $X\in\Ob \SET$, we define:
\[
\eta_X : X \to \BinTree(X) : x\mapsto leaf(x).
\]
For each function $f\in\CHom{\Ob \SET}{X}{\BinTree(Y)}$, we define:
\begin{align*}
f^{*} : \BinTree(X) \to \BinTree(Y) : t \mapsto 
\begin{cases}
f(a) &\quad \text{ if } t=leaf(a),\\
branch(f^{*}(t_1),f^{*}(t_2)) &\quad \text{ if } t=branch(t_1,t_2).
\end{cases}
\end{align*}

We now show that the properties of a Kleisli triple hold:
\begin{enumerate}
\item For each set $X$, we have to show $\eta_X^{*} = \Id[\BinTree(X)]$. We show this by pattern matching on $t$:
\begin{itemize}
\item If $t=leaf(a)$, then
\[
\eta_X^{*}(t) = \eta_X^{*}(leaf(a)) = \eta_X(a) = leaf(a)=t.
\]
\item If $t=branch(t_1,t_2)$, then
\[
\eta_X^{*}(t) = \eta_X^{*}(branch(t_1,t_2)) = branch(\eta_X^{*}(t_1),\eta_X^{*}(t_2)) = branch(t_1,t_2) = t.
\]
\end{itemize}

\item For each function $f:X\to Y$, we have to show $f^{*}(\eta_X(a)) = f(a)$, this indeed holds by the following computation:
\[
f^{*}(\eta_X(a)) = f^{*}(leaf(a)) = f(a).
\]

\item Let $f:X\to \BinTree Y$ and $g:Y\to \BinTree Z$ be functions, we have to show 
\[
g^{*}(f^{*}(t)) = (\co{f}{g^{*}})^{*}(t).
\] 
That this equality holds follows by pattern matching:
\begin{itemize}
\item If $t=leaf(a)$, then
\[
g^{*}(f^{*}(t)) = g^{*}(f^{*}(leaf(a)) = g^{*}(f(a)) = (\co{f}{g^{*}})(a) = (\co{f}{g^{*}})^{*}(leaf(a)) = (\co{f}{g^{*}})^{*}(t).
\]
\item If $t=branch(t_1,t_2)$, then is the left-hand-side given by
\[
g^{*}(f^{*}(t)) = g^{*}(f^{*}(branch(t_1,t_2))) = g^{*}\left(branch(f^{*}(t_1), f^{*}(t_2))\right) = branch(g^{*}(f^{*}(t_1)),g^{*}(f^{*}(t_2))).
\]
The right-hand-side is given by 
\[
(\co{f}{g^{*}})^{*}(t) = (\co{f}{g^{*}})^{*}(branch(t_1,t_2)) = branch\left((\co{f}{g^{*}})(t_1), (\co{f}{g^{*}})(t_2)\right).
\]
Hence, by the induction hypothesis, the both sides are equal.
\end{itemize}

\end{enumerate}
\end{solution}

\begin{solution}[ to \cref{exer:kleisli_triple_maybe}]
\label{sol:kleisli_triple_maybe}
Before we continue with this exercise, we first fix some notation. Since $X+E$ is the disjoint union of $X$ and $E$, we have the canonical inclusions which we denote by
\[ i^X_l : X\to X + E, \quad i^X_r : E\to X+E. \] 
Hence, a function whose domain is $X+E$ is completely determined by specifiying where each $i^X_l(x)$ and each $i^X_r(e)$ are mapped to. \textit{Notice that this is precisely the notation and the universal property of the coproduct (in $\SET$)}.

For each set $X\in\Ob \SET$, we define:
\[
\eta_X : X \to X+E : x\mapsto i^X_l(x).
\]
For each function $f\in\CHom{\Ob \SET}{X}{Y+E}$, we define:
\begin{align*}
f^{*} :X+E \to Y+E : z \mapsto 
\begin{cases}
f(x) &\quad \text{ if } z=i^X_l(x),\\
i^Y_r(e) &\quad \text{ if } z=i^X_r(e).
\end{cases}
\end{align*}


We now show that the properties of a Kleisli triple hold:
\begin{enumerate}
\item For each set $X$, we have to show $\eta_X^{*} = \Id[X+E]$: 
\begin{itemize}
\item If $z=i^X_l(x)$, then 
\[
\eta_X^{*}(z) = \eta_X^{*}(i^X_l(x)) = \eta_X(x) = i^X_l(x) = z = \Id[X+E](z).
\]
\item If $z=i^X_r(e)$, then
\[
\eta_X^{*}(e) = \eta_X^{*}(i^X_r(e)) = i^X_r(e) = z = \Id[X+E](z).
\]
\end{itemize}

\item For each function $f:X\to Y$, we have to show $f^{*}(\eta_X(x)) = f(x)$ but this holds directly by the definition of $(-)^{*}$ since $\eta_X(x)=i_l^X(x)$.

\item Let $f:X\to Y + E$ and $g:Y\to Z + E$ be functions, we have to show 
\[
g^{*}(f^{*}(z)) = (\co{f}{g^{*}})^{*}(z).
\] 
To show this, we do pattern matching on $z\in X+E$:
\begin{itemize}
\item If $z=i^X_l(x)$, then
\[
g^{*}(f^{*}(z)) = g^{*}(f^{*}(i_l^X(x))) = g^{*}(f(x)) = (\co{f}{g^{*}})^{*}(i_l^X(x)) = (\co{f}{g^{*}})^{*}(z).
\]
\item If $z=i^X_r(e)$, then
\[
g^{*}(f^{*}(z)) = g^{*}(f^{*}(i_l^X(e))) = g^{*}(i_l^Y(e)) = i_l^Z(e) = (\co{f}{g^{*}})^{*}(i_l^X(e)) = (\co{f}{g^{*}})^{*}(z).
\]
\end{itemize}

\end{enumerate}
\end{solution}

\begin{solution}[ to \cref{exer:kleisli_triple_nondeterminism}]
For each set $X\in\Ob \SET$, we define:
\[
\eta_X : X \to \mathbb{P}_{fin}(X) : x\mapsto \{x\}.
\]
For each function $f\in\CHom{\Ob \SET}{X}{\mathbb{P}_{fin}(Y)}$, we define:
\begin{align*}
f^{*} : \mathbb{P}_{fin}(X) \to \mathbb{P}_{fin}(Y) : A \mapsto \bigcup_{a\in A} f(a).
\end{align*}
First notice that $\eta_X$ and $f^{*}$ are well-defined, indeed: 
\begin{itemize}
\item $\eta_X(x) = \{x\}$ is clearly finite since it only contains one element.
\item Let $A\in \mathbb{P}_{fin}(X)$. By definition of $f$, for each $a\in A$, $f(a)$ is finite. But there are only a finite number of elements in $A$, so $\bigcup_{a\in A} f(a)$ is a finite union of finite sets, hence, it is again finite.
\end{itemize}
We now show that the properties of a Kleisli triple hold:
\begin{enumerate}
\item For each set $X$, we have to show $\eta_X^{*} = \Id[\mathbb{P}_{fin}(X)]$. Let $A\in \mathbb{P}_{fin}(X)$, the claim then follows by the following computation:
\[
\eta_X^{*}(A) = \bigcup_{a\in A} \eta_X(a) = \bigcup_{a\in A} \{a\} = A = \Id[\mathbb{P}_{fin}(X)](A).
\]

\item For each function $f:X\to Y$, we have to show $f^{*}(\eta_X(x)) = f(x)$ but this holds directly by the definition of $(-)^{*}$ since
\[
f^{*}(\eta_X(x)) = f^{*}(\{x\}) = \bigcup_{a\in \{x\}} f(a) = f(x).
\]

\item Let $f:X\to \mathbb{P}_{fin}(Y)$ and $g:Y\to \mathbb{P}_{fin}(Z)$ be functions, we have to show 
\[
g^{*}(f^{*}(A)) = (\co{f}{g^{*}})^{*}(A).
\] 
Let $A\in \mathbb{P}_{fin}(X)$, the left-hand-side is given as:
\[
g^{*}(f^{*}(A)) = g^{*}\left( \bigcup_{a\in A} f(a) \right) = \bigcup_{b \in \bigcup_{a\in A} f(a)} g(f(a)) = \bigcup_{a\in A} \bigcup_{b\in f(a)} g(f(a)).
\]
The right-hand-side is given as:
\[
(\co{f}{g^{*}})^{*}(A) = \bigcup_{a\in A} (\co{f}{g^{*}})(a) = \bigcup_{a\in A} g^{*}(f(a)) = \bigcup_{a\in A} \bigcup_{b\in f(a)} g(f(a))
\]
Hence, both sides are equal.

\end{enumerate}
\end{solution}

\begin{solution}[ to \cref{exer:kleisli_triple_continuation}]
\label{sol:kleisli_triple_continuation}
For each set $X\in\Ob \SET$, we define:
\[
\eta_X : X \to (X\to R)\to R : x\mapsto (\lambda f. f(x)).
\]
For each function $f\in\CHom{\Ob \SET}{X}{Cont^R(Y)}$, we define:
\begin{align*}
f^{*} : Cont^R(X) \to Cont^R(Y) : i \mapsto \lambda (j:Y\to R), i(f(-)(j)).
\end{align*}
Notice that this is indeed well-defined: Let $i\in Cont^R(X)$, i.e. $i:(X\to R)\to R$. Then $f^{*}(i) : (Y\to R)\to R$. Let $j:Y\to R$. Then $f(-)(j) : X\to R$, hence we can apply it to $i$ and we have $i\left(f(-)(j)\right)\in R$.

Let $i\in Cont^R(X)$. We now show that this data satisfies the properties of a Kleisli triple:
\begin{enumerate}
\item For each set $X$, we have to show $\eta_X^{*} = \Id[Cont^R(X)]$. Let $x\in X$. The claim then follows by the following computation:
\[
\eta_X^{*}(i) = \lambda j, i\left(\eta_X(-)(j)\right) = \lambda j, i\left(\lambda x, j(x)\right) = i
\]

\item For each function $f:X\to Y$, we have to show $f^{*}(\eta_X(x)) = f(x)$, this follows by the following computation:
\[
f^{*}(\eta_X(x)) = f^{*}(\lambda g,g(x)) = \lambda j, \left(\left(\lambda g,g(x)\right)(f(-)(j))\right) = \lambda j, (f(x)(j)) = f(x).
\]

\item Let $f:X\to Cont^R(Y)$ and $g:Y\to Cont^R(Z)$ be functions, we have to show 
\[
g^{*}(f^{*}(i)) = (\co{f}{g^{*}})^{*}(i).
\] 
The left-hand-side is given as:
\[
g^{*}(f^{*}(i)) = g^{*}\left(\lambda j, i\left(f(-)(j)\right)\right) = \lambda \tilde{j}, \left(\lambda j, i\left(f(-)(j)\right)\right)\left(g(-)(\tilde{j})\right) = \lambda \tilde{j}, i\left(f(-)\left(g(-)(\tilde{j})\right)\right).
\]
The right-hand-side is given as:
\[
(\co{f}{g^{*}})^{*}(i) = \lambda j, i\left((\co{f}{g^{*}})(-)(j)\right)
\]
So to show that both sides are equal, it sufficies to show that for each $j$, we have 
\[
f(-)\left(g(-)(j)\right) = (\co{f}{g^{*}})(-)(j).
\]
Notice that these are functions $X\to R$. Hence we will show this pointwise for each $x\in X$. The left-hand-side is given by: 
\[
f(-)\left(g(-)(j)\right)(x) = f(x)\left(g(-)(j)\right) 
\]
The right-hand-side is given by:
\begin{eqnarray*}
(\co{f}{g^{*}})(-)(j)(x) &=& (\co{f}{g^{*}})(x)(j)\\ 
	&=& \left( (\co{f}{g^{*}})(x) \right)(j)\\ 
	&=& \left(g^{*}(f(x))\right)(j)\\ 
	&=& \left( \lambda k, f(x)\left(g(-)(k)\right) \right)(j)\\ 
	&=& f(x)\left(g(-)(j)\right).
\end{eqnarray*}
Hence, both sides are equal.

\end{enumerate}
\end{solution}




\begin{solution}[\cref{exer:initial_pointset}]\label{sol:initial_pointset}
	An initial object is a one-element set $ (\{ \star \}, \star) $. Let $ (X, x) $ be a pointed set. Then we have a unique function $ \star \to X $ that sends $ \star $, the chosen (and only) point of $ \{ \star \} $, to $ x $, the chosen point of $ X $, namely $ f: \star \mapsto x $.
\end{solution}

\begin{solution}[\cref{exer:terminal_set}]\label{sol:terminal_set}
	A terminal set in the category of sets is a one-element set $ \{ \star \} $. Given any set $ X $, we have a unique function $ X \to \star $, since any element of $ X $ must be sent to $ \star $.
\end{solution}

\begin{solution}[\cref{exer:terminal_posetcat}]\label{sol:terminal_posetcat}
	Given a poset $ (X, \leq) $, a terminal object in the category $ \POS(X, \leq) $ is exactly a maximal element in $ X $. Indeed, given such a maximal element $ x \in X = \POS(X, \leq)_0 $, we have for all $ y \in \POS(X, \leq)_0 $, since $ x $ is maximal, that $ y \leq x $. Therefore, we have a morphism $ f: x \to y $. By the definition of the hom-sets in $ \POS(X, \leq) $, $ f $ is unique and we conclude that $ x $ is a terminal object.
	Conversely, unfolding the definition of terminal object shows that a terminal object yields a maximal element.
\end{solution}

\begin{solution}[\cref{exer:terminal-unique}]\label{sol:terminal-unique}
	Suppose that we have a category $ \CC $ and two terminal objects $ B, B^\prime \in \Ob\CC $. Since $ B^\prime $ is terminal, we have a morphism $ f: B \to B^\prime $ and since $ B $ is terminal, we have a morphism $ g: B^\prime \to B $. Note that we have two morphisms from $ B $ to $ B $, namely $ g \circ f $ and $ \Id[B] $. Also, because $ B $ is terminal, there exists a unique morphism $ B \to B $. Therefore, $ g \circ f = \Id[B] $. In the same way, we have $ f \circ g = \Id[B^\prime] $. Therefore, $ f $ is the isomorphism (with inverse $g$) between $ B $ and $ B^\prime $ that we are looking for.
\end{solution}


%%% Local Variables:
%%% mode: latex
%%% TeX-master: "CT4P"
%%% End:



\appendix

\onlydraft{
\section{Forgetful and free functors}
A lot of (mathematical) structures are defined as some other kind of mathematical structure, but where extra structure is added. An example of this is the following:\\
Recall that a monoid is a set $M$ together with a binary operation $m:M\to M\to M$ which is associative and such that there is an identity element $e$ (see \cref{monoidcategory}). In particular, any monoid has an underlying set and any morphism of monoids has an underlying function (between those sets). So forgetting the binary operation and identity element defines a functor from $\MON$ to $\SET$ which is called a \textit{forgetful functor}:
\begin{exa}\label{example:forgetful_montoset} The \textbf{forgetful functor from $\MON$ to $\SET$} is the functor specified by the following data:
\begin{itemize}
\item The function on objects is given by 
\[
\Ob{\MON}\to \Ob{\SET}: (M,m,e)\mapsto M.
\]
\item The function on objects is given by
\[
\CHom{\MON}{(M_1,m_1,e_1)}{(M_2,m_2,e_2)} \to \CHom{\SET}{M_1}{M_2} : f\mapsto f.
\]
\end{itemize}
\end{exa}
Notice that if one defines a category whose objects are sets $M$ together with an associative binary operation $m:M\to M\to M$, then one could analogously also define a forgetful functor from $\MON$ to this category by only forgetting the neutral element.

\begin{lemma} The forgetful functor from $\MON$ to $\SET$ satisfies the properties of a functor.
\begin{proof}
We clearly have that everything is well-defined since the codomain is $\SET$.\\
That the identity morphism is preserved holds by definition because the identity morphism of $(M,m,e)$ (in $\MON$) is given by the identity function and the identity morphism of $M$ (in $\SET$) is also given by the identity function.\\
That the composition of morphisms is preserved also holds by definition because the composition of morphisms (in $\MON$) is given by the composition of the underlying functions which is also the composition in $\SET$.
\end{proof}
\end{lemma}

The forgetful functor $Forget$ from $\MON$ to $\SET$ forgets the \textit{algebraic} structure of a monoid and since there are multiple monoid structures on the same set (given an example of this), hence we do not have that there exists some functor $G: \SET\to \MON$ such that $Forget\Comp G$ is the identity on $\MON$. However, to each set, one can define a monoid which satisfies an important property (this is \cref{prop:UVP_forget_montoset}). The associated monoid is called the \textit{free monoid}:
\begin{exa} Let $X$ be a set. The \textbf{free monoid generated by $X$}, denoted by $Free(X)$, is specified by the following data:
\begin{itemize}
\item The underlying set consists of all finite sequences/strings of elements in $X$ (including the empty sequence).
\item The multiplication is defined by concatenating the sequences, i.e. 
$$m\left((x_1,\cdots,x_n),(y_1,\cdots,y_m)\right) := (x_1,\cdots,x_n,y_1,\cdots,y_m).$$
\item The identity element is given by the empty sequence.
\end{itemize}
\end{exa}

\begin{exa}\label{exa:freemonoids} The \textbf{free functor from $\SET$ to $\MON$} is specified by the following data:
\begin{itemize}
\item The function on objects is given by 
\[
\Ob{\SET}\to \Ob{\MON}: X\mapsto Free(X).
\]
\item The function on morphisms is given as follows: A morphism $f \in \CHom{\SET}{X}{Y}$ (i.e. a function) is mapped to the monoidal morphism which is given by pointwise application of $f$, i.e.
\[
Free(f)(x_1,\cdots,x_n) := (f(x_1),\cdots, f(x_n)).
\]
\end{itemize}
\end{exa}

\begin{exer} Show that $Free$ satisfies the properties of a functor.
\end{exer}

For any set $X$, we have the \textit{canonical function} 
\[
Free^{X}_{!}: X\to Free(X): x\mapsto (x).
\]
This function satisfies the property that any function from $X$ to an arbitrary monoid corresponds with a unique morphism (of monoids) from the free monoid generated by $X$ to that monoid:
\begin{prop}\label{prop:UVP_forget_montoset} Let $(M,m,e)$ be a monoid and $X$ be a set. For any morphism $f \in \CHom{\SET}{X}{M}$ (i.e. a function), there exists a unique  morphism $\phi^{f} \in \CHom{\MON}{Free(X)}{(M,m,e)}$, such that $f = Free^{X}_{!}\Comp \phi^{f}$.
\begin{proof}
For elements $a,b\in M$, we denote their multiplication by $a\times b := m(a,b)$ (note that by associativity we have that $a_1\times a_2\times\cdots\times a_n)$ is well-defined). Let $f\in \CHom{\SET}{X}{M}$ be a function. Define 
\[
\phi^{f}: Free(X)\to (M,m,e): (x_1\cdots,x_n)\mapsto f(x_1) \times\cdots \times f(x_n),
\] 
We have to define separately what happens with the empty sequence. The empty sequence we map to the identity element $e$, so in particular we have that the identity element is preserved under $\phi^{f}$, so in order to conclude that $\phi^{f}$ is a morphism of monoids, it remains to show that it preserves the binary operation but this is clear by definition.\\
That $f = Free^{X}_{!}\Comp \phi^{f}$ holds follows immediately by the definition of $Free^{X}_{!}$ and $\phi^{f}$, indeed:
\[
\left(Free^{X}_{!}\Comp \phi^{f}\right)(x) = \phi^{f}\left((x)\right) = f(x).
\]
So it only remains to show uniqueness. Assume that $\psi$ also satisfies $Free^{X}_{!}\Comp \psi = f$. The claim now follows because $\psi$ is a morphism of monoids, indeed: Since $\psi$ is morphism is monoids, we have that it preserves the multiplication, but the multiplication is given by concatenation, hence, we have that $\psi$ is uniquely determined by the images of the sequences of length $1$ (and length $0$, but this sequence of length $0$ should be mapped under $\psi$ to $e$). But a sequence of length $1$ is of the form $(x) = Free^{X}_{!}(x)$. So the claim indeed follows by the following computation: 
\[
\psi((x)) = \psi(Free^{X}_{!}(x)) = f(x) = \phi^{f}(Free^{X}_{!}(x)) = \phi^{f}((x)).
\]
\end{proof}
\end{prop}

The following exercise shows that there is a special connection between the forgetful functor $\mathit{forget}: \MON\to\SET$ and the free functor $\mathit{free}: \SET\to\MON$. This connection expresses that these form a so-called \textit{adjoint pair} (see \cref{sec:adjunctions}).
\begin{exer}\label{exer:preadjunction_monset}
Show that for any set $X$ and monoid $(M,m,e)$, there exist bijections between the hom-sets:
\[
\alpha_{X}^{(M,m,e)} : \CHom{\MON}{Free(X)}{(M,m,e)} \to \CHom{\SET}{X}{forget(M,m,e)}.
\]
Hint: use \cref{prop:UVP_forget_montoset}.
\end{exer}

\begin{exer} Define a forgetful functor from the category $\POS$ of posets (defined in \cref{example:poset}) to $\SET$ analogous to the the forgetful functor from $\MON$ to $\SET$ (defined in \cref{example:forgetful_montoset}).
\end{exer}
\begin{rem} The story about free monoids can not be repeated for posets, i.e. there is no free poset structure on all sets. But in order to prove this one needs more machinery.
\end{rem}


%%% Local Variables:
%%% mode: latex
%%% TeX-master: "CT4P"
%%% End:
}

\onlydraft{\subsection{Contravariant functors}

A variation on functors are \textbf{contravariant functors}.
A contravariant functor consists of a map on objects, just like a functor.
However, the \textbf{map on morphisms turns the morphisms around}.
We give the formal definition:

\begin{dfn} Let $\CC$ and $\DD$ be categories. A \textbf{contravariant functor} $F$ from $\CC$ to $\DD$ consists of the following data:
\begin{itemize}
\item A function 
\[
\Ob{\CC} \to \Ob{\DD},
\]
written as $X\mapsto F(X)$.
\item For each $X,Y\in \Ob{\CC}$, a function
\[
\CHom{\CC}{X}{Y} \to \CHom{\DD}{F(Y)}{F(X)},
\]
written as $f\mapsto F(f)$.
\end{itemize}
Moreover, this data should satisfy the following properties:
\begin{itemize}
\item (\textbf{Preserves composition}) For $f\in \Hom[\CC]{X}{Y}$ and $g\in \Hom[\CC]{Y}{Z}$, we have $F(\co f g) =  \co {F(g)}{F(f)}$.
\item (\textbf{Preserves identity}) For $X\in\CC$, we have $F(\Id[X]) = \Id[F(X)]$.
\end{itemize}
\end{dfn}

\begin{exer} Notice that the preservation of composition has now changed; why is this the case?
\end{exer}

An example of a contravariant functor is given by the powerset-functor:
\begin{exa} \label{example:powersetfunctor} Recall that the powerset of a set $X$, denoted by $\mathbb{P}(X)$, is the set of all subsets of $X$, i.e. 
\[
\mathbb{P}(X) :=  \left\{A \mid A\subseteq X\right\}.
\]
The contravariant \textbf{powerset-functor}\footnote{Since a function is mapped to the inverse-image function, one also calls the powerset-functor, an inverse image-functor} (on sets), denoted by $\mathbb{P}$, is the functor from $\SET$ to $\SET$ defined by the following data:
\begin{itemize}
\item The function on objects is given by:
\[
\Ob{\SET}\to \Ob{\SET}: X\mapsto \mathbb{P}(X).
\]
\item For each $X,Y\in\SET$, the function on morphisms is given by
\[
\CHom{\SET}{X}{Y} \to \CHom{\SET}{\mathbb{P}(X)}{\mathbb{P}(Y)}: f\mapsto f^{-1},
\]
where $f^{-1}$ given by
\[
f^{-1}:\mathbb{P}(Y)\to \mathbb{P}(X): B\mapsto f^{-1}(B) := \left\{x\in X \mid f(x)\in B\right\}.
\]
\end{itemize}
\end{exa}

\begin{exer} Show that $\mathbb{P}$, defined in \cref{example:powersetfunctor} satisfies the properties of a contravariant functor.
\end{exer}



%%% Local Variables:
%%% mode: latex
%%% TeX-master: "CT4P"
%%% End:
}

\printbibliography

\end{document}

%%% Local Variables:
%%% mode: latex
%%% TeX-master: t
%%% End:
