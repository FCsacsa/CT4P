
\section{Monads and Effects}
\label{sec:monads}
Recall that a monad (as defined, e.g., in Haskell) is a function |m :: * -> *| together with the additional data of a function |pure :: a -> m a| (for each type |a|) and a function |(>>=) :: m a -> (a -> m b) -> m b| (for types |a| and |b|).

The operations |pure| and |(>>=)| are expected to satisfy the following laws:
such that they satisfy the following properties:
\begin{enumerate}
\item |t >>= pure == t| %$\bind(X, pure) = X$.
\item |pure(x) >>= f == f x| %$bind(pure(X), f) = f(X)$.
\item |(t >>= f) >>= g == t >>= (\x -> f x >>= g)|% $bind(bind(X,f), g) = bind(X, \lambda x. bind(f(x), g)$.
\end{enumerate}
However, these laws cannot be enforced in Haskell, since Haskell does not have any infrastructure for logic.

In a category, however, we can define monads including the monad laws.
We will actually give two different definitions of monad;
one called ``Kleisli triple'' (\cref{def:kleisli-triple}), which corresponds to what is called ``monad'' in Haskell,
and one called ``monad'' (\cref{def:monad}). 
The formulation of monads uses that the arguments to |(>>=)| can be reordered.

\begin{dfn}\label{def:kleisli-triple}
  A \textbf{Kleisli triple} over a category $\CC$ is  consisting of the following data:
\begin{itemize}
\item A function $T: \Ob{\CC}\to \Ob{\CC}$.
\item For each $X\in\Ob{\CC}$, a morphism $\eta_X \in \CHom{\CC}{X}{T(X)}$.
\item For each $f\in\CHom{\CC}{X}{T(Y)}$, a morphism $f^{*} \in \CHom{\CC}{T(X)}{T(Y)}$.
\end{itemize}
such that the following properties holds:
\begin{enumerate}
\item For each $X \in \Ob\CC$, we have $\eta_X^{*} = \Id[T(X)]$.
\item For each $f\in\CHom{\CC}{X}{T(Y)}$, the following diagram commutes:
\begin{center}
\begin{tikzcd}
X \arrow[r, "{\eta_X}"] \arrow[rd,swap,"f"] & T(X) \arrow[d, "f^{*}"] \\
& T(Y)
\end{tikzcd}
\end{center}
\item For each $f\in\CHom{\CC}{X}{T(Y)}$ and $g\in\CHom{\CC}{Y}{T(Z)}$, the following diagram commutes:
\begin{center}
\begin{tikzcd}
T(X) \arrow[r, "f^{*}"] \arrow[rd,swap,"{(\co{f}{g^{*}})^{*}}"] & T(Y) \arrow[d, "g^{*}"] \\
& T(Z)
\end{tikzcd}
\end{center}
\end{enumerate}
We denote a Kleisli triple as $(T,\eta, (-)^{*})$.
\end{dfn}

\begin{exer}
  Convince yourself that the operations and laws of a Kleisli triple correspond, in the category $\HASK$, to the operations and properties of a monad in Haskell.
\end{exer}

\begin{exer}[\cref{sol:kleisli_triple_list}]\label{exer:kleisli_triple_list} Show how the following assignment induces a Kleisli triple over the category $\SET$:
\[
X\mapsto \List(X).
\]
The resulting monad is called the \textbf{List monad}.
\end{exer}

\begin{exer}[\cref{sol:kleisli_triple_bintree}]\label{exer:kleisli_triple_bintree} Show how the following assignment induces a Kleisli triple over the category $\SET$:
\[
X\mapsto \BinTree(X),
\]
where $\BinTree(X)$ is the set of binary trees labelled with elements from $X$ at the leaves, that is the set inductively generated by the constructors $leaf: X\to \BinTree(X)$ and $branch : \BinTree(X)\to \BinTree(X)\to \BinTree(X)$.
The resulting monad is called the \textbf{Tree monad}.
\end{exer}



\begin{exer}[\cref{sol:kleisli_triple_maybe}]\label{exer:kleisli_triple_maybe} Let $E$ be a set (considered as a set of \textit{exceptions}). Show how the following assignment induces a Kleisli triple over the category $\SET$:
\[
X\mapsto (X + E),
\]
The resulting monad is called the \textbf{Exception monad}.
\end{exer}



\begin{exer}[\cref{sol:kleisli_triple_nondeterminism}]\label{exer:kleisli_triple_nondeterminism} Show how the following assignment induces a Kleisli triple over the category $\SET$:
\[
X\mapsto \mathbb{P}_{fin}(X) := \left\{A\subseteq X \mid  A \text{ is finite}\right\}.
\]
The resulting monad is called the \textbf{Monad of nondeterminism}.
\end{exer}


\begin{exer}[\cref{sol:kleisli_triple_continuation}]\label{exer:kleisli_triple_continuation} Let $R$ be a set (considered as a set of \textit{results}). Show how the following assignment induces a Kleisli triple over the category $\SET$:
\[
X\mapsto Cont^R(X) := (X \to R) \to R.
\]
The resulting monad is called the \textbf{Continuation monad}.
\end{exer}


\begin{exer}\label{exer:kleisli_triple_familiesofelements} Let $R$ be a set. Show how the following assignment induces a Kleisli triple over the category $\SET$: 
\[
X \mapsto R \to X
\]
The resulting monad is called the \textbf{Monad of families of elements}.

\end{exer}



The notion of a Kleisli triple can equivalently be described  as follows:
\begin{dfn}\label{def:monad}
A \textbf{monad} over a category $\CC$ consists of the following data:
\begin{itemize}
\item A (endo)functor $T:\CC\to\CC$.
\item A natural transformation $\NatTrans{\eta}{\Id[\CC]}{T}$.
\item A natural transformation (``multiplication'') $\NatTrans{\mu}{\co{T}{T}}{T}$.
\end{itemize}
such that for each $X\in\Ob{\CC}$ the following diagrams commute:
\begin{center}
\begin{tikzcd}
T^3(X) \arrow[r, "\mu_{T(X)}"] \arrow[d,swap, "T(\mu_X)"] & T^2(X) \arrow[d, "\mu_X"] \\
T^2(X) \arrow[r,swap, "\mu_X"] & T(X)
\end{tikzcd}
\quad
\begin{tikzcd} 
T(X) \arrow[r, "{\eta_{T(X)}}"] \arrow[rd,swap, "{\Id[T(X)]}"] 
& T^2(X) \arrow[d,"{\mu_X}"] & T(X) \arrow[l,swap,"{T(\mu_X)}"] \arrow[ld, "{\Id[T(X)]}"] \\
& T(X) &
\end{tikzcd}

\end{center}
where we denote $T^2 := \co{T}{T}$ and $T^3 := \co{T}{\co{T}{T}}$.
\end{dfn}

\begin{exer} Given a monad, construct a Kleisli triple from it.
Conversely, given a Kleisli triple, construct a monad from it.
\end{exer}

\begin{exer}
  For each of the Kleisli triples of \cref{exer:examples-kleisli-triples}, describe the monad multiplication $\mu$ obtained from it.
\end{exer}

Every Kleisli triple induces a category:
\begin{dfn} Let $(T,\eta, (-)^{*})$ be a Kleisli triple over $\CC$. The \textbf{Kleisli category}, denoted by $\CC_{T}$, is the category defined by the following data:
\begin{itemize}
\item $\Ob{(\CC_T)} := \Ob{\CC}$.
\item For each $X,Y\in\Ob{(\CC_T)}$, $\CHom{\CC_T}{X}{Y} := \CHom{\CC}{X}{TY}$.
\item The identity on $X\in\Ob{(\CC_T)}$ is $\eta_X$.
\item The composition of $f\in \CHom{\CC_T}{X}{Y}$ and $f\in \CHom{\CC_T}{Y}{Z}$ is $\co{f}{g^{*}}$.
\end{itemize}
\end{dfn}

\begin{exer} Show that for every Kleisli triple, its Kleisli category satisfies the properties of a category.
\end{exer}


%%% Local Variables:
%%% mode: latex
%%% TeX-master: "cats-for-programmers"
%%% End:
