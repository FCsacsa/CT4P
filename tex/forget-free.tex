
\section{Forgetful and Free Functors}
A lot of (mathematical) structures are defined as some other kind of mathematical structure, but where extra structure is added. An example of this is the following:\\
Recall that a monoid is a set $M$ together with a binary operation $m:M\to M\to M$ which is associative and such that there is an identity element $e$ (see \cref{monoidcategory}). In particular, any monoid has an underlying set and any morphism of monoids has an underlying function (between those sets). So forgetting the binary operation and identity element defines a functor from $\MON$ to $\SET$ which is called a \textit{forgetful functor}:
\begin{exa}\label{example:forgetful_montoset} The \textbf{forgetful functor from $\MON$ to $\SET$} is the functor specified by the following data:
\begin{itemize}
\item The function on objects is given by 
\[
\Ob{\MON}\to \Ob{\SET}: (M,m,e)\mapsto M.
\]
\item The function on objects is given by
\[
\CHom{\MON}{(M_1,m_1,e_1)}{(M_2,m_2,e_2)} \to \CHom{\SET}{M_1}{M_2} : f\mapsto f.
\]
\end{itemize}
\end{exa}
Notice that if one defines a category whose objects are sets $M$ together with an associative binary operation $m:M\to M\to M$, then one could analogously also define a forgetful functor from $\MON$ to this category by only forgetting the neutral element.

\begin{lemma} The forgetful functor from $\MON$ to $\SET$ satisfies the properties of a functor.
\begin{proof}
We clearly have that everything is well-defined since the codomain is $\SET$.\\
That the identity morphism is preserved holds by definition because the identity morphism of $(M,m,e)$ (in $\MON$) is given by the identity function and the identity morphism of $M$ (in $\SET$) is also given by the identity function.\\
That the composition of morphisms is preserved also holds by definition because the composition of morphisms (in $\MON$) is given by the composition of the underlying functions which is also the composition in $\SET$.
\end{proof}
\end{lemma}

The forgetful functor $Forget$ from $\MON$ to $\SET$ forgets the \textit{algebraic} structure of a monoid and since there are multiple monoid structures on the same set (given an example of this), hence we do not have that there exists some functor $G: \SET\to \MON$ such that $Forget\Comp G$ is the identity on $\MON$. However, to each set, one can define a monoid which satisfies an important property (this is \cref{prop:UVP_forget_montoset}). The associated monoid is called the \textit{free monoid}:
\begin{exa} Let $X$ be a set. The \textbf{free monoid generated by $X$}, denoted by $Free(X)$, is specified by the following data:
\begin{itemize}
\item The underlying set consists of all finite sequences/strings of elements in $X$ (including the empty sequence).
\item The multiplication is defined by concatenating the sequences, i.e. 
$$m\left((x_1,\cdots,x_n),(y_1,\cdots,y_m)\right) := (x_1,\cdots,x_n,y_1,\cdots,y_m).$$
\item The identity element is given by the empty sequence.
\end{itemize}
\end{exa}

\begin{exa}\label{exa:freemonoids} The \textbf{free functor from $\SET$ to $\MON$} is specified by the following data:
\begin{itemize}
\item The function on objects is given by 
\[
\Ob{\SET}\to \Ob{\MON}: X\mapsto Free(X).
\]
\item The function on morphisms is given as follows: A morphism $f \in \CHom{\SET}{X}{Y}$ (i.e. a function) is mapped to the monoidal morphism which is given by pointwise application of $f$, i.e.
\[
Free(f)(x_1,\cdots,x_n) := (f(x_1),\cdots, f(x_n)).
\]
\end{itemize}
\end{exa}

\begin{exer} Show that $Free$ satisfies the properties of a functor.
\end{exer}

For any set $X$, we have the \textit{canonical function} 
\[
Free^{X}_{!}: X\to Free(X): x\mapsto (x).
\]
This function satisfies the property that any function from $X$ to an arbitrary monoid corresponds with a unique morphism (of monoids) from the free monoid generated by $X$ to that monoid:
\begin{prop}\label{prop:UVP_forget_montoset} Let $(M,m,e)$ be a monoid and $X$ be a set. For any morphism $f \in \CHom{\SET}{X}{M}$ (i.e. a function), there exists a unique  morphism $\phi^{f} \in \CHom{\MON}{Free(X)}{(M,m,e)}$, such that $f = Free^{X}_{!}\Comp \phi^{f}$.
\begin{proof}
For elements $a,b\in M$, we denote their multiplication by $a\times b := m(a,b)$ (note that by associativity we have that $a_1\times a_2\times\cdots\times a_n)$ is well-defined). Let $f\in \CHom{\SET}{X}{M}$ be a function. Define 
\[
\phi^{f}: Free(X)\to (M,m,e): (x_1\cdots,x_n)\mapsto f(x_1) \times\cdots \times f(x_n),
\] 
We have to define separately what happens with the empty sequence. The empty sequence we map to the identity element $e$, so in particular we have that the identity element is preserved under $\phi^{f}$, so in order to conclude that $\phi^{f}$ is a morphism of monoids, it remains to show that it preserves the binary operation but this is clear by definition.\\
That $f = Free^{X}_{!}\Comp \phi^{f}$ holds follows immediately by the definition of $Free^{X}_{!}$ and $\phi^{f}$, indeed:
\[
\left(Free^{X}_{!}\Comp \phi^{f}\right)(x) = \phi^{f}\left((x)\right) = f(x).
\]
So it only remains to show uniqueness. Assume that $\psi$ also satisfies $Free^{X}_{!}\Comp \psi = f$. The claim now follows because $\psi$ is a morphism of monoids, indeed: Since $\psi$ is morphism is monoids, we have that it preserves the multiplication, but the multiplication is given by concatenation, hence, we have that $\psi$ is uniquely determined by the images of the sequences of length $1$ (and length $0$, but this sequence of length $0$ should be mapped under $\psi$ to $e$). But a sequence of length $1$ is of the form $(x) = Free^{X}_{!}(x)$. So the claim indeed follows by the following computation: 
\[
\psi((x)) = \psi(Free^{X}_{!}(x)) = f(x) = \phi^{f}(Free^{X}_{!}(x)) = \phi^{f}((x)).
\]
\end{proof}
\end{prop}

The following exercise shows that there is a special connection between the forgetful functor $\mathit{forget}: \MON\to\SET$ and the free functor $\mathit{free}: \SET\to\MON$. This connection expresses that these form a so-called \textit{adjoint pair} (see \cref{sec:adjunctions}).
\begin{exer}\label{exer:preadjunction_monset}
Show that for any set $X$ and monoid $(M,m,e)$, there exist bijections between the hom-sets:
\[
\alpha_{X}^{(M,m,e)} : \CHom{\MON}{Free(X)}{(M,m,e)} \to \CHom{\SET}{X}{forget(M,m,e)}.
\]
Hint: use \cref{prop:UVP_forget_montoset}.
\end{exer}

\begin{exer} Define a forgetful functor from the category $\POS$ of posets (defined in \cref{example:poset}) to $\SET$ analogous to the the forgetful functor from $\MON$ to $\SET$ (defined in \cref{example:forgetful_montoset}).
\end{exer}
\begin{rem} The story about free monoids can not be repeated for posets, i.e. there is no free poset structure on all sets. But in order to prove this one needs more machinery.
\end{rem}


%%% Local Variables:
%%% mode: latex
%%% TeX-master: "CT4P"
%%% End:
